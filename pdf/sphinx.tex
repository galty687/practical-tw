%% Generated by Sphinx.
\def\sphinxdocclass{report}
\documentclass[letterpaper,10pt,english]{sphinxmanual}
\ifdefined\pdfpxdimen
   \let\sphinxpxdimen\pdfpxdimen\else\newdimen\sphinxpxdimen
\fi \sphinxpxdimen=.75bp\relax
\ifdefined\pdfimageresolution
    \pdfimageresolution= \numexpr \dimexpr1in\relax/\sphinxpxdimen\relax
\fi
%% let collapsible pdf bookmarks panel have high depth per default
\PassOptionsToPackage{bookmarksdepth=5}{hyperref}
%% turn off hyperref patch of \index as sphinx.xdy xindy module takes care of
%% suitable \hyperpage mark-up, working around hyperref-xindy incompatibility
\PassOptionsToPackage{hyperindex=false}{hyperref}
%% memoir class requires extra handling
\makeatletter\@ifclassloaded{memoir}
{\ifdefined\memhyperindexfalse\memhyperindexfalse\fi}{}\makeatother

\PassOptionsToPackage{booktabs}{sphinx}
\PassOptionsToPackage{colorrows}{sphinx}

\PassOptionsToPackage{warn}{textcomp}

\catcode`^^^^00a0\active\protected\def^^^^00a0{\leavevmode\nobreak\ }
\usepackage{cmap}
\usepackage{xeCJK}
\usepackage{amsmath,amssymb,amstext}
\usepackage{babel}



\setmainfont{FreeSerif}[
  Extension      = .otf,
  UprightFont    = *,
  ItalicFont     = *Italic,
  BoldFont       = *Bold,
  BoldItalicFont = *BoldItalic
]
\setsansfont{FreeSans}[
  Extension      = .otf,
  UprightFont    = *,
  ItalicFont     = *Oblique,
  BoldFont       = *Bold,
  BoldItalicFont = *BoldOblique,
]
\setmonofont{FreeMono}[
  Extension      = .otf,
  UprightFont    = *,
  ItalicFont     = *Oblique,
  BoldFont       = *Bold,
  BoldItalicFont = *BoldOblique,
]



\usepackage[Sonny]{fncychap}
\ChNameVar{\Large\normalfont\sffamily}
\ChTitleVar{\Large\normalfont\sffamily}
\usepackage{sphinx}

\fvset{fontsize=\small,formatcom=\xeCJKVerbAddon}
\usepackage{geometry}


% Include hyperref last.
\usepackage{hyperref}
% Fix anchor placement for figures with captions.
\usepackage{hypcap}% it must be loaded after hyperref.
% Set up styles of URL: it should be placed after hyperref.
\urlstyle{same}

\addto\captionsenglish{\renewcommand{\contentsname}{研究介绍}}

\usepackage{sphinxmessages}
\setcounter{tocdepth}{0}



\title{技术写作实用教程}
\date{2024 年 12 月 24 日}
\release{}
\author{Zhijun}
\newcommand{\sphinxlogo}{\vbox{}}
\renewcommand{\releasename}{}
\makeindex
\begin{document}

\ifdefined\shorthandoff
  \ifnum\catcode`\=\string=\active\shorthandoff{=}\fi
  \ifnum\catcode`\"=\active\shorthandoff{"}\fi
\fi

\pagestyle{empty}
\sphinxmaketitle
\pagestyle{plain}
\sphinxtableofcontents
\pagestyle{normal}
\phantomsection\label{\detokenize{index::doc}}


\sphinxstepscope


\chapter{本书介绍}
\label{\detokenize{about/about:id1}}\label{\detokenize{about/about::doc}}
\sphinxAtStartPar
本书定位为快速入门书,让同学们在没有完整掌握各类原理和技术之前,就已经可以上手进行文档开发的工作,并在入门后能在做种选,逐步完善知识结构体系。

\sphinxstepscope


\chapter{技术写作从业能力要求(初级)}
\label{\detokenize{about/tw-comp-model:tw-comp-model}}\label{\detokenize{about/tw-comp-model:id1}}\label{\detokenize{about/tw-comp-model::doc}}\begin{enumerate}
\sphinxsetlistlabels{\arabic}{enumi}{enumii}{}{.}%
\item {} 
\sphinxAtStartPar
技术写作行业

\item {} 
\sphinxAtStartPar
用户分析

\sphinxAtStartPar
(1)  虚拟角色

\sphinxAtStartPar
(2)  用户旅途

\item {} 
\sphinxAtStartPar
内容设计

\sphinxAtStartPar
(1) 最小化文档设计

\sphinxAtStartPar
(2) 信息模块化设计(或:Topic\sphinxhyphen{}based Writing)

\sphinxAtStartPar
(3) 单源设计

\item {} 
\sphinxAtStartPar
内容开发(技术部分)

\sphinxAtStartPar
(1) DITA 的信息分类架构(Concept, Task, Reference, Troubleshooting)

\sphinxAtStartPar
(2) 内容的组织(Linking)

\sphinxAtStartPar
(3) 样式定制

\sphinxAtStartPar
(4) 协同开发

\item {} 
\sphinxAtStartPar
内容开发(语言部分)

\sphinxAtStartPar
(1) 写作风格

\sphinxAtStartPar
(2) 受限语言(ASD\sphinxhyphen{}STE)

\item {} 
\sphinxAtStartPar
内容测试

\sphinxAtStartPar
(1) 语言质量检测(Language Tool, Vale等)

\sphinxAtStartPar
(2) 文档功能测试(超链接跳转、无乱码、图片正常显示)

\sphinxAtStartPar
(3) 可用性测试

\end{enumerate}

\sphinxAtStartPar
各项能力的具体要求如下:


\bigskip\hrule\bigskip



\section{技术写作行业}
\label{\detokenize{about/tw-comp-model:id2}}\begin{enumerate}
\sphinxsetlistlabels{\arabic}{enumi}{enumii}{}{.}%
\item {} 
\sphinxAtStartPar
知晓技术写作能力在行业中的职位;

\item {} 
\sphinxAtStartPar
知晓技术写作工程师所需要具备的能力;

\item {} 
\sphinxAtStartPar
世界范围内的行业知名协会、机构与组织;

\end{enumerate}


\section{用户分析}
\label{\detokenize{about/tw-comp-model:id3}}\begin{enumerate}
\sphinxsetlistlabels{\arabic}{enumi}{enumii}{}{.}%
\item {} 
\sphinxAtStartPar
知道用户分析在技术写作过程中所发挥的作用;

\item {} 
\sphinxAtStartPar
掌握常见的用研方法,如“卡片分类”法;

\end{enumerate}


\subsection{虚拟角色}
\label{\detokenize{about/tw-comp-model:id4}}\begin{enumerate}
\sphinxsetlistlabels{\arabic}{enumi}{enumii}{}{.}%
\item {} 
\sphinxAtStartPar
掌握虚拟角色的主要概念和方法;

\item {} 
\sphinxAtStartPar
能将用研的结果以用户画像的方式输出;

\end{enumerate}


\subsection{用户旅途地图}
\label{\detokenize{about/tw-comp-model:id5}}\begin{enumerate}
\sphinxsetlistlabels{\arabic}{enumi}{enumii}{}{.}%
\item {} 
\sphinxAtStartPar
掌握用户旅途的主要概念和方法;

\item {} 
\sphinxAtStartPar
能将用户使用信息的过程以用户旅途地图的方式的方式输出;

\end{enumerate}


\section{内容设计}
\label{\detokenize{about/tw-comp-model:id6}}\begin{enumerate}
\sphinxsetlistlabels{\arabic}{enumi}{enumii}{}{.}%
\item {} 
\sphinxAtStartPar
知道设计环节在技术写作过程中发挥的作用,即针对特定对象进行内容的设计,让内容易读易懂易用;

\item {} 
\sphinxAtStartPar
知道常用的设计方法,如”以用户为中心的设计”、”包容性设计“等;

\end{enumerate}


\subsection{最小化文档设计}
\label{\detokenize{about/tw-comp-model:id7}}\begin{enumerate}
\sphinxsetlistlabels{\arabic}{enumi}{enumii}{}{.}%
\item {} 
\sphinxAtStartPar
知道最小化文档设计方法的历史沿革与主要的理论贡献者;

\item {} 
\sphinxAtStartPar
掌握四个基本原则及其原则下的常见设计启发

\item {} 
\sphinxAtStartPar
能将其用于技术文档写作中;

\end{enumerate}


\subsection{信息模块化设计}
\label{\detokenize{about/tw-comp-model:id8}}\begin{enumerate}
\sphinxsetlistlabels{\arabic}{enumi}{enumii}{}{.}%
\item {} 
\sphinxAtStartPar
掌握该方法的基本概念,即将信息主题化 (Topic),然后以Topic为单位,进行内容的写作;

\item {} 
\sphinxAtStartPar
知晓常见的信息模块化流派(Information Mapping, DITA等)

\end{enumerate}


\subsection{单源设计 (Single Sourcing)}
\label{\detokenize{about/tw-comp-model:single-sourcing}}\begin{enumerate}
\sphinxsetlistlabels{\arabic}{enumi}{enumii}{}{.}%
\item {} 
\sphinxAtStartPar
知晓其核心概念,即做内容开发的时候,确保所有的信息仅有唯一的来源,确保信息的一致性;

\item {} 
\sphinxAtStartPar
知晓常见的单源开发方法(例如dita中使用引用、交叉引用、变量等形式确保内容的单源)

\end{enumerate}


\section{内容开发(技术部分)}
\label{\detokenize{about/tw-comp-model:id9}}

\subsection{DITA 的信息分类架构}
\label{\detokenize{about/tw-comp-model:dita}}\begin{enumerate}
\sphinxsetlistlabels{\arabic}{enumi}{enumii}{}{.}%
\item {} 
\sphinxAtStartPar
掌握业界常用的DITA技术,并能写作各类主题;

\item {} 
\sphinxAtStartPar
掌握 Concept, Task, Reference, Troubleshooting 主题的常见标记集

\end{enumerate}


\subsection{DITA 内容的组织}
\label{\detokenize{about/tw-comp-model:id10}}\begin{enumerate}
\sphinxsetlistlabels{\arabic}{enumi}{enumii}{}{.}%
\item {} 
\sphinxAtStartPar
能将各主题进行链接,形成有组织的内容;

\item {} 
\sphinxAtStartPar
掌握简单的内容重用与复用方法

\end{enumerate}


\subsection{内容发布}
\label{\detokenize{about/tw-comp-model:id11}}\begin{enumerate}
\sphinxsetlistlabels{\arabic}{enumi}{enumii}{}{.}%
\item {} 
\sphinxAtStartPar
知晓常见的交付格式,如PDF、WebHelp或Epub等;

\item {} 
\sphinxAtStartPar
能将内容发布为常见交付格式,并在发布前能输入参数调整输出结果;

\end{enumerate}


\subsection{样式定制}
\label{\detokenize{about/tw-comp-model:id12}}\begin{enumerate}
\sphinxsetlistlabels{\arabic}{enumi}{enumii}{}{.}%
\item {} 
\sphinxAtStartPar
能掌握样式表的基本知识,对输出物格式进行微调;

\item {} 
\sphinxAtStartPar
能读懂 CSS 和 XSL\sphinxhyphen{}FO并能修改参数;

\end{enumerate}


\subsection{协同开发}
\label{\detokenize{about/tw-comp-model:id13}}\begin{enumerate}
\sphinxsetlistlabels{\arabic}{enumi}{enumii}{}{.}%
\item {} 
\sphinxAtStartPar
能基于CCMS 或 WebDav/Git/SVN等技术进行协同写作;

\end{enumerate}


\section{内容开发(语言部分)}
\label{\detokenize{about/tw-comp-model:id14}}\begin{enumerate}
\sphinxsetlistlabels{\arabic}{enumi}{enumii}{}{.}%
\item {} 
\sphinxAtStartPar
知道写作风格的作用,熟悉科技公司技术语言写作的基本要求;

\item {} 
\sphinxAtStartPar
知晓行业常见的受限语言的规则,如ASD\sphinxhyphen{}STE等;

\end{enumerate}


\section{内容测试}
\label{\detokenize{about/tw-comp-model:id15}}

\subsection{语言质量测试}
\label{\detokenize{about/tw-comp-model:id16}}
\sphinxAtStartPar
能使用文档质量检测内容的语言质量;
能创建规则识别简单错误;


\subsection{文档功能测试}
\label{\detokenize{about/tw-comp-model:id17}}\begin{enumerate}
\sphinxsetlistlabels{\arabic}{enumi}{enumii}{}{.}%
\item {} 
\sphinxAtStartPar
知道常见输出物功能(目录、超链接、索引);

\item {} 
\sphinxAtStartPar
能使用工具自动化测试文档的功能能否正常运行;

\end{enumerate}


\subsection{可用性测试}
\label{\detokenize{about/tw-comp-model:id18}}\begin{enumerate}
\sphinxsetlistlabels{\arabic}{enumi}{enumii}{}{.}%
\item {} 
\sphinxAtStartPar
知晓可用性的测试目标;

\item {} 
\sphinxAtStartPar
能设计简易可用性实验;

\item {} 
\sphinxAtStartPar
能对实验结果进行分析,并基于数据改进文档质量;

\end{enumerate}

\sphinxstepscope


\chapter{课程安排}
\label{\detokenize{about/syllabus-cat:id1}}\label{\detokenize{about/syllabus-cat::doc}}

\begin{savenotes}\sphinxattablestart
\sphinxthistablewithglobalstyle
\centering
\begin{tabulary}{\linewidth}[t]{TTT}
\sphinxtoprule
\sphinxstyletheadfamily 
\sphinxAtStartPar
序号
&\sphinxstyletheadfamily 
\sphinxAtStartPar
课程
&\sphinxstyletheadfamily 
\sphinxAtStartPar
内容
\\
\sphinxmidrule
\sphinxtableatstartofbodyhook
\sphinxAtStartPar
1
&
\sphinxAtStartPar
技术写作导论
&
\sphinxAtStartPar

\\
\sphinxhline
\sphinxAtStartPar
2
&
\sphinxAtStartPar
用户研究
&
\sphinxAtStartPar
虚拟角色与用户旅途
\\
\sphinxhline
\sphinxAtStartPar
3
&
\sphinxAtStartPar
用户任务分析
&
\sphinxAtStartPar
对用户需要执行的任务进行拆解
\\
\sphinxhline
\sphinxAtStartPar
4
&
\sphinxAtStartPar
结构化与模块化写作
&
\sphinxAtStartPar
介绍常见写作任务的结构
\\
\sphinxhline
\sphinxAtStartPar
5
&
\sphinxAtStartPar
写作技术基础
&
\sphinxAtStartPar
XML、CSS (XSLT)、DTD等
\\
\sphinxhline
\sphinxAtStartPar
6
&
\sphinxAtStartPar
DITA 入门
&
\sphinxAtStartPar
Concept、Task、Troubleshooting和Reference
\\
\sphinxhline
\sphinxAtStartPar
7
&
\sphinxAtStartPar
oXygen XML
&
\sphinxAtStartPar
使用oXygen XML 写作DITA并发布
\\
\sphinxhline
\sphinxAtStartPar
8
&
\sphinxAtStartPar
语言风格
&
\sphinxAtStartPar
技术文档写作时常见的语言风格要求
\\
\sphinxhline
\sphinxAtStartPar
9
&
\sphinxAtStartPar
文档版面设计
&
\sphinxAtStartPar
介绍C.R.A.P原则以及常见的排版知识
\\
\sphinxhline
\sphinxAtStartPar
10
&
\sphinxAtStartPar
使用CSS定制文档样式
&
\sphinxAtStartPar
定制PDF和WebHelp的输出外观
\\
\sphinxhline
\sphinxAtStartPar
11
&
\sphinxAtStartPar
技术文档质量评估
&
\sphinxAtStartPar
内容分析、可用性实验以及量表
\\
\sphinxbottomrule
\end{tabulary}
\sphinxtableafterendhook\par
\sphinxattableend\end{savenotes}


\section{课程项目}
\label{\detokenize{about/syllabus-cat:id2}}
\sphinxAtStartPar
使用DITA,撰写一份专业的Zotero文献管理工具的技术文档,鼓励大家提交至官网。
\begin{enumerate}
\sphinxsetlistlabels{\arabic}{enumi}{enumii}{}{.}%
\item {} 
\sphinxAtStartPar
写作对象:北京大学的师生

\item {} 
\sphinxAtStartPar
交付物:PDF和Webhelp

\end{enumerate}


\section{期末论文}
\label{\detokenize{about/syllabus-cat:id3}}\begin{enumerate}
\sphinxsetlistlabels{\arabic}{enumi}{enumii}{}{.}%
\item {} 
\sphinxAtStartPar
研究论文
\begin{quote}

\sphinxAtStartPar
需就某一问题做深入而具体的研究,例如“技术文档多长是较为适合的长度”。应能:
\begin{enumerate}
\sphinxsetlistlabels{\arabic}{enumii}{enumiii}{}{.}%
\item {} 
\sphinxAtStartPar
提出问题

\item {} 
\sphinxAtStartPar
文献综述(看前人是否有类似研究,是否解决了你提出的问题,或者你是否有更好的解决问题)

\item {} 
\sphinxAtStartPar
实验设计

\item {} 
\sphinxAtStartPar
验证问题的解决

\end{enumerate}

\sphinxAtStartPar
整体篇幅在5\sphinxhyphen{}10页纸左右,选题范围应与技术写作相关,感兴趣的同学可以自行选题,并按照Exploring Research (9th Edition) 一书中,Chapter 13 Writing a Research Proposal一章的介绍撰写proposal,写作语言中英文均可。质量较高的文章将推荐至期刊,并可能获得资助去参加国际会议。
\end{quote}

\item {} 
\sphinxAtStartPar
撰写教材
\begin{quote}

\sphinxAtStartPar
认领教材章节,贡献一部分内容
\end{quote}

\item {} 
\sphinxAtStartPar
技术报告
\begin{quote}

\sphinxAtStartPar
对当前热门与前沿的问题做综述
\end{quote}

\end{enumerate}

\sphinxstepscope


\chapter{技术写作}
\label{\detokenize{introduction/tw:id1}}\label{\detokenize{introduction/tw::doc}}
\sphinxAtStartPar
按照读者和目的的不同,写作可以分为:
\begin{enumerate}
\sphinxsetlistlabels{\arabic}{enumi}{enumii}{}{.}%
\item {} 
\sphinxAtStartPar
创意写作

\item {} 
\sphinxAtStartPar
文案写作

\item {} 
\sphinxAtStartPar
商务写作

\item {} 
\sphinxAtStartPar
技术写作

\end{enumerate}


\section{技术写作示例}
\label{\detokenize{introduction/tw:id2}}\begin{enumerate}
\sphinxsetlistlabels{\arabic}{enumi}{enumii}{}{.}%
\item {} 
\sphinxAtStartPar
\sphinxhref{https://help.aliyun.com}{阿里云文档}

\item {} 
\sphinxAtStartPar
\sphinxhref{http://docs.pingcap.com}{PingCAP 文档}

\end{enumerate}

\sphinxAtStartPar
\#\#为什么需要技术写作
\begin{enumerate}
\sphinxsetlistlabels{\arabic}{enumi}{enumii}{}{.}%
\item {} 
\sphinxAtStartPar
知识诅咒。越懂一件事越难跟别人解释;

\item {} 
\sphinxAtStartPar
解释说明是一个专门的能力,需要运用多种方法才能做到信息简明易懂;

\end{enumerate}


\section{技术写作的定义}
\label{\detokenize{introduction/tw:id3}}
\sphinxstepscope


\chapter{虚拟角色}
\label{\detokenize{user-research/persona:id1}}\label{\detokenize{user-research/persona::doc}}

\section{文学作品的虚拟角色}
\label{\detokenize{user-research/persona:id2}}
\sphinxAtStartPar
塑造成功的人物角色,虽然现实生活中没有这样具体这样的人,但导出又都会有这样的任务身影,例如阿Q、祥林嫂。

\sphinxstepscope


\chapter{焦点小组}
\label{\detokenize{user-research/focus-group:id1}}\label{\detokenize{user-research/focus-group::doc}}\begin{itemize}
\item {} 
\sphinxAtStartPar
作者:汤佳琪

\item {} 
\sphinxAtStartPar
学号:2301212004

\item {} 
\sphinxAtStartPar
时间:2023.12.30

\end{itemize}


\section{焦点小组概述}
\label{\detokenize{user-research/focus-group:id2}}
\sphinxAtStartPar
焦点小组是一种常见的用户分析方法,通过组织目标用户进行小组讨论,以深入了解用户的需求、行为、态度和意见。在与目标用户进行深入交流的基础上,可以进一步了解其对于技术文档的具体需求和期望,收集关于文档内容、结构、可读性等方面的反馈,有助于明确文档的写作目的和方向,确保文档内容与用户需求紧密相关,从而优化文档设计,提高用户满意度。


\subsection{定义与背景}
\label{\detokenize{user-research/focus-group:id3}}

\subsubsection{焦点小组的概念}
\label{\detokenize{user-research/focus-group:id4}}
\sphinxAtStartPar
焦点小组是一种重要的定性研究方法。作为舶来品,焦点小组的中文译名亦存在不同的版本,包括焦点小组、焦点团体、焦点访谈、聚焦的访谈等。它通过召集一组目标用户,利用有组织的小组讨论,收集参与者对特定主题或产品的态度、感受、偏好等方面的信息,从而了解目标用户的需求、偏好以及对特定技术内容的反应。


\subsubsection{焦点小组的工作原理}
\label{\detokenize{user-research/focus-group:id5}}
\sphinxAtStartPar
1946年,默顿和坎德尔指出,焦点访谈的特性是研究者预先设计好主要的研究问题,在访谈提要的指导下,研究者组织被访者参与讨论,他们在访谈中表达自己的主观感受。

\sphinxAtStartPar
焦点小组应由技术作者或专业调研员主持,参与者包括来自目标用户群体的代表,讨论内容围绕文档的清晰度、可访问性、实用性等方面展开,通过对讨论数据的分析处理,得出用户需求、开发任务、架构设计、内容编写、文档测试、文档维护等方面的相关结论。

\sphinxAtStartPar
焦点小组一般由6到10位受访者组成,有线上(即时或非即时)和线下不同形式,时间通常为一到两个小时。这个规模既能保证多样性的观点,又足够小以便每个人都有机会发言。会议在录音或录像的环境中进行,以便后续分析。通过焦点小组,研究人员可以观察参与者之间的互动,捕捉他们的直接反馈和非言语信息,从而获得关于用户行为、偏好和感受的深入理解。


\subsubsection{焦点小组的历史沿革}
\label{\detokenize{user-research/focus-group:id6}}
\sphinxAtStartPar
焦点小组方法的历史可以追溯到20世纪20年代,但作为一种研究社会的系统方法,最终形成和发展于20世纪40年代。拉扎斯菲尔德领导的哥伦比亚大学应用社会研究所曾将其应用于受众研究。1946年,著名社会学家默顿和肯德尔在《美国社会学杂志》上发表专文,对焦点小组方法进行了系统论述。

\sphinxAtStartPar
后来,这种方法被引入市场研究、社会科学等领域,成为收集定性数据的主要手段之一。除了在商业市场研究中的广泛应用,焦点小组也被用于公共卫生、政策制定、教育改革等多个领域,展示了其作为一种强大的研究工具的多样性和适应性。


\subsection{作用与优势}
\label{\detokenize{user-research/focus-group:id7}}

\subsubsection{焦点小组的作用}
\label{\detokenize{user-research/focus-group:id8}}
\sphinxAtStartPar
焦点小组让研究者能够直接从目标用户群体中获得反馈,这是了解用户需求、测试产品概念或评估服务体验不可或缺的一环。在用户分析中,焦点小组可以:
\begin{enumerate}
\sphinxsetlistlabels{\arabic}{enumi}{enumii}{}{.}%
\item {} 
\sphinxAtStartPar
\sphinxstylestrong{捕获用户需求和偏好}:揭示用户对技术文档内容、结构和表现形式的具体需求和偏好,帮助开发者了解最受欢迎的信息形式(如文本、图表、视频)。

\item {} 
\sphinxAtStartPar
\sphinxstylestrong{识别使用中的问题和挑战}:通过参与者讨论,识别文档中难以理解的概念、不明确的信息等问题。

\item {} 
\sphinxAtStartPar
\sphinxstylestrong{提供改进建议}:参与者可以提出如何改善文档可读性、增加实用示例、改进导航结构等具体建议。

\item {} 
\sphinxAtStartPar
\sphinxstylestrong{测试新的概念和设计}:通过焦点小组测试新设计,评估其可行性和用户接受度。

\item {} 
\sphinxAtStartPar
\sphinxstylestrong{加深对用户行为的理解}:提供关于用户如何与技术文档互动的深入见解。

\item {} 
\sphinxAtStartPar
\sphinxstylestrong{建立用户参与}:通过直接参与改进过程,增强用户的归属感和满意度。

\end{enumerate}


\subsubsection{焦点小组的优势}
\label{\detokenize{user-research/focus-group:id9}}
\sphinxAtStartPar
焦点小组在收集深入的用户见解、理解用户行为和偏好方面展现出独特的优势:
\begin{enumerate}
\sphinxsetlistlabels{\arabic}{enumi}{enumii}{}{.}%
\item {} 
\sphinxAtStartPar
\sphinxstylestrong{建立深度了解}:促进丰富讨论,提供更深层次的见解。

\item {} 
\sphinxAtStartPar
\sphinxstylestrong{互动性强}:用户之间的互动能激发新的见解和创意。

\item {} 
\sphinxAtStartPar
\sphinxstylestrong{成本效率}:较少参与者和资源即可提供丰富数据。

\item {} 
\sphinxAtStartPar
\sphinxstylestrong{灵活性高}:可根据讨论进程调整话题和深度。

\item {} 
\sphinxAtStartPar
\sphinxstylestrong{促进用户积极参与和共创}:增强用户满意度,促进社区形成。

\item {} 
\sphinxAtStartPar
\sphinxstylestrong{动态反馈循环}:通过反复反馈和迭代改进,确保文档满足用户需求。

\item {} 
\sphinxAtStartPar
\sphinxstylestrong{提高决策质量}:在项目早期识别用户需求,避免后期昂贵修改。

\end{enumerate}

\sphinxAtStartPar
与其他方法相比:
\begin{itemize}
\item {} 
\sphinxAtStartPar
\sphinxstylestrong{与量化研究相比}:提供比问卷调查更深层次的洞见。

\item {} 
\sphinxAtStartPar
\sphinxstylestrong{与单一用户访谈相比}:通过小组互动揭示更多视角。

\item {} 
\sphinxAtStartPar
\sphinxstylestrong{与直接观察相比}:使用户的内在感受和想法得以表达。

\end{itemize}


\section{设计焦点小组研究}
\label{\detokenize{user-research/focus-group:id10}}

\subsection{选定研究主题}
\label{\detokenize{user-research/focus-group:id11}}
\sphinxAtStartPar
首先,需要明确焦点小组的研究目标。在进行技术文档的用户分析时,这可能是评估特定技术文档的有效性、理解用户在查找信息时遇到的挑战、或探索用户对新产品功能的文档需求。一旦目标确定,接下来是具体化研究问题。这些问题应该直接反映用户分析的核心关注点,比如:“用户在使用我们的安装指南时,最常遇到哪些问题?”或“新用户与经验丰富的用户在文档需求上有何不同?”

\sphinxAtStartPar
在进行研究主题选择时,要充分考虑以下关键因素:
\begin{enumerate}
\sphinxsetlistlabels{\arabic}{enumi}{enumii}{}{.}%
\item {} 
\sphinxAtStartPar
\sphinxstylestrong{用户相关性}:研究主题必须紧密关联目标用户群体的实际需求和问题。考虑用户的背景、经验水平和使用场景,确保研究主题能够涵盖他们最关心的方面,与用户分析目标一致。

\item {} 
\sphinxAtStartPar
\sphinxstylestrong{实际应用价值}:选择的研究主题应具有明确的实际应用价值,能够指导文档内容的改进或开发。主题应与技术文档的目标和业务目标保持一致,确保研究成果能够直接转化为改进措施。

\item {} 
\sphinxAtStartPar
\sphinxstylestrong{可行性和范围}:应当考虑研究的可行性,包括时间、资源和技术的限制。确保研究主题既不过于狭窄,导致结果缺乏足够的深度和广度,也不过于宽泛,使研究难以在有限的时间内完成。

\item {} 
\sphinxAtStartPar
\sphinxstylestrong{创新性和发展潜力}:探索研究主题是否能为技术文档领域带来新的见解和方法,是否有助于扩展现有的知识基础,是否能够为未来的文档策略和发展提供指导。

\end{enumerate}


\subsection{参与者招募与选择}
\label{\detokenize{user-research/focus-group:id12}}

\subsubsection{目标人群的确定}
\label{\detokenize{user-research/focus-group:id13}}
\sphinxAtStartPar
在策划焦点小组以进行技术文档的用户分析时,需要明确和选择合适的目标人群,以确保能够收集到代表性和多样化的用户反馈,从而提高技术文档的质量和适用性。
\begin{itemize}
\item {} 
\sphinxAtStartPar
\sphinxstylestrong{用户画像}:构建详细的用户画像,包括用户的基本人口统计特征(如年龄、性别、地理位置、教育背景)、技术背景,以及与技术文档相关的需求和使用习惯。

\item {} 
\sphinxAtStartPar
\sphinxstylestrong{多样性覆盖}:参与者的选拔应涵盖技术文档的主要使用情景,确保从多种使用背景中获取反馈。无论是初学者还是资深用户,其经验和视角都应在焦点小组中得到代表。

\item {} 
\sphinxAtStartPar
\sphinxstylestrong{小组规模}:焦点小组的理想人数通常为6到10人,这样的小组规模既可以保证讨论的深度,又能确保每位参与者都有机会表达自己的观点。

\end{itemize}


\subsubsection{如何招募适当的参与者}
\label{\detokenize{user-research/focus-group:id14}}\begin{enumerate}
\sphinxsetlistlabels{\arabic}{enumi}{enumii}{}{.}%
\item {} 
\sphinxAtStartPar
\sphinxstylestrong{招募渠道}:利用社交媒体、专业论坛、用户群组和邮件列表等多元渠道扩大招募范围。对于已有的产品或服务,现有客户数据库是一个宝贵的资源。

\item {} 
\sphinxAtStartPar
\sphinxstylestrong{激励措施}:提供适当的激励措施(如现金、礼品或其他形式的奖励)以鼓励参与,并表达对参与者时间和贡献的尊重。

\item {} 
\sphinxAtStartPar
\sphinxstylestrong{参与者筛选}:通过简短问卷或初步访谈评估潜在参与者是否符合研究需求。明确参与者的角色和期望,并确保他们了解研究内容和目的。

\item {} 
\sphinxAtStartPar
\sphinxstylestrong{抽样方法}:随机抽样和方便抽样是常用方法。随机抽样提高代表性,而方便抽样适用于资源有限的情况,但需注意样本偏差。

\end{enumerate}


\subsection{制定开放性问题}
\label{\detokenize{user-research/focus-group:id15}}
\sphinxAtStartPar
制定恰当的开放性问题能够激发深入讨论,帮助参与者分享经验和感受。以下是需要注意的要点:
\begin{itemize}
\item {} 
\sphinxAtStartPar
\sphinxstylestrong{设计开放性问题}:避免使用是非题或选择题,采用如“请描述一下您使用产品文档的体验”或“您觉得产品文档有哪些可以改进的地方”这样的开放式问题,鼓励参与者详细叙述经历和感受。

\item {} 
\sphinxAtStartPar
\sphinxstylestrong{促使反思}:问题设计应促使参与者反思,如“您在查找解决问题的信息时遇到了哪些困难?”或“您觉得产品文档中有哪些信息缺失或难以理解?”。

\item {} 
\sphinxAtStartPar
\sphinxstylestrong{通俗易懂}:避免使用专业术语或假设性语言,确保所有参与者都能理解问题。

\item {} 
\sphinxAtStartPar
\sphinxstylestrong{讨论引导}:设计讨论指南,从简单问题开始,逐步过渡到更深入、更具挑战性的议题,同时保持灵活性以适应实际情况。

\end{itemize}


\subsection{时间、地点与设备}
\label{\detokenize{user-research/focus-group:id16}}
\sphinxAtStartPar
时间和地点的选择对焦点小组研究的成功至关重要:
\begin{itemize}
\item {} 
\sphinxAtStartPar
\sphinxstylestrong{场地选择}:优先考虑私密性和舒适度。选择安静、不受干扰的场所(如圆桌会议室)进行讨论,确保参与者能够自由表达。

\item {} 
\sphinxAtStartPar
\sphinxstylestrong{时间安排}:选择一个方便大多数参与者的时间,考虑他们的日程安排和交通情况。时间长度应适中,避免冗长或仓促。

\item {} 
\sphinxAtStartPar
\sphinxstylestrong{设备准备}:准备可靠的录音或录像设备,确保讨论内容能够清晰记录。设置备份设备以防止故障,同时合理放置设备以避免干扰讨论氛围。

\end{itemize}

\sphinxAtStartPar
通过精心安排场地、设备和时间,可以为焦点小组研究创造良好的环境,确保参与者的便利性和舒适感。这种准备有助于提高研究的效度,为后续数据分析和洞察提供坚实基础。


\section{执行焦点小组研究}
\label{\detokenize{user-research/focus-group:id17}}

\subsection{主持人与记录员}
\label{\detokenize{user-research/focus-group:id18}}

\subsubsection{主持人的角色与责任}
\label{\detokenize{user-research/focus-group:id19}}
\sphinxAtStartPar
在焦点小组研究中,主持人(调研员)是关键人物,负责引导讨论、推进研究并确保讨论的效用最大化。
\begin{enumerate}
\sphinxsetlistlabels{\arabic}{enumi}{enumii}{}{.}%
\item {} 
\sphinxAtStartPar
\sphinxstylestrong{组织与主持讨论}:
\begin{itemize}
\item {} 
\sphinxAtStartPar
\sphinxstylestrong{开场介绍}:向参与者说明焦点小组的目的、流程、研究背景及匿名性保障,建立信任感和安全感。

\item {} 
\sphinxAtStartPar
\sphinxstylestrong{准备引导问题}:设计开放式、针对性强的问题以引发深入讨论,确保内容与研究目标一致。

\item {} 
\sphinxAtStartPar
\sphinxstylestrong{实时观察与记录}:密切关注参与者的表现,记录关键观点及群体动态变化。

\end{itemize}

\item {} 
\sphinxAtStartPar
\sphinxstylestrong{鼓励参与者积极发言}:
\begin{itemize}
\item {} 
\sphinxAtStartPar
\sphinxstylestrong{创造非评判性环境}:尊重每位参与者的观点,避免评判。

\item {} 
\sphinxAtStartPar
\sphinxstylestrong{使用探询式问题}:深入了解参与者的意见和理由,鼓励分享更多细节。

\item {} 
\sphinxAtStartPar
\sphinxstylestrong{平衡参与}:调控讨论动态,避免某些参与者过多占据时间。

\end{itemize}

\item {} 
\sphinxAtStartPar
\sphinxstylestrong{处理冲突与敏感问题}:
\begin{itemize}
\item {} 
\sphinxAtStartPar
\sphinxstylestrong{提前准备}:预测可能出现的冲突,制定应对策略。

\item {} 
\sphinxAtStartPar
\sphinxstylestrong{保持冷静和专业}:以适当语言和语气缓和紧张气氛。

\item {} 
\sphinxAtStartPar
\sphinxstylestrong{私下跟进}:在会后深入了解参与者的观点和感受。

\end{itemize}

\end{enumerate}


\subsubsection{记录员的角色与责任}
\label{\detokenize{user-research/focus-group:id20}}
\sphinxAtStartPar
记录员负责详细记录讨论内容,为后续分析和报告编写提供可靠依据。
\begin{enumerate}
\sphinxsetlistlabels{\arabic}{enumi}{enumii}{}{.}%
\item {} 
\sphinxAtStartPar
\sphinxstylestrong{详细记录讨论内容}:
\begin{itemize}
\item {} 
\sphinxAtStartPar
\sphinxstylestrong{准确记录}:记录言语内容、情感表达和非言语行为,确保信息完整性。

\item {} 
\sphinxAtStartPar
\sphinxstylestrong{识别关键信息}:区分主要观点与次要观点,记录群体动态变化。

\item {} 
\sphinxAtStartPar
\sphinxstylestrong{整理和分类信息}:将讨论内容归类,为分析和报告编写提供便利。

\end{itemize}

\item {} 
\sphinxAtStartPar
\sphinxstylestrong{技术支持与管理}:
\begin{itemize}
\item {} 
\sphinxAtStartPar
\sphinxstylestrong{熟悉设备操作}:掌握录音或录像设备的基本操作。

\item {} 
\sphinxAtStartPar
\sphinxstylestrong{检查设备状态}:确保设备工作正常,避免故障。

\item {} 
\sphinxAtStartPar
\sphinxstylestrong{妥善管理设备}:保障录音和录像的完整性与可靠性。

\end{itemize}

\end{enumerate}


\subsection{记录与分析}
\label{\detokenize{user-research/focus-group:id21}}

\subsubsection{有效的记录和整理方法}
\label{\detokenize{user-research/focus-group:id22}}\begin{enumerate}
\sphinxsetlistlabels{\arabic}{enumi}{enumii}{}{.}%
\item {} 
\sphinxAtStartPar
\sphinxstylestrong{录音与录像}:
\begin{itemize}
\item {} 
\sphinxAtStartPar
使用录音和录像设备记录完整讨论过程,捕捉言语和非言语信息。

\item {} 
\sphinxAtStartPar
在讨论前测试设备,确保电量充足和存储空间足够。

\end{itemize}

\item {} 
\sphinxAtStartPar
\sphinxstylestrong{笔记记录}:
\begin{itemize}
\item {} 
\sphinxAtStartPar
记录关键观点和情感表现,及时标注重要讨论话题。

\item {} 
\sphinxAtStartPar
提高信息的可读性和定位效率。

\end{itemize}

\item {} 
\sphinxAtStartPar
\sphinxstylestrong{数据整理}:
\begin{itemize}
\item {} 
\sphinxAtStartPar
转录录音内容,保持言语的原始性(包括停顿和重复)。

\item {} 
\sphinxAtStartPar
删除无关信息,确保数据质量和可读性。

\end{itemize}

\end{enumerate}


\subsubsection{如何有效分析数据}
\label{\detokenize{user-research/focus-group:id23}}\begin{enumerate}
\sphinxsetlistlabels{\arabic}{enumi}{enumii}{}{.}%
\item {} 
\sphinxAtStartPar
\sphinxstylestrong{定性数据分析}:
\begin{itemize}
\item {} 
\sphinxAtStartPar
\sphinxstylestrong{内容分析}:深入阅读转录文本,识别主题和模式。

\item {} 
\sphinxAtStartPar
\sphinxstylestrong{编码过程}:分类和编码主题,构建分析框架。

\end{itemize}

\item {} 
\sphinxAtStartPar
\sphinxstylestrong{使用分析工具}:
\begin{itemize}
\item {} 
\sphinxAtStartPar
\sphinxstylestrong{定性分析软件}:如 NVivo 和 Atlas.ti,提高数据管理和分析效率。

\item {} 
\sphinxAtStartPar
\sphinxstylestrong{视觉化工具}:利用思维导图或矩阵展现数据间的关系和模式。

\end{itemize}

\item {} 
\sphinxAtStartPar
\sphinxstylestrong{主题归纳}:
\begin{itemize}
\item {} 
\sphinxAtStartPar
\sphinxstylestrong{主题提取}:从编码结果中提取关键主题。

\item {} 
\sphinxAtStartPar
\sphinxstylestrong{关系建构}:分析主题间的互动和逻辑关系。

\end{itemize}

\item {} 
\sphinxAtStartPar
\sphinxstylestrong{结果解释}:
\begin{itemize}
\item {} 
\sphinxAtStartPar
\sphinxstylestrong{深入理解}:挖掘参与者观点和需求背后的原因。

\item {} 
\sphinxAtStartPar
\sphinxstylestrong{背景联系}:将分析结果与研究目标关联,确保实用性。

\end{itemize}

\item {} 
\sphinxAtStartPar
\sphinxstylestrong{保持反思性}:
\begin{itemize}
\item {} 
\sphinxAtStartPar
\sphinxstylestrong{研究者立场}:反思自身立场与偏见,保持客观性。

\item {} 
\sphinxAtStartPar
\sphinxstylestrong{参与者验证}:反馈分析结果给参与者,确保准确性和有效性。

\end{itemize}

\end{enumerate}


\section{数据应用}
\label{\detokenize{user-research/focus-group:id24}}

\subsection{洞察提炼}
\label{\detokenize{user-research/focus-group:id25}}
\sphinxAtStartPar
提炼关键用户洞察是焦点小组研究的重要环节。研究者通过对比分析和情境分析,寻找跨小组的共性与差异,并挖掘用户行为背后的动机和情境因素。同时,关注用户在讨论中表达的情感和态度,以揭示用户对特定问题或产品特性的深层感受。

\sphinxAtStartPar
研究者应明确识别与研究目标直接相关的核心洞察,并以具体证据支持这些洞察,例如参与者的直接话语或行为实例。核心洞察需要被转化为具体的策略建议,为优化用户体验提供指导。在研究报告中,清晰表达关键洞察并考虑建议的实际可行性,以确保建议能够被采纳和执行。


\subsection{报告撰写}
\label{\detokenize{user-research/focus-group:id26}}
\sphinxAtStartPar
撰写焦点小组研究报告的目的是总结研究发现并为决策提供依据。报告内容包括以下部分:
\begin{enumerate}
\sphinxsetlistlabels{\arabic}{enumi}{enumii}{}{.}%
\item {} 
\sphinxAtStartPar
\sphinxstylestrong{引言}:概述研究背景、目标和方法选择原因。

\item {} 
\sphinxAtStartPar
\sphinxstylestrong{方法论}:描述参与者选择、数据收集和处理方法。

\item {} 
\sphinxAtStartPar
\sphinxstylestrong{研究发现}:按主题组织内容,展示关键洞察,并通过参与者的话语引用支持这些洞察。

\item {} 
\sphinxAtStartPar
\sphinxstylestrong{讨论}:深入探讨研究发现的意义及其对产品开发、服务改进的应用。

\item {} 
\sphinxAtStartPar
\sphinxstylestrong{结论与建议}:总结研究的主要发现,并提出具体的行动建议。

\end{enumerate}

\sphinxAtStartPar
报告应使用图表和视觉工具突出重要信息,附录中可包括讨论指南、参与者信息和数据分析细节,以供读者评估研究质量和结论的可靠性。


\subsection{结果应用}
\label{\detokenize{user-research/focus-group:id27}}
\sphinxAtStartPar
焦点小组的研究结果可以应用于产品开发、策略规划和用户体验改进等方面。


\subsubsection{产品开发}
\label{\detokenize{user-research/focus-group:id28}}
\sphinxAtStartPar
焦点小组的反馈有助于:
\begin{itemize}
\item {} 
\sphinxAtStartPar
明确技术文档中的关键信息和需要改进的内容。

\item {} 
\sphinxAtStartPar
激发全新内容创意,为文档创新提供灵感。

\item {} 
\sphinxAtStartPar
采用用户驱动的设计方法,根据用户的视角和习惯优化文档结构、语言和示例。

\end{itemize}

\sphinxAtStartPar
通过紧密合作和反馈循环,确保技术文档能够满足用户需求并提升用户体验。


\subsubsection{策略规划}
\label{\detokenize{user-research/focus-group:id29}}
\sphinxAtStartPar
焦点小组结果可以帮助:
\begin{itemize}
\item {} 
\sphinxAtStartPar
界定技术文档的市场定位和内容策略。

\item {} 
\sphinxAtStartPar
理解用户的交互方式和使用挑战。

\item {} 
\sphinxAtStartPar
针对不同用户群体(如初学者或专业人士)定制内容,提升用户满意度和文档实用性。

\end{itemize}


\subsubsection{用户体验改进}
\label{\detokenize{user-research/focus-group:id30}}
\sphinxAtStartPar
通过用户反馈,研究者能够:
\begin{itemize}
\item {} 
\sphinxAtStartPar
确定技术文档的结构和内容优化方向。

\item {} 
\sphinxAtStartPar
解决用户提出的复杂或模糊问题。

\item {} 
\sphinxAtStartPar
创建真正符合用户需求的技术文档,提升其实用价值。

\end{itemize}


\subsubsection{实施和评估}
\label{\detokenize{user-research/focus-group:id31}}\begin{itemize}
\item {} 
\sphinxAtStartPar
\sphinxstylestrong{快速原型}:开发初步版本产品以测试用户反应。

\item {} 
\sphinxAtStartPar
\sphinxstylestrong{迭代开发}:根据用户反馈持续优化产品,解决问题或提升性能。

\item {} 
\sphinxAtStartPar
\sphinxstylestrong{效果评估}:量化改进效果,通过用户满意度、使用频率等指标衡量结果。

\end{itemize}

\sphinxAtStartPar
通过快速原型、迭代开发和效果评估,将焦点小组的洞察转化为实际改进,确保产品与用户需求一致,提高质量和市场表现。


\section{焦点小组的局限性与改进措施}
\label{\detokenize{user-research/focus-group:id32}}
\sphinxAtStartPar
焦点小组是一种有用的研究工具,但其局限性和适用范围需要被正确认识。虽然焦点小组能够提供具体的数据和丰富的见解,但并不能替代其他研究方法。研究者应明确如何合理运用焦点小组,以最大化其有效性。


\subsection{局限性}
\label{\detokenize{user-research/focus-group:id33}}

\subsubsection{主观性和偏差}
\label{\detokenize{user-research/focus-group:id34}}\begin{enumerate}
\sphinxsetlistlabels{\arabic}{enumi}{enumii}{}{.}%
\item {} 
\sphinxAtStartPar
\sphinxstylestrong{主持人影响}:主持人的态度、提问方式或反应可能会引导讨论方向,从而影响参与者的表达。这可能导致参与者表达出与其真实想法不符的观点。

\item {} 
\sphinxAtStartPar
\sphinxstylestrong{群体压力}:在小组环境中,参与者可能受到其他成员观点的影响,倾向于迎合多数意见,而不是表达真实想法。这种压力可能导致讨论结果缺乏多样性。

\end{enumerate}


\subsubsection{结果的代表性和泛化性}
\label{\detokenize{user-research/focus-group:id35}}\begin{enumerate}
\sphinxsetlistlabels{\arabic}{enumi}{enumii}{}{.}%
\item {} 
\sphinxAtStartPar
\sphinxstylestrong{样本限制}:由于焦点小组的参与者数量较少,结果可能无法完全代表更广泛的目标用户群体。

\item {} 
\sphinxAtStartPar
\sphinxstylestrong{情境限制}:焦点小组讨论中的社会期望效应可能导致参与者表达符合群体或社会期望的观点,而非真实感受,这限制了结果的泛化性。

\end{enumerate}


\subsection{改进措施}
\label{\detokenize{user-research/focus-group:id36}}

\subsubsection{减少主观性和偏差}
\label{\detokenize{user-research/focus-group:id37}}\begin{enumerate}
\sphinxsetlistlabels{\arabic}{enumi}{enumii}{}{.}%
\item {} 
\sphinxAtStartPar
\sphinxstylestrong{主持人培训}:确保主持人接受专业培训,以掌握中立的讨论引导技巧,避免自身偏见影响讨论。

\item {} 
\sphinxAtStartPar
\sphinxstylestrong{多方法结合}:将焦点小组与问卷调查或一对一访谈等方法结合,以验证和补充焦点小组的发现,从不同角度全面了解用户需求和行为。

\item {} 
\sphinxAtStartPar
\sphinxstylestrong{分析修正}:在分析结果时,注意识别并调整因群体压力或主持人偏见引起的偏差。

\end{enumerate}


\subsubsection{提高结果的代表性和泛化性}
\label{\detokenize{user-research/focus-group:id38}}\begin{enumerate}
\sphinxsetlistlabels{\arabic}{enumi}{enumii}{}{.}%
\item {} 
\sphinxAtStartPar
\sphinxstylestrong{多样化样本选择}:通过精心设计抽样策略,确保参与者群体在性别、年龄、背景等方面的多样性,以更全面地反映目标用户的需求。

\item {} 
\sphinxAtStartPar
\sphinxstylestrong{开展多组讨论}:在不同地点或针对不同目标群体开展多个焦点小组,以收集更多元化的观点。

\item {} 
\sphinxAtStartPar
\sphinxstylestrong{跨情境比较}:通过对不同背景下的讨论进行比较和分析,寻找共通问题和趋势,以提高结果的泛化性。

\end{enumerate}

\sphinxAtStartPar
焦点小组的有效性取决于研究设计和执行的质量。通过减少偏差、提高结果的代表性和泛化性,可以更好地发挥焦点小组的作用,为用户分析提供有力支持。


\section{注意事项与伦理考量}
\label{\detokenize{user-research/focus-group:id39}}
\sphinxAtStartPar
在用户分析过程中,遵循伦理原则是确保研究科学性和道德性的关键。随着技术的快速发展和用户数据的日益丰富,研究者需要特别关注隐私保护、知情同意等伦理要求。本部分探讨这些关键伦理原则,为用户研究提供有益指南,促进研究的健康发展。


\subsection{隐私保护和匿名性}
\label{\detokenize{user-research/focus-group:id40}}
\sphinxAtStartPar
隐私保护和匿名性是用户分析中的基本原则。
\begin{enumerate}
\sphinxsetlistlabels{\arabic}{enumi}{enumii}{}{.}%
\item {} 
\sphinxAtStartPar
\sphinxstylestrong{保护个人信息}:确保用户的敏感信息(如姓名、联系方式等)不会在研究中泄露。在报告和公开讨论中,使用化名或编号代替真实信息。

\item {} 
\sphinxAtStartPar
\sphinxstylestrong{遵循隐私法规}:严格遵守相关法律和伦理要求,确保数据的收集、处理和存储安全。

\item {} 
\sphinxAtStartPar
\sphinxstylestrong{限制数据访问}:敏感信息仅限授权人员访问,以防止未经授权的使用或滥用。

\end{enumerate}

\sphinxAtStartPar
通过以上措施,可以建立可信赖的研究环境,保护参与者隐私。


\subsection{知情同意}
\label{\detokenize{user-research/focus-group:id41}}
\sphinxAtStartPar
知情同意是确保研究伦理性的重要环节。
\begin{enumerate}
\sphinxsetlistlabels{\arabic}{enumi}{enumii}{}{.}%
\item {} 
\sphinxAtStartPar
\sphinxstylestrong{提供充分信息}:在研究开始前,向参与者清楚说明研究的目的、方法、潜在风险和权益,确保其能够做出明智决定。

\item {} 
\sphinxAtStartPar
\sphinxstylestrong{尊重选择权}:参与者有权随时退出研究,且其决定不应受到任何形式的压力或影响。

\item {} 
\sphinxAtStartPar
\sphinxstylestrong{确保自愿参与}:参与者需完全理解研究内容并自愿参与,以符合知情同意的要求。

\end{enumerate}

\sphinxAtStartPar
通过知情同意,能够建立公正和可信的研究环境。


\subsection{无偏见研究}
\label{\detokenize{user-research/focus-group:id42}}
\sphinxAtStartPar
无偏见研究是提升研究结果可信度的重要方式。
\begin{enumerate}
\sphinxsetlistlabels{\arabic}{enumi}{enumii}{}{.}%
\item {} 
\sphinxAtStartPar
\sphinxstylestrong{设计中立问题}:避免使用引导性问题,鼓励参与者自由表达真实观点。

\item {} 
\sphinxAtStartPar
\sphinxstylestrong{数据客观分析}:采用多种方法分析数据,消除主观偏见,确保结果的公正性。

\item {} 
\sphinxAtStartPar
\sphinxstylestrong{反思研究过程}:通过同行评审、重复实验和公开方法,持续审视可能导致偏见的因素,并进行必要的调整。

\end{enumerate}

\sphinxAtStartPar
无偏见研究有助于提高结果的准确性和公信力。


\subsection{伦理审查}
\label{\detokenize{user-research/focus-group:id43}}
\sphinxAtStartPar
伦理审查是确保研究符合伦理标准的重要步骤。
\begin{enumerate}
\sphinxsetlistlabels{\arabic}{enumi}{enumii}{}{.}%
\item {} 
\sphinxAtStartPar
\sphinxstylestrong{提交研究计划}:在研究开始前,向伦理审查委员会提交计划,接受多领域专家的评估和建议。

\item {} 
\sphinxAtStartPar
\sphinxstylestrong{关注敏感群体}:对涉及儿童或弱势群体的研究,需额外关注其权益和安全,采取保护措施。

\item {} 
\sphinxAtStartPar
\sphinxstylestrong{遵守指导意见}:研究过程中遵循伦理审查委员会的建议,包括数据处理、参与者招募、隐私保护等方面。

\end{enumerate}

\sphinxAtStartPar
通过伦理审查,可以确保研究活动科学且合乎道德,为参与者提供更高水平的保护。


\section{总结与展望}
\label{\detokenize{user-research/focus-group:id44}}

\subsection{总结}
\label{\detokenize{user-research/focus-group:id45}}
\sphinxAtStartPar
焦点小组作为用户分析的重要方法,以其互动式讨论形式,为研究者提供了深入了解用户感受、态度、需求和偏好的强有力工具。在用户分析中的关键作用主要体现在以下几个方面:
\begin{enumerate}
\sphinxsetlistlabels{\arabic}{enumi}{enumii}{}{.}%
\item {} 
\sphinxAtStartPar
\sphinxstylestrong{直接倾听用户声音}:焦点小组通过组织目标用户的小组讨论,让研究者能够直接听取用户的观点和需求。这种直接交流方式能够捕捉用户的真实感受,为产品开发、服务改进和策略制定提供可靠依据。

\item {} 
\sphinxAtStartPar
\sphinxstylestrong{挖掘细微差别和深层次见解}:通过用户之间的互动和分享,焦点小组能够激发新的观点和想法,帮助研究者发现用户的深层需求和关键偏好,这些信息通常在单独访谈或问卷调查中难以获取。

\item {} 
\sphinxAtStartPar
\sphinxstylestrong{比较不同用户群体的需求}:通过对不同背景、年龄和性别用户的讨论进行比较分析,焦点小组能够揭示不同用户群体的特点和差异,帮助企业更好地定位目标用户群体并优化产品和服务。

\end{enumerate}

\sphinxAtStartPar
综上所述,焦点小组在用户分析中具有不可替代的价值。通过合理运用焦点小组方法,研究者不仅能够提升研究质量,还能更好地满足用户需求,为企业的竞争力提供支持。


\subsection{未来发展方向}
\label{\detokenize{user-research/focus-group:id46}}
\sphinxAtStartPar
随着技术的进步和用户需求的变化,焦点小组方法也在不断演变。以下是其未来发展的主要方向:
\begin{enumerate}
\sphinxsetlistlabels{\arabic}{enumi}{enumii}{}{.}%
\item {} 
\sphinxAtStartPar
\sphinxstylestrong{技术赋能}:
\begin{itemize}
\item {} 
\sphinxAtStartPar
\sphinxstylestrong{虚拟现实技术}:未来可能利用虚拟现实技术创建更加沉浸式的讨论环境,增强参与者的体验感和讨论的真实性。

\item {} 
\sphinxAtStartPar
\sphinxstylestrong{人工智能辅助分析}:AI技术可用于自动化数据分析和洞察提炼,提高分析效率并减少主观偏差。

\end{itemize}

\item {} 
\sphinxAtStartPar
\sphinxstylestrong{线上焦点小组的普及}:
\begin{itemize}
\item {} 
\sphinxAtStartPar
远程工作和数字化交流的普及使线上焦点小组成为常态。这种形式能够打破地理限制,吸引更广泛的参与者。

\item {} 
\sphinxAtStartPar
利用数字工具增强交流的效率和深度,例如在线投票、实时数据可视化等功能,可以丰富讨论形式和内容。

\end{itemize}

\item {} 
\sphinxAtStartPar
\sphinxstylestrong{探索新的方法与工具}:
\begin{itemize}
\item {} 
\sphinxAtStartPar
面对不断变化的研究需求,研究者需要保持开放态度,持续探索新工具和方法,推动焦点小组方法的多元化发展。

\item {} 
\sphinxAtStartPar
在探索新技术的同时,研究者需要特别重视伦理和隐私保护,确保研究质量和参与者权益。

\end{itemize}

\end{enumerate}


\subsection{展望}
\label{\detokenize{user-research/focus-group:id47}}
\sphinxAtStartPar
焦点小组作为一种灵活且深具洞察力的研究方法,其未来将在技术赋能和应用场景多元化的推动下进一步发展。通过更先进的技术、更广泛的参与方式以及更深层的洞察提炼,焦点小组将为用户研究领域带来更多机遇和可能性。


\section{参考文献}
\label{\detokenize{user-research/focus-group:id48}}
\sphinxAtStartPar
{[}1{]} 王正辉. 焦点小组在博物馆观众评估领域的应用{[}D{]}. 山东大学, 2023. DOI: 10.27272/d.cnki.gshdu.2022.002428.\\
{[}2{]} 胡浩. 焦点小组访谈理论及其应用{[}J{]}. 现代商业, 2010, (26): 282. DOI: 10.14097/j.cnki.5392/2010.26.109.\\
{[}3{]} 肖维青, 刘禹辰. 视听翻译新热点: 中国无障碍电影研究——一项焦点小组访谈的调查{[}J{]}. 翻译研究, 2023, (01): 182\sphinxhyphen{}195.\\
{[}4{]} 孙龙. 焦点小组方法在组织研究中的应用{[}J{]}. 江苏行政学院学报, 2005, (03): 66\sphinxhyphen{}71.\\
{[}5{]} 风笑天. 社会研究方法{[}M{]}. 第五版. 北京: 中国人民大学出版社, 2018: 323.\\
{[}6{]} 王玲. 定性研究方法之焦点小组简析{[}J{]}. 戏剧之家, 2016, (13): 258\sphinxhyphen{}259.

\sphinxstepscope


\chapter{最小化文档设计}
\label{\detokenize{design/minimalism:id1}}\label{\detokenize{design/minimalism::doc}}

\section{传统文档}
\label{\detokenize{design/minimalism:id2}}
\sphinxAtStartPar
传统文档主要是使用系统化的方法,这种方法的基本思想:把一个复杂问题的求解过程分阶段进行,而且这种分解是自顶向下,逐层分解,使得每个阶段处理的问题都控制在人们容易理解和处理的范围内。这样开发出来的文档,常常比较厚,很多时候文档太长,用户无法阅读完。


\section{成年学习者}
\label{\detokenize{design/minimalism:id3}}\begin{enumerate}
\sphinxsetlistlabels{\arabic}{enumi}{enumii}{}{.}%
\item {} 
\sphinxAtStartPar
没有耐心,想尽快开始做一些有效率的事情

\item {} 
\sphinxAtStartPar
在手册和帮助文档中跳读,很少全读

\item {} 
\sphinxAtStartPar
犯错,但是修复错误的时候学得最多

\item {} 
\sphinxAtStartPar
自我探索时最有动力

\item {} 
\sphinxAtStartPar
巨型文档(将每个步骤分解为子步骤的方式)会打消而非鼓励积极性。

\end{enumerate}


\section{最小化文档设计原则}
\label{\detokenize{design/minimalism:id4}}

\subsection{1. Choose an action\sphinxhyphen{}oriented approach}
\label{\detokenize{design/minimalism:choose-an-action-oriented-approach}}
\sphinxAtStartPar
Heuristic 1.1 Provide an immediate opportunity to act
Heuristic 1.2 Encourage and support exploration and innovation
Heuristic 1.3 Respect the integrity of the user’s activity


\subsection{2. Anchor the tool in the task domain}
\label{\detokenize{design/minimalism:anchor-the-tool-in-the-task-domain}}
\sphinxAtStartPar
Heuristic 2.1 Select or design instructional activities that are real tasks
Heuristic 2.2 The components of the instruction should reflect the task structure


\subsection{3. Support Error Recognition and Recovery}
\label{\detokenize{design/minimalism:support-error-recognition-and-recovery}}
\sphinxAtStartPar
Heuristic 3.1 Prevent mistakes whenever possible
Heuristic 3.2 Provide error information when actions are error prone or when correction is difficult
Heuristic 3.3 Provide error information that supports detection, diagnosis, and recovery
Heuristic 3.4 Provide on\sphinxhyphen{}the\sphinxhyphen{}spot error information


\subsection{4 Support reading to do, study and locate}
\label{\detokenize{design/minimalism:support-reading-to-do-study-and-locate}}
\sphinxAtStartPar
Heuristic 4.1 Be brief; don’t spell out everything
Heuristic 4.2 Provide closure for chapters

\begin{sphinxadmonition}{tip}{小技巧:}
\sphinxAtStartPar
屯特大学极简主义详细介绍

\sphinxAtStartPar
更多信息请访问 Hans Van de Meij教授的 \sphinxhref{https://www.utwente.nl/en/bms/ist/minimalism/}{MINIMALISM} 介绍。
\end{sphinxadmonition}

\sphinxstepscope


\chapter{XML}
\label{\detokenize{XML/xml:xml}}\label{\detokenize{XML/xml::doc}}\begin{quote}\begin{description}
\sphinxlineitem{date}
\sphinxAtStartPar
2018\sphinxhyphen{}05\sphinxhyphen{}08

\sphinxlineitem{author}
\sphinxAtStartPar
高志军

\end{description}\end{quote}

\sphinxAtStartPar
XML是非常重要的存储数据的方式,作为技术写作从业人员,需要了解XML的基本技术,具体如下:
\begin{itemize}
\item {} 
\sphinxAtStartPar
XML

\item {} 
\sphinxAtStartPar
CSS/XSLT

\item {} 
\sphinxAtStartPar
DTD/Schema

\end{itemize}


\section{创建XML}
\label{\detokenize{XML/xml:id1}}\begin{quote}

\begin{sphinxVerbatim}[commandchars=\\\{\}]
\PYG{c+cp}{\PYGZlt{}?xml version=\PYGZdq{}1.0\PYGZdq{}?\PYGZgt{}}

\PYG{n+nt}{\PYGZlt{}BusinessCard}\PYG{n+nt}{\PYGZgt{}}
\PYG{+w}{    }\PYG{n+nt}{\PYGZlt{}Name}\PYG{n+nt}{\PYGZgt{}}Gao\PYG{+w}{ }Zhijun\PYG{n+nt}{\PYGZlt{}/Name\PYGZgt{}}
\PYG{+w}{    }\PYG{n+nt}{\PYGZlt{}phone}\PYG{+w}{ }\PYG{n+na}{type=}\PYG{l+s}{\PYGZdq{}mobile\PYGZdq{}}\PYG{n+nt}{\PYGZgt{}}+86\PYG{+w}{ }10\PYG{+w}{ }8264\PYG{+w}{ }9812\PYG{n+nt}{\PYGZlt{}/phone\PYGZgt{}}
\PYG{+w}{    }\PYG{n+nt}{\PYGZlt{}phone}\PYG{+w}{ }\PYG{n+na}{type=}\PYG{l+s}{\PYGZdq{}work\PYGZdq{}}\PYG{n+nt}{\PYGZgt{}}+86\PYG{+w}{ }10\PYG{+w}{ }6127\PYG{+w}{ }3510\PYG{n+nt}{\PYGZlt{}/phone\PYGZgt{}}
\PYG{+w}{    }\PYG{n+nt}{\PYGZlt{}phone}\PYG{+w}{ }\PYG{n+na}{type=}\PYG{l+s}{\PYGZdq{}fax\PYGZdq{}}\PYG{n+nt}{\PYGZgt{}}+86\PYG{+w}{ }10\PYG{+w}{ }6127\PYG{+w}{ }3510\PYG{n+nt}{\PYGZlt{}/phone\PYGZgt{}}
\PYG{+w}{    }\PYG{n+nt}{\PYGZlt{}email}\PYG{n+nt}{\PYGZgt{}}gaozhijun@ss.pku.edu.cn\PYG{n+nt}{\PYGZlt{}/email\PYGZgt{}}
\PYG{n+nt}{\PYGZlt{}/BusinessCard\PYGZgt{}}
\end{sphinxVerbatim}
\end{quote}


\section{创建CSS}
\label{\detokenize{XML/xml:css}}\begin{quote}

\begin{sphinxVerbatim}[commandchars=\\\{\}]
\PYG{n+nt}{BusinessCard}\PYG{+w}{ }\PYG{p}{\PYGZob{}}
\PYG{+w}{    }\PYG{k}{font\PYGZhy{}family}\PYG{p}{:}\PYG{+w}{ }\PYG{n}{Arial}\PYG{p}{,}\PYG{+w}{ }\PYG{n}{Helvetica}\PYG{p}{,}\PYG{+w}{ }\PYG{k+kc}{sans\PYGZhy{}serif}\PYG{p}{;}
\PYG{+w}{    }\PYG{k}{background\PYGZhy{}color}\PYG{p}{:}\PYG{+w}{ }\PYG{l+m+mh}{\PYGZsh{}DACFE5}\PYG{p}{;}
\PYG{+w}{    }\PYG{k}{width}\PYG{p}{:}\PYG{+w}{ }\PYG{l+m+mi}{300}\PYG{k+kt}{px}\PYG{p}{;}
\PYG{+w}{    }\PYG{k}{display}\PYG{p}{:}\PYG{+w}{ }\PYG{k+kc}{block}\PYG{p}{;}
\PYG{+w}{    }\PYG{k}{padding}\PYG{p}{:}\PYG{+w}{ }\PYG{l+m+mi}{10}\PYG{k+kt}{pt}\PYG{p}{;}
\PYG{+w}{    }\PYG{k}{border}\PYG{p}{:}\PYG{+w}{ }\PYG{l+m+mi}{1}\PYG{k+kt}{px}\PYG{+w}{ }\PYG{k+kc}{solid}\PYG{+w}{ }\PYG{l+m+mh}{\PYGZsh{}0D3427}\PYG{p}{;}
\PYG{+w}{    }\PYG{k}{margin}\PYG{p}{:}\PYG{+w}{ }\PYG{l+m+mi}{5}\PYG{k+kt}{px}\PYG{p}{;}
\PYG{+w}{    }\PYG{k}{text\PYGZhy{}align}\PYG{p}{:}\PYG{+w}{ }\PYG{k+kc}{left}\PYG{p}{;}
\PYG{p}{\PYGZcb{}}

\PYG{n+nt}{Name}\PYG{+w}{ }\PYG{p}{\PYGZob{}}
\PYG{+w}{    }\PYG{k}{color}\PYG{p}{:}\PYG{+w}{ }\PYG{l+m+mh}{\PYGZsh{}0D3427}\PYG{p}{;}
\PYG{+w}{    }\PYG{k}{font\PYGZhy{}weight}\PYG{p}{:}\PYG{+w}{ }\PYG{k+kc}{bold}\PYG{p}{;}
\PYG{+w}{    }\PYG{k}{font\PYGZhy{}size}\PYG{p}{:}\PYG{+w}{ }\PYG{l+m+mi}{140}\PYG{k+kt}{\PYGZpc{}}\PYG{p}{;}
\PYG{+w}{    }\PYG{k}{display}\PYG{p}{:}\PYG{+w}{ }\PYG{k+kc}{block}\PYG{p}{;}
\PYG{+w}{    }\PYG{k}{margin\PYGZhy{}bottom}\PYG{p}{:}\PYG{+w}{ }\PYG{l+m+mi}{3}\PYG{k+kt}{\PYGZpc{}}\PYG{p}{;}
\PYG{p}{\PYGZcb{}}

\PYG{n+nt}{phone}\PYG{+w}{ }\PYG{p}{\PYGZob{}}
\PYG{+w}{    }\PYG{k}{font\PYGZhy{}size}\PYG{p}{:}\PYG{+w}{ }\PYG{l+m+mi}{90}\PYG{k+kt}{\PYGZpc{}}\PYG{p}{;}
\PYG{+w}{    }\PYG{k}{color}\PYG{p}{:}\PYG{+w}{ }\PYG{l+m+mh}{\PYGZsh{}523819}\PYG{p}{;}
\PYG{+w}{    }\PYG{k}{font\PYGZhy{}size}\PYG{p}{:}\PYG{+w}{ }\PYG{l+m+mi}{90}\PYG{k+kt}{\PYGZpc{}}\PYG{p}{;}
\PYG{+w}{    }\PYG{k}{display}\PYG{p}{:}\PYG{+w}{ }\PYG{k+kc}{block}\PYG{p}{;}
\PYG{p}{\PYGZcb{}}

\PYG{n+nt}{email}\PYG{+w}{ }\PYG{p}{\PYGZob{}}
\PYG{+w}{    }\PYG{k}{color}\PYG{p}{:}\PYG{+w}{ }\PYG{l+m+mh}{\PYGZsh{}0D3427}\PYG{p}{;}
\PYG{+w}{    }\PYG{k}{font\PYGZhy{}size}\PYG{p}{:}\PYG{+w}{ }\PYG{l+m+mi}{90}\PYG{k+kt}{\PYGZpc{}}\PYG{p}{;}
\PYG{+w}{    }\PYG{k}{font\PYGZhy{}weight}\PYG{p}{:}\PYG{+w}{ }\PYG{k+kc}{bold}\PYG{p}{;}
\PYG{+w}{    }\PYG{k}{display}\PYG{p}{:}\PYG{+w}{ }\PYG{k+kc}{block}\PYG{p}{;}
\PYG{+w}{    }\PYG{k}{margin\PYGZhy{}top}\PYG{p}{:}\PYG{+w}{ }\PYG{l+m+mi}{3}\PYG{k+kt}{\PYGZpc{}}\PYG{p}{;}
\PYG{p}{\PYGZcb{}}
\end{sphinxVerbatim}
\end{quote}

\sphinxAtStartPar
关联CSS至XML

\begin{sphinxVerbatim}[commandchars=\\\{\}]
\PYGZlt{}?xml\PYGZhy{}stylesheet type=\PYGZdq{}text/css\PYGZdq{} href=\PYGZdq{}businesscard.css\PYGZdq{}?\PYGZgt{}
\end{sphinxVerbatim}


\section{创建页面内DTD}
\label{\detokenize{XML/xml:dtd}}\begin{quote}

\begin{sphinxVerbatim}[commandchars=\\\{\}]
\PYG{c+cp}{\PYGZlt{}!DOCTYPE BusinessCard [}
\PYG{c+cp}{\PYGZlt{}!ELEMENT BusinessCard (Name, phone+, email?)\PYGZgt{}}
\PYG{c+cp}{\PYGZlt{}!ELEMENT Name (\PYGZsh{}PCDATA)\PYGZgt{}}
\PYG{c+cp}{\PYGZlt{}!ELEMENT phone (\PYGZsh{}PCDATA)\PYGZgt{}}
\PYG{c+cp}{\PYGZlt{}!ATTLIST phone type (mobile | fax | Work | home) \PYGZsh{}REQUIRED\PYGZgt{}}

\PYG{c+cp}{\PYGZlt{}!ELEMENT emai (\PYGZsh{}PCDATA)\PYGZgt{}}

]\PYGZgt{}
\end{sphinxVerbatim}
\end{quote}

\sphinxAtStartPar
DTD 参考:\sphinxurl{https://www.w3cschool.cn/dtd/dtd-intro.html}

\begin{sphinxadmonition}{note}{备注:}\begin{itemize}
\item {} 
\sphinxAtStartPar
\#PCDATA (Parsed Character Data),简单解释就是元素内只有文本,没有子元素。

\end{itemize}
\end{sphinxadmonition}


\section{创建独立DTD并关联至xml}
\label{\detokenize{XML/xml:dtdxml}}
\begin{sphinxVerbatim}[commandchars=\\\{\}]
\PYGZlt{}!DOCTYPE BusinessCard SYSTEM \PYGZdq{}businesscard.dtd\PYGZdq{}\PYGZgt{}
\end{sphinxVerbatim}

\sphinxstepscope


\chapter{CSS 基本介绍}
\label{\detokenize{XML/css:css}}\label{\detokenize{XML/css::doc}}

\section{什么是CSS}
\label{\detokenize{XML/css:id1}}
\sphinxAtStartPar
样式语言用于内容的格式呈现。


\subsection{需要掌握的内容:}
\label{\detokenize{XML/css:id2}}

\subsection{单个}
\label{\detokenize{XML/css:id3}}\begin{enumerate}
\sphinxsetlistlabels{\arabic}{enumi}{enumii}{}{.}%
\item {} 
\sphinxAtStartPar
选择元素

\item {} 
\sphinxAtStartPar
选择类

\item {} 
\sphinxAtStartPar
选择 id

\end{enumerate}


\subsection{组合}
\label{\detokenize{XML/css:id4}}\begin{enumerate}
\sphinxsetlistlabels{\arabic}{enumi}{enumii}{}{.}%
\item {} 
\sphinxAtStartPar
.ancestor . child \{property: value\}

\item {} 
\sphinxAtStartPar
element.class child\sphinxhyphen{}element

\item {} 
\sphinxAtStartPar
element child\sphinxhyphen{}element

\item {} 
\end{enumerate}


\section{CSS语法}
\label{\detokenize{XML/css:id5}}
\sphinxAtStartPar
\sphinxcode{\sphinxupquote{Selector \{property1: value1; property2: value2\}}}


\subsection{选择元素}
\label{\detokenize{XML/css:id6}}
\begin{sphinxVerbatim}[commandchars=\\\{\}]
\PYG{n+nt}{selector}\PYG{+w}{ }\PYG{p}{\PYGZob{}}
\PYG{+w}{  }\PYG{n}{property1}\PYG{p}{:}\PYG{+w}{ }\PYG{n}{value1}\PYG{p}{;}
\PYG{+w}{  }\PYG{n}{property2}\PYG{p}{:}\PYG{+w}{ }\PYG{n}{value2}\PYG{p}{;}
\PYG{p}{\PYGZcb{}}
\end{sphinxVerbatim}

\sphinxAtStartPar
例如:

\begin{sphinxVerbatim}[commandchars=\\\{\}]
\PYG{n+nt}{h1}\PYG{+w}{ }\PYG{p}{\PYGZob{}}
\PYG{+w}{	}\PYG{k}{color}\PYG{p}{:}\PYG{+w}{ }\PYG{k+kc}{blue}\PYG{p}{;}
\PYG{p}{\PYGZcb{}}
\end{sphinxVerbatim}


\subsection{选择“类”}
\label{\detokenize{XML/css:id7}}
\begin{sphinxVerbatim}[commandchars=\\\{\}]
\PYG{p}{\PYGZlt{}}\PYG{n+nt}{h1} \PYG{n+na}{class}\PYG{o}{=}\PYG{l+s}{\PYGZdq{}big\PYGZhy{}header\PYGZdq{}}\PYG{p}{\PYGZgt{}}Title\PYG{p}{\PYGZlt{}}\PYG{p}{/}\PYG{n+nt}{h1}\PYG{p}{\PYGZgt{}}
\PYG{p}{\PYGZlt{}}\PYG{n+nt}{a} \PYG{n+na}{href}\PYG{o}{=}\PYG{l+s}{\PYGZdq{}index.html\PYGZdq{}}\PYG{p}{\PYGZgt{}}Home\PYG{p}{\PYGZlt{}}\PYG{p}{/}\PYG{n+nt}{a}\PYG{p}{\PYGZgt{}}
\end{sphinxVerbatim}

\sphinxAtStartPar
\sphinxcode{\sphinxupquote{.class\sphinxhyphen{}name \{property: value\}}}

\begin{sphinxVerbatim}[commandchars=\\\{\}]
\PYG{p}{\PYGZlt{}}\PYG{n+nt}{button} \PYG{n+na}{class}\PYG{o}{=}\PYG{l+s}{\PYGZdq{}btn btn\PYGZhy{}1\PYGZdq{}}\PYG{p}{\PYGZgt{}}
  button 1
\PYG{p}{\PYGZlt{}}\PYG{p}{/}\PYG{n+nt}{button}\PYG{p}{\PYGZgt{}}

\PYG{p}{\PYGZlt{}}\PYG{n+nt}{button} \PYG{n+na}{class}\PYG{o}{=}\PYG{l+s}{\PYGZdq{}btn btn\PYGZhy{}2\PYGZdq{}}\PYG{p}{\PYGZgt{}}
  button 2
\PYG{p}{\PYGZlt{}}\PYG{p}{/}\PYG{n+nt}{button}\PYG{p}{\PYGZgt{}}

\PYG{p}{\PYGZlt{}}\PYG{n+nt}{button} \PYG{n+na}{class}\PYG{o}{=}\PYG{l+s}{\PYGZdq{}btn btn\PYGZhy{}3\PYGZdq{}}\PYG{p}{\PYGZgt{}}
  button 3
\PYG{p}{\PYGZlt{}}\PYG{p}{/}\PYG{n+nt}{button}\PYG{p}{\PYGZgt{}}
\end{sphinxVerbatim}

\begin{sphinxVerbatim}[commandchars=\\\{\}]
\PYG{p}{.}\PYG{n+nc}{btn}\PYG{+w}{ }\PYG{p}{\PYGZob{}}
\PYG{+w}{  }\PYG{k}{padding}\PYG{p}{:}\PYG{+w}{ }\PYG{l+m+mi}{10}\PYG{k+kt}{px}\PYG{p}{;}
\PYG{+w}{  }\PYG{k}{color}\PYG{p}{:}\PYG{+w}{ }\PYG{k+kc}{white}\PYG{p}{;}
\PYG{p}{\PYGZcb{}}

\PYG{p}{.}\PYG{n+nc}{bth\PYGZhy{}1}\PYG{+w}{ }\PYG{p}{\PYGZob{}}\PYG{k}{background\PYGZhy{}color}\PYG{p}{:}\PYG{+w}{ }\PYG{k+kc}{green}\PYG{p}{\PYGZcb{}}
\PYG{p}{.}\PYG{n+nc}{bth\PYGZhy{}2}\PYG{+w}{ }\PYG{p}{\PYGZob{}}\PYG{k}{background\PYGZhy{}color}\PYG{p}{:}\PYG{+w}{ }\PYG{k+kc}{blue}\PYG{p}{\PYGZcb{}}
\PYG{p}{.}\PYG{n+nc}{bth\PYGZhy{}3}\PYG{+w}{ }\PYG{p}{\PYGZob{}}\PYG{k}{background\PYGZhy{}color}\PYG{p}{:}\PYG{+w}{ }\PYG{k+kc}{purple}\PYG{p}{\PYGZcb{}}
\end{sphinxVerbatim}


\subsection{选择 id}
\label{\detokenize{XML/css:id}}
\begin{sphinxVerbatim}[commandchars=\\\{\}]
\PYG{p}{\PYGZsh{}}\PYG{n+nn}{id}\PYG{+w}{ }\PYG{p}{\PYGZob{}}
\PYG{+w}{  }\PYG{n}{property}\PYG{p}{:}\PYG{+w}{ }\PYG{n}{value}\PYG{p}{;}
\PYG{p}{\PYGZcb{}}
\end{sphinxVerbatim}

\begin{sphinxVerbatim}[commandchars=\\\{\}]
\PYG{n+nt}{selector1}\PYG{p}{.}\PYG{n+nc}{selector\PYGZhy{}2}\PYG{+w}{ }\PYG{p}{\PYGZob{}}

\PYG{+w}{	}\PYG{n}{property}\PYG{p}{:}\PYG{+w}{ }\PYG{n}{value}\PYG{p}{;}

\PYG{p}{\PYGZcb{}}
\end{sphinxVerbatim}

\begin{sphinxVerbatim}[commandchars=\\\{\}]
\PYG{p}{\PYGZlt{}}\PYG{n+nt}{h1} \PYG{n+na}{class}\PYG{o}{=}\PYG{l+s}{\PYGZdq{}large\PYGZhy{}heading\PYGZdq{}}\PYG{p}{\PYGZgt{}}Title\PYG{p}{\PYGZlt{}}\PYG{p}{/}\PYG{n+nt}{h1}\PYG{p}{\PYGZgt{}}
\end{sphinxVerbatim}

\begin{sphinxVerbatim}[commandchars=\\\{\}]
\PYG{n+nt}{h1}\PYG{p}{.}\PYG{n+nc}{large\PYGZhy{}header}\PYG{+w}{ }\PYG{p}{\PYGZob{}}
\PYG{+w}{  }\PYG{k}{font\PYGZhy{}size}\PYG{p}{:}\PYG{+w}{ }\PYG{l+m+mi}{200}\PYG{k+kt}{\PYGZpc{}}
\PYG{p}{\PYGZcb{}}
\end{sphinxVerbatim}

\begin{sphinxVerbatim}[commandchars=\\\{\}]
\PYG{p}{\PYGZlt{}}\PYG{n+nt}{h2} \PYG{n+na}{id}\PYG{o}{=}\PYG{l+s}{\PYGZdq{}big\PYGZhy{}blue\PYGZdq{}} \PYG{n+na}{class}\PYG{o}{=}\PYG{l+s}{\PYGZdq{}large blue\PYGZdq{}}\PYG{p}{\PYGZgt{}}Title\PYG{p}{\PYGZlt{}}\PYG{p}{/}\PYG{n+nt}{h2}\PYG{p}{\PYGZgt{}}
\end{sphinxVerbatim}

\begin{sphinxVerbatim}[commandchars=\\\{\}]
\PYG{c+c1}{\PYGZsh{}big\PYGZhy{}blue.large.blue}
\end{sphinxVerbatim}

\sphinxAtStartPar
参考资源
\begin{itemize}
\item {} 
\sphinxAtStartPar
\sphinxhref{https://cssreference.io}{CSS Reference }

\item {} 
\sphinxAtStartPar
\sphinxhref{https://www.w3schools.com/colors/colors\_converter.asp}{Color Converter}

\item {} 
\sphinxAtStartPar
\sphinxhref{https://printcss.live/}{PrintCSS Playground}

\item {} 
\sphinxAtStartPar
\sphinxhref{https://developer.mozilla.org/en-US/docs/Learn/CSS/CSS\_layout/Introduction}{Introduction to CSS layout}

\end{itemize}

\sphinxstepscope


\chapter{DITA 快速入门}
\label{\detokenize{dita/dita-quick-demo:dita}}\label{\detokenize{dita/dita-quick-demo::doc}}

\section{DITA Topic 类型}
\label{\detokenize{dita/dita-quick-demo:dita-topic}}
\sphinxAtStartPar
DITA认为技术信息的基本构成单位是 Topic ,而 Topic 根据其使用目的又可以分为以下四类:
\begin{itemize}
\item {} 
\sphinxAtStartPar
Task。如何做某件事,例如,如何启动投影仪;

\item {} 
\sphinxAtStartPar
Concept。介绍某个概念或实物,例如,什么是VGA接口;

\item {} 
\sphinxAtStartPar
Reference。介绍具体参数信息,例如投影仪灯泡的规格、API接口等;

\item {} 
\sphinxAtStartPar
Troubleshooting。介绍如何排除故障,例如,无法启动投影仪怎么办;

\end{itemize}


\section{新建 Concept 主题}
\label{\detokenize{dita/dita-quick-demo:concept}}
\sphinxAtStartPar
Concept Topic 基本结构

\begin{sphinxVerbatim}[commandchars=\\\{\}]
        \PYG{o}{\PYGZhy{}} \PYG{n}{title}
          \PYG{o}{\PYGZhy{}} \PYG{n}{shortdesc}
          \PYG{o}{\PYGZhy{}} \PYG{n}{conbody}
        \PYG{o}{\PYGZhy{}} \PYG{n}{p} \PYG{p}{(}\PYG{n}{optional}\PYG{p}{)}
        \PYG{o}{\PYGZhy{}} \PYG{n}{section}
            \PYG{o}{\PYGZhy{}} \PYG{n}{title}
        \PYG{o}{\PYGZhy{}} \PYG{n}{p}
        \PYG{o}{\PYGZhy{}} \PYG{n}{image}
        \PYG{o}{\PYGZhy{}} \PYG{n}{codeblock}
        \PYG{o}{\PYGZhy{}} \PYG{n}{etc}\PYG{o}{.}
\end{sphinxVerbatim}

\sphinxAtStartPar
这里演示如何 使用oXygen XML 24 创建Concept.


\section{新建 Task 主题}
\label{\detokenize{dita/dita-quick-demo:task}}
\begin{sphinxVerbatim}[commandchars=\\\{\}]
\PYG{o}{\PYGZhy{}}\PYG{n}{task}
\PYG{o}{\PYGZhy{}}\PYG{n}{title}
\PYG{o}{\PYGZhy{}}\PYG{n}{shortdesc}
\PYG{o}{\PYGZhy{}} \PYG{n}{taskbody}
     \PYG{o}{\PYGZhy{}} \PYG{n}{prereq}
     \PYG{o}{\PYGZhy{}} \PYG{n}{context}
     \PYG{o}{\PYGZhy{}} \PYG{n}{stepsection}
     \PYG{o}{\PYGZhy{}} \PYG{n}{steps}
          \PYG{o}{\PYGZhy{}} \PYG{n}{step}
               \PYG{o}{\PYGZhy{}}\PYG{n}{cmd}
               \PYG{o}{\PYGZhy{}} \PYG{n}{info}
                    \PYG{o}{\PYGZhy{}} \PYG{n}{note}
                    \PYG{o}{\PYGZhy{}} \PYG{n}{image}
                    \PYG{o}{\PYGZhy{}} \PYG{n}{codeblock}
               \PYG{o}{\PYGZhy{}} \PYG{n}{stepxmp}
               \PYG{o}{\PYGZhy{}} \PYG{n}{stepresult}
               \PYG{o}{\PYGZhy{}} \PYG{n}{choices}
               \PYG{o}{\PYGZhy{}} \PYG{n}{substeps}
     \PYG{o}{\PYGZhy{}} \PYG{n}{result}
     \PYG{o}{\PYGZhy{}} \PYG{n}{example}
     \PYG{o}{\PYGZhy{}} \PYG{n}{postreq}
\end{sphinxVerbatim}


\section{新建 Reference 主题}
\label{\detokenize{dita/dita-quick-demo:reference}}

\section{新建 Troubleshooting 主题}
\label{\detokenize{dita/dita-quick-demo:troubleshooting}}

\section{使用 DITAMAP 创建输出}
\label{\detokenize{dita/dita-quick-demo:ditamap}}
\sphinxstepscope


\chapter{Troubleshooting}
\label{\detokenize{dita/Troubleshooting:troubleshooting}}\label{\detokenize{dita/Troubleshooting::doc}}

\section{故障}
\label{\detokenize{dita/Troubleshooting:id1}}
\sphinxAtStartPar
只要用户开始使用产品就会遇到故障,很多用户看手册是要寻找解决问题的信息。\sphinxhref{https://research.utwente.nl/en/publications/the-effect-of-error-information-in-tutorial-documentation}{根据 Lazonder 等人的研究},用户25\%的时间是在应对各类错误。


\section{故障排除信息}
\label{\detokenize{dita/Troubleshooting:id2}}

\subsection{结构}
\label{\detokenize{dita/Troubleshooting:id3}}

\begin{savenotes}\sphinxattablestart
\sphinxthistablewithglobalstyle
\centering
\begin{tabulary}{\linewidth}[t]{TTT}
\sphinxtoprule
\sphinxstyletheadfamily 
\sphinxAtStartPar
元素
&\sphinxstyletheadfamily 
\sphinxAtStartPar
目的
&\sphinxstyletheadfamily 
\sphinxAtStartPar
示例
\\
\sphinxmidrule
\sphinxtableatstartofbodyhook
\sphinxAtStartPar
故障识别
&
\sphinxAtStartPar
让用户知道错误的存在
&
\sphinxAtStartPar
DITA 输出中文乱码
\\
\sphinxhline
\sphinxAtStartPar
故障分析
&
\sphinxAtStartPar
描述错误的原因
&
\sphinxAtStartPar
未指定xml的语言为中文
\\
\sphinxhline
\sphinxAtStartPar
故障排除方法
&
\sphinxAtStartPar
提供解决故障的方式
&
\sphinxAtStartPar
在map元素中增加 xml:lang=”zh\_CN”
\\
\sphinxbottomrule
\end{tabulary}
\sphinxtableafterendhook\par
\sphinxattableend\end{savenotes}


\subsection{建模}
\label{\detokenize{dita/Troubleshooting:id4}}

\begin{savenotes}\sphinxattablestart
\sphinxthistablewithglobalstyle
\centering
\begin{tabulary}{\linewidth}[t]{TT}
\sphinxtoprule
\sphinxstyletheadfamily 
\sphinxAtStartPar
元素
&\sphinxstyletheadfamily 
\sphinxAtStartPar
含义
\\
\sphinxmidrule
\sphinxtableatstartofbodyhook
\sphinxAtStartPar
\sphinxcode{\sphinxupquote{<troubleshooting>}}
&
\sphinxAtStartPar

\\
\sphinxhline
\sphinxAtStartPar
\sphinxcode{\sphinxupquote{<title>}}
&
\sphinxAtStartPar
故障识别
\\
\sphinxhline
\sphinxAtStartPar
\sphinxcode{\sphinxupquote{<condition>}}
&
\sphinxAtStartPar
故障发生的条件
\\
\sphinxhline
\sphinxAtStartPar
\sphinxcode{\sphinxupquote{<cause>}}
&
\sphinxAtStartPar
诊断
\\
\sphinxhline
\sphinxAtStartPar
\sphinxcode{\sphinxupquote{<remedy>}}
&
\sphinxAtStartPar
故障排除方法
\\
\sphinxbottomrule
\end{tabulary}
\sphinxtableafterendhook\par
\sphinxattableend\end{savenotes}


\section{写作风格}
\label{\detokenize{dita/Troubleshooting:id5}}

\subsection{\sphinxstyleliteralintitle{\sphinxupquote{<title>}}}
\label{\detokenize{dita/Troubleshooting:title}}\begin{enumerate}
\sphinxsetlistlabels{\arabic}{enumi}{enumii}{}{.}%
\item {} 
\sphinxAtStartPar
使用用户最有可能描述问题的方式来写标题。

\item {} 
\sphinxAtStartPar
增加 \sphinxcode{\sphinxupquote{<indexterms>}}方便搜索。

\end{enumerate}


\section{示例}
\label{\detokenize{dita/Troubleshooting:id6}}
\sphinxAtStartPar
来自dita.xml.org的对各元素的说明

\begin{sphinxVerbatim}[commandchars=\\\{\}]
\PYG{c+cp}{\PYGZlt{}?xml version=\PYGZdq{}1.0\PYGZdq{} encoding=\PYGZdq{}utf\PYGZhy{}8\PYGZdq{}?\PYGZgt{}}
\PYG{c+cp}{\PYGZlt{}!DOCTYPE troubleshooting}
\PYG{c+cp}{  PUBLIC \PYGZdq{}\PYGZhy{}//OASIS//DTD DITA Troubleshooting//EN\PYGZdq{}}
\PYG{c+cp}{  \PYGZdq{}troubleshooting.dtd\PYGZdq{}\PYGZgt{}}
\PYG{n+nt}{\PYGZlt{}troubleshooting}\PYG{+w}{ }\PYG{n+na}{id=}\PYG{l+s}{\PYGZdq{}REPLACE\PYGZus{}THIS\PYGZdq{}}\PYG{n+nt}{\PYGZgt{}}
\PYG{+w}{  }\PYG{n+nt}{\PYGZlt{}title}\PYG{n+nt}{\PYGZgt{}}\PYG{c+cm}{\PYGZlt{}!\PYGZhy{}\PYGZhy{} State the essential nature of the problem \PYGZhy{}\PYGZhy{}\PYGZgt{}}\PYG{n+nt}{\PYGZlt{}/title\PYGZgt{}}
\PYG{+w}{  }\PYG{n+nt}{\PYGZlt{}shortdesc}\PYG{n+nt}{\PYGZgt{}}\PYG{c+cm}{\PYGZlt{}!\PYGZhy{}\PYGZhy{} Be more specific about the problem here \PYGZhy{}\PYGZhy{}\PYGZgt{}}\PYG{n+nt}{\PYGZlt{}/shortdesc\PYGZgt{}}
\PYG{+w}{  }\PYG{n+nt}{\PYGZlt{}prolog}\PYG{n+nt}{\PYGZgt{}}
\PYG{+w}{    }\PYG{n+nt}{\PYGZlt{}metadata}\PYG{n+nt}{\PYGZgt{}}
\PYG{+w}{      }\PYG{n+nt}{\PYGZlt{}keywords}\PYG{n+nt}{\PYGZgt{}}\PYG{c+cm}{\PYGZlt{}!\PYGZhy{}\PYGZhy{}}
\PYG{c+cm}{          Make sure to use indexterms. Readers often find troubleshooting content through the index.}
\PYG{c+cm}{          Provide indexterm variants that readers might use. For example:}
\PYG{c+cm}{        \PYGZlt{}indexterm\PYGZgt{}Logon}
\PYG{c+cm}{          \PYGZlt{}indexterm\PYGZgt{}troubleshooting\PYGZlt{}/indexterm\PYGZgt{}}
\PYG{c+cm}{        \PYGZlt{}/indexterm\PYGZgt{}}
\PYG{c+cm}{        \PYGZlt{}indexterm\PYGZgt{}troubleshooting}
\PYG{c+cm}{          \PYGZlt{}indexterm\PYGZgt{}Logon\PYGZlt{}/indexterm\PYGZgt{}}
\PYG{c+cm}{        \PYGZlt{}/indexterm\PYGZgt{}}
\PYG{c+cm}{        \PYGZhy{}\PYGZhy{}\PYGZgt{}}
\PYG{+w}{      }\PYG{n+nt}{\PYGZlt{}/keywords\PYGZgt{}}
\PYG{+w}{    }\PYG{n+nt}{\PYGZlt{}/metadata\PYGZgt{}}
\PYG{+w}{  }\PYG{n+nt}{\PYGZlt{}/prolog\PYGZgt{}}
\PYG{+w}{  }\PYG{n+nt}{\PYGZlt{}troublebody}\PYG{n+nt}{\PYGZgt{}}
\PYG{+w}{    }\PYG{n+nt}{\PYGZlt{}condition}\PYG{n+nt}{\PYGZgt{}}\PYG{c+cm}{\PYGZlt{}!\PYGZhy{}\PYGZhy{}}
\PYG{c+cm}{        The topic title and the shortdesc should have already told your reader a lot about the}
\PYG{c+cm}{        condition that this topic seeks to fix. Use condition to expand upon that. Do not simply}
\PYG{c+cm}{        repeat what you have already put into the title and the shortdesc. Condition is the}
\PYG{c+cm}{        appropriate place to put impact and severity information. If multiple solutions exist and}
\PYG{c+cm}{        their relationships with each other are complex, you can discuss that here.\PYGZhy{}\PYGZhy{}\PYGZgt{}}
\PYG{+w}{      }\PYG{n+nt}{\PYGZlt{}title}\PYG{n+nt}{\PYGZgt{}}\PYG{c+cm}{\PYGZlt{}!\PYGZhy{}\PYGZhy{} Optional title. Use \PYGZdq{}Condition\PYGZdq{}. \PYGZhy{}\PYGZhy{}\PYGZgt{}}\PYG{n+nt}{\PYGZlt{}/title\PYGZgt{}}
\PYG{+w}{      }\PYG{n+nt}{\PYGZlt{}p}\PYG{n+nt}{\PYGZgt{}}\PYG{n+nt}{\PYGZlt{}/p\PYGZgt{}}
\PYG{+w}{    }\PYG{n+nt}{\PYGZlt{}/condition\PYGZgt{}}
\PYG{+w}{    }\PYG{n+nt}{\PYGZlt{}troubleSolution}\PYG{n+nt}{\PYGZgt{}}\PYG{c+cm}{\PYGZlt{}!\PYGZhy{}\PYGZhy{}}
\PYG{c+cm}{        troubleSolution is meant to hold pairs of cause and remedy. Occassionally, you might have}
\PYG{c+cm}{        cause without remedy or remedy without cause, but that should be rare. Be sure to order}
\PYG{c+cm}{        mutliple troubleSolution elements in a sequence that makes sense. For example, order them}
\PYG{c+cm}{        by the likelihood of a cause occuring. You may wish to deviate from that if a remedy for a}
\PYG{c+cm}{        less likely cause is much easier to try. Remember to use conref for troubleSolution}
\PYG{c+cm}{        elements that are the same across multiple troubleshooting topics.\PYGZhy{}\PYGZhy{}\PYGZgt{}}
\PYG{+w}{      }\PYG{n+nt}{\PYGZlt{}cause}\PYG{n+nt}{\PYGZgt{}}
\PYG{+w}{        }\PYG{n+nt}{\PYGZlt{}title}\PYG{n+nt}{\PYGZgt{}}\PYG{c+cm}{\PYGZlt{}!\PYGZhy{}\PYGZhy{}}
\PYG{c+cm}{          Optional title. Use \PYGZdq{}Cause\PYGZdq{}. For topics with mutliple troubleSolutions, state the}
\PYG{c+cm}{          essential nature of this particular cause instead of just using \PYGZdq{}Cause\PYGZdq{}.\PYGZhy{}\PYGZhy{}\PYGZgt{}}
\PYG{+w}{        }\PYG{n+nt}{\PYGZlt{}/title\PYGZgt{}}
\PYG{+w}{        }\PYG{n+nt}{\PYGZlt{}p}\PYG{n+nt}{\PYGZgt{}}\PYG{n+nt}{\PYGZlt{}/p\PYGZgt{}}
\PYG{+w}{      }\PYG{n+nt}{\PYGZlt{}/cause\PYGZgt{}}
\PYG{+w}{      }\PYG{n+nt}{\PYGZlt{}remedy}\PYG{n+nt}{\PYGZgt{}}
\PYG{+w}{        }\PYG{n+nt}{\PYGZlt{}title}\PYG{n+nt}{\PYGZgt{}}\PYG{c+cm}{\PYGZlt{}!\PYGZhy{}\PYGZhy{} Optional title. Use \PYGZdq{}Remedy\PYGZdq{} or \PYGZdq{}Solution\PYGZdq{}. \PYGZhy{}\PYGZhy{}\PYGZgt{}}
\PYG{+w}{        }\PYG{n+nt}{\PYGZlt{}/title\PYGZgt{}}
\PYG{+w}{        }\PYG{n+nt}{\PYGZlt{}responsibleParty}\PYG{n+nt}{\PYGZgt{}}\PYG{c+cm}{\PYGZlt{}!\PYGZhy{}\PYGZhy{}}
\PYG{c+cm}{          Optional. Use this element to indicate the role of who ought to be performing the steps}
\PYG{c+cm}{          in the remedy. Here are some examples: \PYGZdq{}engineer\PYGZdq{}, \PYGZdq{}customer\PYGZdq{}, \PYGZdq{}field\PYGZhy{}support\PYGZdq{}.}
\PYG{c+cm}{        \PYGZhy{}\PYGZhy{}\PYGZgt{}}
\PYG{+w}{        }\PYG{n+nt}{\PYGZlt{}/responsibleParty\PYGZgt{}}
\PYG{+w}{        }\PYG{c+cm}{\PYGZlt{}!\PYGZhy{}\PYGZhy{} You can use steps\PYGZhy{}unordered or steps\PYGZhy{}informal instead of steps. \PYGZhy{}\PYGZhy{}\PYGZgt{}}
\PYG{+w}{        }\PYG{n+nt}{\PYGZlt{}steps}\PYG{n+nt}{\PYGZgt{}}
\PYG{+w}{          }\PYG{c+cm}{\PYGZlt{}!\PYGZhy{}\PYGZhy{} }
\PYG{c+cm}{               Remember that you can use conref to bring in steps from tasks or other}
\PYG{c+cm}{               troubleshooting topics. \PYGZhy{}\PYGZhy{}\PYGZgt{}}
\PYG{+w}{          }\PYG{n+nt}{\PYGZlt{}step}\PYG{n+nt}{\PYGZgt{}}
\PYG{+w}{            }\PYG{n+nt}{\PYGZlt{}cmd}\PYG{n+nt}{/\PYGZgt{}}
\PYG{+w}{          }\PYG{n+nt}{\PYGZlt{}/step\PYGZgt{}}
\PYG{+w}{        }\PYG{n+nt}{\PYGZlt{}/steps\PYGZgt{}}
\PYG{+w}{      }\PYG{n+nt}{\PYGZlt{}/remedy\PYGZgt{}}
\PYG{+w}{    }\PYG{n+nt}{\PYGZlt{}/troubleSolution\PYGZgt{}}
\PYG{+w}{  }\PYG{n+nt}{\PYGZlt{}/troublebody\PYGZgt{}}
\PYG{n+nt}{\PYGZlt{}/troubleshooting\PYGZgt{}}
\end{sphinxVerbatim}

\sphinxstepscope


\chapter{DITA 深入了解}
\label{\detokenize{dita/dita-dive-in:dita}}\label{\detokenize{dita/dita-dive-in::doc}}

\section{内容复用}
\label{\detokenize{dita/dita-dive-in:id1}}\begin{enumerate}
\sphinxsetlistlabels{\arabic}{enumi}{enumii}{}{.}%
\item {} 
\sphinxAtStartPar
将需要复用的文本片段存储再 \sphinxcode{\sphinxupquote{snippets.dita}}中

\item {} 
\sphinxAtStartPar
在需要复用的地方使用 \sphinxcode{\sphinxupquote{file name\#topic\_id/element\_id}}进行内容复用

\end{enumerate}

\sphinxAtStartPar
例如:

\sphinxAtStartPar
Snippets.dita 片段中的提示信息

\begin{sphinxVerbatim}[commandchars=\\\{\}]
\PYG{n+nt}{\PYGZlt{}topic}\PYG{+w}{ }\PYG{n+na}{id=}\PYG{l+s}{\PYGZdq{}topic\PYGZus{}slh\PYGZus{}bsm\PYGZus{}lrb\PYGZdq{}}\PYG{n+nt}{\PYGZgt{}}
\PYG{+w}{    }\PYG{n+nt}{\PYGZlt{}title}\PYG{n+nt}{\PYGZgt{}}snippets\PYG{n+nt}{\PYGZlt{}/title\PYGZgt{}}
\PYG{+w}{    }\PYG{n+nt}{\PYGZlt{}body}\PYG{n+nt}{\PYGZgt{}}
\PYG{+w}{        }\PYG{n+nt}{\PYGZlt{}note}\PYG{+w}{ }\PYG{n+na}{id=}\PYG{l+s}{\PYGZdq{}reminder\PYGZdq{}}\PYG{n+nt}{\PYGZgt{}}可能会有微小误差,在国家标准之内\PYG{n+nt}{\PYGZlt{}/note\PYGZgt{}}
\PYG{+w}{    }\PYG{n+nt}{\PYGZlt{}/body\PYGZgt{}}
\PYG{n+nt}{\PYGZlt{}/topic\PYGZgt{}}
\end{sphinxVerbatim}

\sphinxAtStartPar
在projector\_spec.dita中引用上方提示信息

\begin{sphinxVerbatim}[commandchars=\\\{\}]
\PYG{+w}{            }\PYG{n+nt}{\PYGZlt{}strow}\PYG{n+nt}{\PYGZgt{}}
\PYG{+w}{                }\PYG{n+nt}{\PYGZlt{}stentry}\PYG{n+nt}{\PYGZgt{}}
\PYG{+w}{                    }\PYG{n+nt}{\PYGZlt{}p}\PYG{n+nt}{\PYGZgt{}}光圈范围\PYG{n+nt}{\PYGZlt{}/p\PYGZgt{}}
\PYG{+w}{                }\PYG{n+nt}{\PYGZlt{}/stentry\PYGZgt{}}
\PYG{+w}{                }\PYG{n+nt}{\PYGZlt{}stentry}\PYG{n+nt}{\PYGZgt{}}
\PYG{+w}{                    }\PYG{n+nt}{\PYGZlt{}p}\PYG{n+nt}{\PYGZgt{}}F=1.75\PYGZhy{}2.42\PYG{n+nt}{\PYGZlt{}/p\PYGZgt{}}
\PYG{+w}{                    }\PYG{n+nt}{\PYGZlt{}note}\PYG{+w}{ }\PYG{n+na}{conref=}\PYG{l+s}{\PYGZdq{}snippets.dita\PYGZsh{}topic\PYGZus{}slh\PYGZus{}bsm\PYGZus{}lrb/reminder\PYGZdq{}}\PYG{n+nt}{/\PYGZgt{}}
\PYG{+w}{                }\PYG{n+nt}{\PYGZlt{}/stentry\PYGZgt{}}
\PYG{+w}{            }\PYG{n+nt}{\PYGZlt{}/strow\PYGZgt{}}
\end{sphinxVerbatim}


\section{文本链接}
\label{\detokenize{dita/dita-dive-in:id2}}\begin{itemize}
\item {} 
\sphinxAtStartPar
\sphinxstylestrong{Hierarchical links} 。文档结构。\sphinxcode{\sphinxupquote{<topicref>}}

\item {} 
\sphinxAtStartPar
\sphinxstylestrong{inline links}. 文内超链接。\sphinxcode{\sphinxupquote{<xref>}}

\item {} 
\sphinxAtStartPar
\sphinxstylestrong{Related Links}.相关链接

\end{itemize}


\subsection{相关链接}
\label{\detokenize{dita/dita-dive-in:id3}}

\subsubsection{直接添加}
\label{\detokenize{dita/dita-dive-in:id4}}
\begin{sphinxVerbatim}[commandchars=\\\{\}]
\PYG{o}{\PYGZlt{}}\PYG{n}{related}\PYG{o}{\PYGZhy{}}\PYG{n}{links}\PYG{o}{\PYGZgt{}}
    \PYG{o}{\PYGZlt{}}\PYG{n}{link} \PYG{n}{href}\PYG{o}{=}\PYG{l+s+s2}{\PYGZdq{}}\PYG{l+s+s2}{sample1.dita}\PYG{l+s+s2}{\PYGZdq{}}\PYG{o}{\PYGZgt{}}\PYG{o}{\PYGZlt{}}\PYG{n}{linktext}\PYG{o}{\PYGZgt{}}\PYG{n}{Sample} \PYG{l+m+mi}{1} \PYG{n}{Page}\PYG{o}{\PYGZlt{}}\PYG{o}{/}\PYG{n}{linktext}\PYG{o}{\PYGZgt{}}\PYG{o}{\PYGZlt{}}\PYG{o}{/}\PYG{n}{link}\PYG{o}{\PYGZgt{}}
    \PYG{o}{\PYGZlt{}}\PYG{n}{link} \PYG{n}{href}\PYG{o}{=}\PYG{l+s+s2}{\PYGZdq{}}\PYG{l+s+s2}{sample2.dita}\PYG{l+s+s2}{\PYGZdq{}}\PYG{o}{/}\PYG{o}{\PYGZgt{}}
    \PYG{o}{\PYGZlt{}}\PYG{n}{link} \PYG{n}{href}\PYG{o}{=}\PYG{l+s+s2}{\PYGZdq{}}\PYG{l+s+s2}{sample3.dita}\PYG{l+s+s2}{\PYGZdq{}}\PYG{o}{/}\PYG{o}{\PYGZgt{}}
 \PYG{o}{\PYGZlt{}}\PYG{o}{/}\PYG{n}{related}\PYG{o}{\PYGZhy{}}\PYG{n}{links}\PYG{o}{\PYGZgt{}}
\end{sphinxVerbatim}


\subsubsection{关系表}
\label{\detokenize{dita/dita-dive-in:id5}}
\begin{sphinxVerbatim}[commandchars=\\\{\}]
    \PYG{o}{\PYGZlt{}}\PYG{n}{reltable}\PYG{o}{\PYGZgt{}}
      \PYG{o}{\PYGZlt{}}\PYG{n}{relheader}\PYG{o}{\PYGZgt{}}
   \PYG{o}{\PYGZlt{}}\PYG{n}{relcolspec} \PYG{n+nb}{type}\PYG{o}{=}\PYG{l+s+s2}{\PYGZdq{}}\PYG{l+s+s2}{concept}\PYG{l+s+s2}{\PYGZdq{}}\PYG{o}{\PYGZgt{}}
   \PYG{o}{\PYGZlt{}}\PYG{n}{relcolspec} \PYG{n+nb}{type}\PYG{o}{=}\PYG{l+s+s2}{\PYGZdq{}}\PYG{l+s+s2}{task}\PYG{l+s+s2}{\PYGZdq{}}\PYG{o}{\PYGZgt{}}
   \PYG{o}{\PYGZlt{}}\PYG{n}{relcolspec} \PYG{n+nb}{type}\PYG{o}{=}\PYG{l+s+s2}{\PYGZdq{}}\PYG{l+s+s2}{reference}\PYG{l+s+s2}{\PYGZdq{}}\PYG{o}{\PYGZgt{}}
      \PYG{o}{\PYGZlt{}}\PYG{o}{/}\PYG{n}{relheader}\PYG{o}{\PYGZgt{}}
     
      \PYG{o}{\PYGZlt{}}\PYG{n}{relrow}\PYG{o}{\PYGZgt{}}
        \PYG{o}{\PYGZlt{}}\PYG{n}{relcell}\PYG{o}{\PYGZgt{}}
          \PYG{o}{\PYGZlt{}}\PYG{n}{topicref} \PYG{n}{href}\PYG{o}{=}\PYG{l+s+s2}{\PYGZdq{}}\PYG{l+s+s2}{sample\PYGZus{}concept.dita}\PYG{l+s+s2}{\PYGZdq{}}\PYG{o}{/}\PYG{o}{\PYGZgt{}}
        \PYG{o}{\PYGZlt{}}\PYG{o}{/}\PYG{n}{relcell}\PYG{o}{\PYGZgt{}}
      
        \PYG{o}{\PYGZlt{}}\PYG{n}{relcell}\PYG{o}{\PYGZgt{}}
          \PYG{o}{\PYGZlt{}}\PYG{n}{topicref} \PYG{n}{href}\PYG{o}{=}\PYG{l+s+s2}{\PYGZdq{}}\PYG{l+s+s2}{task\PYGZus{}example1.dita}\PYG{l+s+s2}{\PYGZdq{}}\PYG{o}{/}\PYG{o}{\PYGZgt{}}
          \PYG{o}{\PYGZlt{}}\PYG{n}{topicref} \PYG{n}{href}\PYG{o}{=}\PYG{l+s+s2}{\PYGZdq{}}\PYG{l+s+s2}{task\PYGZus{}example2.dita}\PYG{l+s+s2}{\PYGZdq{}}\PYG{o}{/}\PYG{o}{\PYGZgt{}}
        \PYG{o}{\PYGZlt{}}\PYG{o}{/}\PYG{n}{relcell}\PYG{o}{\PYGZgt{}}

        \PYG{o}{\PYGZlt{}}\PYG{n}{relcell}\PYG{o}{\PYGZgt{}}
          \PYG{o}{\PYGZlt{}}\PYG{n}{topicref} \PYG{n}{href}\PYG{o}{=}\PYG{l+s+s2}{\PYGZdq{}}\PYG{l+s+s2}{referencefile.dita}\PYG{l+s+s2}{\PYGZdq{}}\PYG{o}{/}\PYG{o}{\PYGZgt{}}
         \PYG{o}{\PYGZlt{}}\PYG{o}{/}\PYG{n}{relcell}\PYG{o}{\PYGZgt{}}
      \PYG{o}{\PYGZlt{}}\PYG{o}{/}\PYG{n}{relrow}\PYG{o}{\PYGZgt{}}
    \PYG{o}{\PYGZlt{}}\PYG{o}{/}\PYG{n}{reltable}\PYG{o}{\PYGZgt{}}
\end{sphinxVerbatim}


\section{相关资源}
\label{\detokenize{dita/dita-dive-in:id6}}\begin{itemize}
\item {} 
\sphinxAtStartPar
\sphinxhref{https://idratherbewriting.com/ditaval/}{DITA Notes}

\end{itemize}

\sphinxstepscope


\chapter{DITA源码阅读}
\label{\detokenize{dita-examples/dita-source-reading:dita}}\label{\detokenize{dita-examples/dita-source-reading::doc}}
\sphinxAtStartPar
正如学习编码时阅读软件源代码一样,学习DITA时,阅读文档的dita源码也是一种很好的学习习惯。这里以DITA\sphinxhyphen{}OT的文档为例,带大家阅读一下文档,并与dita源码相联系。


\section{文档}
\label{\detokenize{dita-examples/dita-source-reading:id1}}

\subsection{文档导航}
\label{\detokenize{dita-examples/dita-source-reading:id2}}
\sphinxAtStartPar
如图片左侧导航所示,左侧为文档导航区,主要通过ditamap来实现。

\sphinxAtStartPar
\sphinxincludegraphics{{navigation}.png}

\sphinxAtStartPar
对应的ditamap源码

\begin{sphinxVerbatim}[commandchars=\\\{\}]
\PYG{n+nt}{\PYGZlt{}map}\PYG{+w}{ }\PYG{n+na}{xml:lang=}\PYG{l+s}{\PYGZdq{}en\PYGZhy{}US\PYGZdq{}}\PYG{n+nt}{\PYGZgt{}}
\PYG{+w}{  }\PYG{n+nt}{\PYGZlt{}title}\PYG{n+nt}{\PYGZgt{}}DITA\PYG{+w}{ }Open\PYG{+w}{ }Toolkit\PYG{+w}{ }\PYG{n+nt}{\PYGZlt{}keyword}\PYG{+w}{ }\PYG{n+na}{keyref=}\PYG{l+s}{\PYGZdq{}release\PYGZdq{}}\PYG{n+nt}{/\PYGZgt{}}\PYG{n+nt}{\PYGZlt{}/title\PYGZgt{}}

\PYG{+w}{  }\PYG{n+nt}{\PYGZlt{}topicref}\PYG{+w}{ }\PYG{n+na}{keyref=}\PYG{l+s}{\PYGZdq{}landing\PYGZhy{}page\PYGZdq{}}\PYG{n+nt}{/\PYGZgt{}}
\PYG{+w}{  }\PYG{n+nt}{\PYGZlt{}topicref}\PYG{+w}{ }\PYG{n+na}{keyref=}\PYG{l+s}{\PYGZdq{}release\PYGZhy{}notes\PYGZdq{}}\PYG{n+nt}{\PYGZgt{}}
\PYG{+w}{    }\PYG{n+nt}{\PYGZlt{}topicref}\PYG{+w}{ }\PYG{n+na}{keyref=}\PYG{l+s}{\PYGZdq{}release\PYGZhy{}history\PYGZdq{}}\PYG{n+nt}{/\PYGZgt{}}
\PYG{+w}{  }\PYG{n+nt}{\PYGZlt{}/topicref\PYGZgt{}}
\PYG{+w}{  }\PYG{n+nt}{\PYGZlt{}topicref}\PYG{+w}{ }\PYG{n+na}{keyref=}\PYG{l+s}{\PYGZdq{}installing\PYGZhy{}client\PYGZdq{}}\PYG{n+nt}{\PYGZgt{}}
\PYG{+w}{    }\PYG{n+nt}{\PYGZlt{}mapref}\PYG{+w}{ }\PYG{n+na}{href=}\PYG{l+s}{\PYGZdq{}../topics/installing.ditamap\PYGZdq{}}\PYG{n+nt}{/\PYGZgt{}}
\PYG{+w}{  }\PYG{n+nt}{\PYGZlt{}/topicref\PYGZgt{}}
\PYG{+w}{  }\PYG{n+nt}{\PYGZlt{}topicref}\PYG{+w}{ }\PYG{n+na}{keyref=}\PYG{l+s}{\PYGZdq{}building\PYGZhy{}output\PYGZdq{}}\PYG{n+nt}{\PYGZgt{}}
\PYG{+w}{    }\PYG{n+nt}{\PYGZlt{}mapref}\PYG{+w}{ }\PYG{n+na}{href=}\PYG{l+s}{\PYGZdq{}../topics/publishing.ditamap\PYGZdq{}}\PYG{n+nt}{/\PYGZgt{}}
\PYG{+w}{  }\PYG{n+nt}{\PYGZlt{}/topicref\PYGZgt{}}
\PYG{+w}{  }\PYG{n+nt}{\PYGZlt{}topicref}\PYG{+w}{ }\PYG{n+na}{keyref=}\PYG{l+s}{\PYGZdq{}input\PYGZhy{}formats\PYGZdq{}}\PYG{+w}{ }\PYG{n+na}{collection\PYGZhy{}type=}\PYG{l+s}{\PYGZdq{}family\PYGZdq{}}\PYG{n+nt}{\PYGZgt{}}
\PYG{+w}{    }\PYG{n+nt}{\PYGZlt{}mapref}\PYG{+w}{ }\PYG{n+na}{href=}\PYG{l+s}{\PYGZdq{}../topics/input\PYGZhy{}formats.ditamap\PYGZdq{}}\PYG{n+nt}{/\PYGZgt{}}
\PYG{+w}{  }\PYG{n+nt}{\PYGZlt{}/topicref\PYGZgt{}}
\PYG{+w}{  }\PYG{n+nt}{\PYGZlt{}topicref}\PYG{+w}{ }\PYG{n+na}{keyref=}\PYG{l+s}{\PYGZdq{}output\PYGZhy{}formats\PYGZdq{}}\PYG{n+nt}{\PYGZgt{}}
\PYG{+w}{    }\PYG{n+nt}{\PYGZlt{}mapref}\PYG{+w}{ }\PYG{n+na}{href=}\PYG{l+s}{\PYGZdq{}../topics/transformations.ditamap\PYGZdq{}}\PYG{n+nt}{/\PYGZgt{}}
\PYG{+w}{  }\PYG{n+nt}{\PYGZlt{}/topicref\PYGZgt{}}
\PYG{+w}{  }\PYG{n+nt}{\PYGZlt{}topicref}\PYG{+w}{ }\PYG{n+na}{keyref=}\PYG{l+s}{\PYGZdq{}parameters\PYGZdq{}}\PYG{n+nt}{\PYGZgt{}}
\PYG{+w}{    }\PYG{n+nt}{\PYGZlt{}mapref}\PYG{+w}{ }\PYG{n+na}{href=}\PYG{l+s}{\PYGZdq{}../parameters/parameters.ditamap\PYGZdq{}}\PYG{n+nt}{/\PYGZgt{}}
\PYG{+w}{  }\PYG{n+nt}{\PYGZlt{}/topicref\PYGZgt{}}
\PYG{+w}{  }\PYG{n+nt}{\PYGZlt{}topicref}\PYG{+w}{ }\PYG{n+na}{keyref=}\PYG{l+s}{\PYGZdq{}html\PYGZhy{}customization\PYGZdq{}}\PYG{+w}{ }\PYG{n+na}{collection\PYGZhy{}type=}\PYG{l+s}{\PYGZdq{}family\PYGZdq{}}\PYG{n+nt}{\PYGZgt{}}
\PYG{+w}{    }\PYG{n+nt}{\PYGZlt{}mapref}\PYG{+w}{ }\PYG{n+na}{href=}\PYG{l+s}{\PYGZdq{}../topics/html\PYGZhy{}customization.ditamap\PYGZdq{}}\PYG{n+nt}{/\PYGZgt{}}
\PYG{+w}{  }\PYG{n+nt}{\PYGZlt{}/topicref\PYGZgt{}}
\PYG{+w}{  }\PYG{n+nt}{\PYGZlt{}topicref}\PYG{+w}{ }\PYG{n+na}{keyref=}\PYG{l+s}{\PYGZdq{}pdf\PYGZhy{}customization\PYGZdq{}}\PYG{n+nt}{\PYGZgt{}}
\PYG{+w}{    }\PYG{n+nt}{\PYGZlt{}mapref}\PYG{+w}{ }\PYG{n+na}{href=}\PYG{l+s}{\PYGZdq{}../topics/pdf\PYGZhy{}customization.ditamap\PYGZdq{}}\PYG{n+nt}{/\PYGZgt{}}
\PYG{+w}{  }\PYG{n+nt}{\PYGZlt{}/topicref\PYGZgt{}}
\PYG{+w}{  }\PYG{n+nt}{\PYGZlt{}topicref}\PYG{+w}{ }\PYG{n+na}{keyref=}\PYG{l+s}{\PYGZdq{}adding\PYGZhy{}plugins\PYGZdq{}}\PYG{n+nt}{\PYGZgt{}}
\PYG{+w}{    }\PYG{n+nt}{\PYGZlt{}mapref}\PYG{+w}{ }\PYG{n+na}{href=}\PYG{l+s}{\PYGZdq{}../topics/using\PYGZhy{}plugins.ditamap\PYGZdq{}}\PYG{n+nt}{/\PYGZgt{}}
\PYG{+w}{  }\PYG{n+nt}{\PYGZlt{}/topicref\PYGZgt{}}
\PYG{+w}{  }\PYG{n+nt}{\PYGZlt{}topicref}\PYG{+w}{ }\PYG{n+na}{keyref=}\PYG{l+s}{\PYGZdq{}custom\PYGZhy{}plugins\PYGZdq{}}\PYG{n+nt}{\PYGZgt{}}
\PYG{+w}{    }\PYG{n+nt}{\PYGZlt{}mapref}\PYG{+w}{ }\PYG{n+na}{href=}\PYG{l+s}{\PYGZdq{}../topics/creating\PYGZhy{}plugins.ditamap\PYGZdq{}}\PYG{n+nt}{/\PYGZgt{}}
\PYG{+w}{  }\PYG{n+nt}{\PYGZlt{}/topicref\PYGZgt{}}
\PYG{+w}{  }\PYG{n+nt}{\PYGZlt{}topicref}\PYG{+w}{ }\PYG{n+na}{keyref=}\PYG{l+s}{\PYGZdq{}troubleshooting\PYGZhy{}overview\PYGZdq{}}\PYG{n+nt}{\PYGZgt{}}
\PYG{+w}{    }\PYG{n+nt}{\PYGZlt{}mapref}\PYG{+w}{ }\PYG{n+na}{href=}\PYG{l+s}{\PYGZdq{}../topics/troubleshooting.ditamap\PYGZdq{}}\PYG{n+nt}{/\PYGZgt{}}
\PYG{+w}{  }\PYG{n+nt}{\PYGZlt{}/topicref\PYGZgt{}}
\PYG{+w}{  }\PYG{n+nt}{\PYGZlt{}topicref}\PYG{+w}{ }\PYG{n+na}{keyref=}\PYG{l+s}{\PYGZdq{}reference\PYGZdq{}}\PYG{n+nt}{\PYGZgt{}}
\PYG{+w}{    }\PYG{n+nt}{\PYGZlt{}mapref}\PYG{+w}{ }\PYG{n+na}{href=}\PYG{l+s}{\PYGZdq{}../reference/reference.ditamap\PYGZdq{}}\PYG{n+nt}{/\PYGZgt{}}
\PYG{+w}{  }\PYG{n+nt}{\PYGZlt{}/topicref\PYGZgt{}}
\PYG{+w}{  }\PYG{n+nt}{\PYGZlt{}topicref}\PYG{+w}{ }\PYG{n+na}{keyref=}\PYG{l+s}{\PYGZdq{}dita\PYGZhy{}and\PYGZhy{}dita\PYGZhy{}ot\PYGZhy{}resources\PYGZdq{}}\PYG{n+nt}{\PYGZgt{}}
\PYG{+w}{    }\PYG{n+nt}{\PYGZlt{}mapref}\PYG{+w}{ }\PYG{n+na}{href=}\PYG{l+s}{\PYGZdq{}../topics/dita\PYGZhy{}resources.ditamap\PYGZdq{}}\PYG{+w}{ }\PYG{n+na}{toc=}\PYG{l+s}{\PYGZdq{}no\PYGZdq{}}\PYG{n+nt}{/\PYGZgt{}}
\PYG{+w}{    }\PYG{n+nt}{\PYGZlt{}mapref}\PYG{+w}{ }\PYG{n+na}{href=}\PYG{l+s}{\PYGZdq{}../topics/dita\PYGZhy{}ot\PYGZhy{}day\PYGZhy{}videos.ditamap\PYGZdq{}}\PYG{n+nt}{/\PYGZgt{}}
\PYG{+w}{  }\PYG{n+nt}{\PYGZlt{}/topicref\PYGZgt{}}

\PYG{+w}{  }\PYG{n+nt}{\PYGZlt{}topicgroup}\PYG{+w}{ }\PYG{n+na}{processing\PYGZhy{}role=}\PYG{l+s}{\PYGZdq{}resource\PYGZhy{}only\PYGZdq{}}\PYG{n+nt}{\PYGZgt{}}
\PYG{+w}{    }\PYG{n+nt}{\PYGZlt{}mapref}\PYG{+w}{ }\PYG{n+na}{href=}\PYG{l+s}{\PYGZdq{}common\PYGZhy{}resources.ditamap\PYGZdq{}}\PYG{n+nt}{/\PYGZgt{}}
\PYG{+w}{  }\PYG{n+nt}{\PYGZlt{}/topicgroup\PYGZgt{}}
\PYG{n+nt}{\PYGZlt{}/map\PYGZgt{}}
\end{sphinxVerbatim}


\subsection{Choicetable}
\label{\detokenize{dita-examples/dita-source-reading:choicetable}}
\sphinxAtStartPar
通过Tab的形式,实现了信息的折叠。

\sphinxAtStartPar
\sphinxincludegraphics{{choicetable}.png}

\sphinxAtStartPar
源代码

\begin{sphinxVerbatim}[commandchars=\\\{\}]
          \PYG{o}{\PYGZlt{}}\PYG{n}{chrow} \PYG{n}{platform}\PYG{o}{=}\PYG{l+s+s2}{\PYGZdq{}}\PYG{l+s+s2}{linux mac}\PYG{l+s+s2}{\PYGZdq{}}\PYG{o}{\PYGZgt{}}
            \PYG{o}{\PYGZlt{}}\PYG{n}{choption}\PYG{o}{\PYGZgt{}}\PYG{n}{Linux} \PYG{o+ow}{or} \PYG{n}{macOS}\PYG{o}{\PYGZam{}}\PYG{c+c1}{\PYGZsh{}xA0;\PYGZlt{}/choption\PYGZgt{}}
            \PYG{o}{\PYGZlt{}}\PYG{n}{chdesc}\PYG{o}{\PYGZgt{}}\PYG{n}{Type} \PYG{o}{\PYGZlt{}}\PYG{n}{userinput}\PYG{o}{\PYGZgt{}}\PYG{n}{Terminal}\PYG{o}{\PYGZlt{}}\PYG{o}{/}\PYG{n}{userinput}\PYG{o}{\PYGZgt{}}\PYG{o}{.}\PYG{o}{\PYGZlt{}}\PYG{o}{/}\PYG{n}{chdesc}\PYG{o}{\PYGZgt{}}
          \PYG{o}{\PYGZlt{}}\PYG{o}{/}\PYG{n}{chrow}\PYG{o}{\PYGZgt{}}
          \PYG{o}{\PYGZlt{}}\PYG{n}{chrow} \PYG{n}{platform}\PYG{o}{=}\PYG{l+s+s2}{\PYGZdq{}}\PYG{l+s+s2}{windows}\PYG{l+s+s2}{\PYGZdq{}}\PYG{o}{\PYGZgt{}}
            \PYG{o}{\PYGZlt{}}\PYG{n}{choption}\PYG{o}{\PYGZgt{}}\PYG{n}{Windows}\PYG{o}{\PYGZlt{}}\PYG{o}{/}\PYG{n}{choption}\PYG{o}{\PYGZgt{}}
            \PYG{o}{\PYGZlt{}}\PYG{n}{chdesc}\PYG{o}{\PYGZgt{}}\PYG{n}{Type} \PYG{o}{\PYGZlt{}}\PYG{n}{userinput}\PYG{o}{\PYGZgt{}}\PYG{n}{Command} \PYG{n}{Prompt}\PYG{o}{\PYGZlt{}}\PYG{o}{/}\PYG{n}{userinput}\PYG{o}{\PYGZgt{}}\PYG{o}{.}\PYG{o}{\PYGZlt{}}\PYG{o}{/}\PYG{n}{chdesc}\PYG{o}{\PYGZgt{}}
          \PYG{o}{\PYGZlt{}}\PYG{o}{/}\PYG{n}{chrow}\PYG{o}{\PYGZgt{}}
        \PYG{o}{\PYGZlt{}}\PYG{o}{/}\PYG{n}{choicetable}\PYG{o}{\PYGZgt{}}
\end{sphinxVerbatim}


\subsection{footnote}
\label{\detokenize{dita-examples/dita-source-reading:footnote}}
\sphinxAtStartPar
\sphinxincludegraphics{{footnote}.png}
\sphinxincludegraphics{{footnote-result}.png}

\begin{sphinxVerbatim}[commandchars=\\\{\}]
\PYG{+w}{ }\PYG{n+nt}{\PYGZlt{}fn}\PYG{n+nt}{\PYGZgt{}}Homebrew’s\PYG{+w}{ }default\PYG{+w}{ }installation\PYG{+w}{ }location\PYG{+w}{ }depends\PYG{+w}{ }on\PYG{+w}{ }the\PYG{+w}{ }operating\PYG{+w}{ }system\PYG{+w}{ }architecture:
\PYG{+w}{          }\PYG{n+nt}{\PYGZlt{}ul}\PYG{n+nt}{\PYGZgt{}}
\PYG{+w}{            }\PYG{n+nt}{\PYGZlt{}li}\PYG{n+nt}{\PYGZgt{}}\PYG{n+nt}{\PYGZlt{}filepath}\PYG{n+nt}{\PYGZgt{}}/usr/local\PYG{n+nt}{\PYGZlt{}/filepath\PYGZgt{}}\PYG{+w}{ }on\PYG{+w}{ }macOS\PYG{+w}{ }Intel\PYG{n+nt}{\PYGZlt{}/li\PYGZgt{}}
\PYG{+w}{            }\PYG{n+nt}{\PYGZlt{}li}\PYG{n+nt}{\PYGZgt{}}\PYG{n+nt}{\PYGZlt{}filepath}\PYG{n+nt}{\PYGZgt{}}/opt/homebrew\PYG{n+nt}{\PYGZlt{}/filepath\PYGZgt{}}\PYG{+w}{ }on\PYG{+w}{ }macOS\PYG{+w}{ }ARM\PYG{n+nt}{\PYGZlt{}/li\PYGZgt{}}
\PYG{+w}{            }\PYG{n+nt}{\PYGZlt{}li}\PYG{n+nt}{\PYGZgt{}}\PYG{n+nt}{\PYGZlt{}filepath}\PYG{n+nt}{\PYGZgt{}}/home/linuxbrew/.linuxbrew\PYG{n+nt}{\PYGZlt{}/filepath\PYGZgt{}}\PYG{+w}{ }on\PYG{+w}{ }Linux\PYG{n+nt}{\PYGZlt{}/li\PYGZgt{}}
\PYG{+w}{          }\PYG{n+nt}{\PYGZlt{}/ul\PYGZgt{}}
\PYG{+w}{ }\PYG{n+nt}{\PYGZlt{}/fn\PYGZgt{}}\PYG{+w}{ }to\PYG{+w}{ }ensure\PYG{+w}{ }that\PYG{+w}{ }Homebrew\PYGZhy{}installed\PYG{+w}{ }software
\end{sphinxVerbatim}


\subsection{relationship table}
\label{\detokenize{dita-examples/dita-source-reading:relationship-table}}
\sphinxAtStartPar
\sphinxincludegraphics{{reltables}.png}

\begin{sphinxVerbatim}[commandchars=\\\{\}]
\PYG{+w}{   }\PYG{n+nt}{\PYGZlt{}title}\PYG{n+nt}{\PYGZgt{}}CTR:\PYG{+w}{ }Ant\PYG{+w}{ }topics\PYG{n+nt}{\PYGZlt{}/title\PYGZgt{}}
\PYG{+w}{    }\PYG{n+nt}{\PYGZlt{}relheader}\PYG{n+nt}{\PYGZgt{}}
\PYG{+w}{      }\PYG{n+nt}{\PYGZlt{}relcolspec}\PYG{+w}{ }\PYG{n+na}{type=}\PYG{l+s}{\PYGZdq{}concept\PYGZdq{}}\PYG{n+nt}{/\PYGZgt{}}
\PYG{+w}{      }\PYG{n+nt}{\PYGZlt{}relcolspec}\PYG{+w}{ }\PYG{n+na}{type=}\PYG{l+s}{\PYGZdq{}task\PYGZdq{}}\PYG{n+nt}{/\PYGZgt{}}
\PYG{+w}{      }\PYG{n+nt}{\PYGZlt{}relcolspec}\PYG{+w}{ }\PYG{n+na}{type=}\PYG{l+s}{\PYGZdq{}reference\PYGZdq{}}\PYG{n+nt}{/\PYGZgt{}}
\PYG{+w}{    }\PYG{n+nt}{\PYGZlt{}/relheader\PYGZgt{}}
\PYG{+w}{    }\PYG{n+nt}{\PYGZlt{}relrow}\PYG{n+nt}{\PYGZgt{}}
\PYG{+w}{      }\PYG{n+nt}{\PYGZlt{}relcell}\PYG{n+nt}{\PYGZgt{}}
\PYG{+w}{        }\PYG{n+nt}{\PYGZlt{}topicref}\PYG{+w}{ }\PYG{n+na}{keyref=}\PYG{l+s}{\PYGZdq{}ant\PYGZdq{}}\PYG{n+nt}{/\PYGZgt{}}
\PYG{+w}{      }\PYG{n+nt}{\PYGZlt{}/relcell\PYGZgt{}}
\PYG{+w}{      }\PYG{n+nt}{\PYGZlt{}relcell}\PYG{n+nt}{\PYGZgt{}}
\PYG{+w}{        }\PYG{n+nt}{\PYGZlt{}topicgroup}\PYG{+w}{ }\PYG{n+na}{collection\PYGZhy{}type=}\PYG{l+s}{\PYGZdq{}family\PYGZdq{}}\PYG{n+nt}{\PYGZgt{}}
\PYG{+w}{          }\PYG{n+nt}{\PYGZlt{}topicref}\PYG{+w}{ }\PYG{n+na}{keyref=}\PYG{l+s}{\PYGZdq{}building\PYGZhy{}with\PYGZhy{}ant\PYGZdq{}}\PYG{n+nt}{/\PYGZgt{}}
\PYG{+w}{          }\PYG{n+nt}{\PYGZlt{}topicref}\PYG{+w}{ }\PYG{n+na}{keyref=}\PYG{l+s}{\PYGZdq{}creating\PYGZhy{}an\PYGZhy{}ant\PYGZhy{}build\PYGZhy{}script\PYGZdq{}}\PYG{n+nt}{/\PYGZgt{}}
\PYG{+w}{        }\PYG{n+nt}{\PYGZlt{}/topicgroup\PYGZgt{}}
\PYG{+w}{      }\PYG{n+nt}{\PYGZlt{}/relcell\PYGZgt{}}
\PYG{+w}{      }\PYG{n+nt}{\PYGZlt{}relcell}\PYG{n+nt}{\PYGZgt{}}
\PYG{+w}{        }\PYG{n+nt}{\PYGZlt{}topicref}\PYG{+w}{ }\PYG{n+na}{keyref=}\PYG{l+s}{\PYGZdq{}parameters\PYGZus{}intro\PYGZdq{}}\PYG{+w}{ }\PYG{n+na}{linking=}\PYG{l+s}{\PYGZdq{}targetonly\PYGZdq{}}\PYG{n+nt}{/\PYGZgt{}}
\PYG{+w}{        }\PYG{n+nt}{\PYGZlt{}topicref}\PYG{+w}{ }\PYG{n+na}{keyref=}\PYG{l+s}{\PYGZdq{}ant\PYGZhy{}manual\PYGZdq{}}\PYG{n+nt}{\PYGZgt{}}
\PYG{+w}{          }\PYG{n+nt}{\PYGZlt{}topicmeta}\PYG{n+nt}{\PYGZgt{}}
\PYG{+w}{            }\PYG{n+nt}{\PYGZlt{}linktext}\PYG{n+nt}{\PYGZgt{}}Apache\PYG{+w}{ }Ant\PYG{+w}{ }documentation\PYG{n+nt}{\PYGZlt{}/linktext\PYGZgt{}}
\PYG{+w}{          }\PYG{n+nt}{\PYGZlt{}/topicmeta\PYGZgt{}}
\PYG{+w}{        }\PYG{n+nt}{\PYGZlt{}/topicref\PYGZgt{}}
\PYG{+w}{      }\PYG{n+nt}{\PYGZlt{}/relcell\PYGZgt{}}
\PYG{+w}{    }\PYG{n+nt}{\PYGZlt{}/relrow\PYGZgt{}}
\PYG{+w}{    }\PYG{n+nt}{\PYGZlt{}relrow}\PYG{n+nt}{\PYGZgt{}}
\end{sphinxVerbatim}


\subsection{XML element reference}
\label{\detokenize{dita-examples/dita-source-reading:xml-element-reference}}
\sphinxAtStartPar
引用XML文件中的xml元素

\sphinxAtStartPar
\sphinxincludegraphics{{xmlelement}.png}

\begin{sphinxVerbatim}[commandchars=\\\{\}]
\PYG{+w}{      }\PYG{n+nt}{\PYGZlt{}title}\PYG{n+nt}{\PYGZgt{}}\PYG{n+nt}{\PYGZlt{}xmlelement}\PYG{n+nt}{\PYGZgt{}}plugin\PYG{n+nt}{\PYGZlt{}/xmlelement\PYGZgt{}}\PYG{n+nt}{\PYGZlt{}/title\PYGZgt{}}
\PYG{+w}{      }\PYG{n+nt}{\PYGZlt{}p}\PYG{n+nt}{\PYGZgt{}}The\PYG{+w}{ }root\PYG{+w}{ }element\PYG{+w}{ }of\PYG{+w}{ }the\PYG{+w}{ }\PYG{n+nt}{\PYGZlt{}filepath}\PYG{n+nt}{\PYGZgt{}}plugin.xml\PYG{n+nt}{\PYGZlt{}/filepath\PYGZgt{}}\PYG{+w}{ }file\PYG{+w}{ }is\PYG{+w}{ }\PYG{n+nt}{\PYGZlt{}xmlelement}\PYG{n+nt}{\PYGZgt{}}plugin\PYG{n+nt}{\PYGZlt{}/xmlelement\PYGZgt{}},\PYG{+w}{ }which\PYG{+w}{ }has\PYG{+w}{ }a\PYG{+w}{ }required\PYG{+w}{ }\PYG{n+nt}{\PYGZlt{}xmlatt}\PYG{n+nt}{\PYGZgt{}}id\PYG{n+nt}{\PYGZlt{}/xmlatt\PYGZgt{}}\PYG{+w}{ }attribute\PYG{+w}{ }set\PYG{+w}{ }to\PYG{+w}{ }the\PYG{+w}{ }unique\PYG{+w}{ }plug\PYGZhy{}in\PYG{+w}{ }identifier.\PYG{n+nt}{\PYGZlt{}/p\PYGZgt{}}
\end{sphinxVerbatim}

\sphinxstepscope


\chapter{轻量级DITA}
\label{\detokenize{dita/lightweight-dita:dita}}\label{\detokenize{dita/lightweight-dita::doc}}
\sphinxAtStartPar
数字化时代,技术文档的编写和管理愈发重要。通常,技术文档工程师会运用XML/DITA及相关工具进行技术文档的创作、管理和交付。DITA(Darwin Information Typing Architecture)是基于XML的体系结构,其单一源内容可以通过不同的方式,对各个模块进行重用,生成多样的交付内容,显著提高了技术文档的重用性和可维护性。除此之外,DITA还支持主题化的信息创建方法,其中一个topic(主题)可以分为concept(概念)、task(任务)、reference(参考),和troubleshooting(故障处理)等多种基本类型,再通过类似于思维导图的map将各个topic组织形成完整的文档(详见图1),使得文档的开发任务可以很方便地分解到各个文档编写人员手中,最终组合生成格式统一、内容规范的文档。

\sphinxAtStartPar
\sphinxincludegraphics{{dita-frame}.png}

\sphinxAtStartPar
然而,传统的DITA结构相对复杂,对于一些小型项目或团队而言,使用起来可能有些繁琐。同时,产品工程师、市场营销人员、经理等人员可能并不熟悉DITA,难以进行内容输出,所有重担都落在了文档工程师身上。在这种情况下,如何让那些不使用XML的公司员工也能参与内容创作,同时又不失去结构化文档的能力呢?此时,轻量级DITA(Lightweight DITA)应运而生,成为技术文档领域的一颗新星。

\sphinxAtStartPar
轻量级DITA(LwDITA)是DITA(Darwin Information Typing Architecture)的精简版本。通过全方位简化传统DITA,LwDITA满足了更广泛的用户需求和应用场景,旨在使内容的创建和管理更为高效。


\section{背景}
\label{\detokenize{dita/lightweight-dita:id1}}
\sphinxAtStartPar
要理解LwDITA,首先需要了解DITA的基础概念:
\begin{itemize}
\item {} 
\sphinxAtStartPar
\sphinxstylestrong{DITA简介}:DITA全称为Darwin Information Typing Architecture(达尔文信息分类体系结构),是一种模块化内容创作框架。DITA通过将内容分解为可复用的模块(topics)并结合“思维导图”式的map文件实现内容的高效整合和多样化应用。

\item {} 
\sphinxAtStartPar
\sphinxstylestrong{传统DITA的挑战}:DITA的文档结构复杂且学习成本高,对非技术文档工程师(如市场营销人员或经理)不够友好,导致内容创作责任主要集中在文档工程师身上。

\end{itemize}

\sphinxAtStartPar
\sphinxstylestrong{轻量级DITA的出现}旨在降低DITA的使用门槛,使更多非技术人员也能够轻松参与内容创作。


\section{核心特点}
\label{\detokenize{dita/lightweight-dita:id2}}
\sphinxAtStartPar
LwDITA通过以下特点实现了DITA的简化:


\subsection{简化元素类型、文档类型、属性集}
\label{\detokenize{dita/lightweight-dita:id3}}
\sphinxAtStartPar
LwDITA减少了元素种类,例如:
\begin{itemize}
\item {} 
\sphinxAtStartPar
LwDITA包含48种元素,而DITA 1.3包含610种元素。

\item {} 
\sphinxAtStartPar
示例:LwDITA简化了title相关元素,将复杂的多种分类精简为统一的title标记。

\end{itemize}

\sphinxAtStartPar
LwDITA保留了topic和map两种文档类型,移除了复杂的database文档类型,更适合小型企业和非技术用户的使用需求。

\sphinxAtStartPar
LwDITA仅保留了最核心的属性集,删除了复杂的功能。例如,LwDITA支持id和conref属性,而DITA 1.3中额外支持conkeyref等高级功能。

\sphinxAtStartPar
\sphinxincludegraphics{{lwdita-vs-dita1.3}.png}


\subsection{严格内容模型}
\label{\detokenize{dita/lightweight-dita:id4}}
\sphinxAtStartPar
LwDITA通过减少元素和引入严格规则实现更严谨的内容模型。例如:
\begin{itemize}
\item {} 
\sphinxAtStartPar
所有文本必须位于paragraph元素内。

\item {} 
\sphinxAtStartPar
内容类型的标记更加清晰,提升了内容的重用效率和精确度。

\end{itemize}


\subsection{新增多媒体元素}
\label{\detokenize{dita/lightweight-dita:id5}}
\sphinxAtStartPar
为了支持更广泛的用户群体,LwDITA新增了9项多媒体相关元素(如Audio、Video、Source等),支持音频和视频内容的创作。


\subsection{支持非XML格式}
\label{\detokenize{dita/lightweight-dita:xml}}
\sphinxAtStartPar
LwDITA支持以下三种创作格式:
\begin{itemize}
\item {} 
\sphinxAtStartPar
\sphinxstylestrong{XDITA}:基于XML的变体。

\item {} 
\sphinxAtStartPar
\sphinxstylestrong{HDITA}:基于HTML5的变体。

\item {} 
\sphinxAtStartPar
\sphinxstylestrong{MDITA}:基于Markdown的变体。

\end{itemize}

\sphinxAtStartPar
这种多格式支持让非技术用户(如销售人员或博主)可以使用熟悉的工具参与内容创作,并通过简单的转换实现文档发布。


\section{应用优势}
\label{\detokenize{dita/lightweight-dita:id6}}
\sphinxAtStartPar
LwDITA通过简化功能和支持多种创作格式,让更多非技术人员能够参与内容创作。

\sphinxAtStartPar
不同部门间的人员能够协同创作,提高知识共享和信息传递的效率。

\sphinxAtStartPar
LwDITA具有较高的灵活性,能够随着技术发展和用户需求的变化持续改进。


\section{总结与展望}
\label{\detokenize{dita/lightweight-dita:id7}}
\sphinxAtStartPar
LwDITA通过简化DITA的复杂性和提升使用便捷性,为文档创作带来了革新。未来,LwDITA可能会进一步利用新技术(如人工智能、虚拟现实)优化用户体验,并在远程协作和多媒体内容创作中发挥更大作用。


\subsection{展望}
\label{\detokenize{dita/lightweight-dita:id8}}
\sphinxAtStartPar
随着数字时代的到来,LwDITA不仅是一种工具,更是一种推动知识共享和团队协作的文化。它将继续塑造高效、便捷、包容的文档生态,为用户研究和内容创作带来更多机遇。

\sphinxstepscope


\chapter{DITA 资源}
\label{\detokenize{dita/dita_resources:dita}}\label{\detokenize{dita/dita_resources::doc}}

\section{写作风格}
\label{\detokenize{dita/dita_resources:id1}}\begin{itemize}
\item {} 
\sphinxAtStartPar
\sphinxhref{https://www.oxygenxml.com/dita/styleguide/webhelp-feedback/index.html\#Artefact/Authoring\_Concepts/c\_About\_the\_Style\_Guide.html}{The DITA Style Guide Best Practices for Authors}

\end{itemize}


\section{元素}
\label{\detokenize{dita/dita_resources:id2}}\begin{itemize}
\item {} 
\sphinxAtStartPar
\sphinxhref{https://www.oxygenxml.com/dita/1.3/specs/langRef/containers/technical-content-elements.html}{DITA Technical content elements}

\item {} 
\sphinxAtStartPar
\sphinxhref{https://www.oxygenxml.com/dita/1.3/specs/langRef/langRef-learningTraining.html}{DITA Language reference}

\end{itemize}


\section{轻量级DITA}
\label{\detokenize{dita/dita_resources:id3}}\begin{itemize}
\item {} 
\sphinxAtStartPar
\sphinxhref{https://www.dita-ot.org/dev/topics/markdown-dita-syntax-reference.html}{Markdown DITA syntax reference}

\end{itemize}


\section{DITA相关工具}
\label{\detokenize{dita/dita_resources:id4}}
\sphinxAtStartPar
\sphinxhref{http://dita.xml.org/products}{More products}


\section{相关链接}
\label{\detokenize{dita/dita_resources:id5}}\begin{itemize}
\item {} 
\sphinxAtStartPar
\sphinxhref{http://dita-archive.xml.org/book/dita-wiki-knowledgebase}{DITA Wiki Knowledgebase}

\end{itemize}

\sphinxstepscope


\chapter{XSL\sphinxhyphen{}FO 样式}
\label{\detokenize{formatting/XSL-FO:xsl-fo}}\label{\detokenize{formatting/XSL-FO::doc}}

\section{什么是XSL\sphinxhyphen{}FO}
\label{\detokenize{formatting/XSL-FO:id1}}
\sphinxAtStartPar
XSL\sphinxhyphen{}FO  (Extensible Stylesheet Language Formating Object) 是定义XML数据输出格式的语言,以前是XSL的一部分。后来对W3C对XSL进行了拆分,于是有了如下的三个技术:
\begin{itemize}
\item {} 
\sphinxAtStartPar
XSLT,转换XML文件

\item {} 
\sphinxAtStartPar
XSL 或 XSL\sphinxhyphen{}FO,格式化XML文件

\item {} 
\sphinxAtStartPar
XPath,选择XML元素和属性

\end{itemize}


\section{上手练习}
\label{\detokenize{formatting/XSL-FO:id2}}
\sphinxAtStartPar
之前我们练习了使用CSS输出XML的样式,现在我们来学习如何使用xsl\sphinxhyphen{}fo来定义XML的样式,并输出为PDF。

\sphinxAtStartPar
XML文件

\begin{sphinxVerbatim}[commandchars=\\\{\}]
\PYG{c+cp}{\PYGZlt{}?xml version=\PYGZdq{}1.0\PYGZdq{} encoding=\PYGZdq{}utf\PYGZhy{}8\PYGZdq{}?\PYGZgt{}}

\PYG{n+nt}{\PYGZlt{}businesscard}\PYG{n+nt}{\PYGZgt{}}
\PYG{+w}{   }
\PYG{+w}{   }\PYG{n+nt}{\PYGZlt{}name}\PYG{n+nt}{\PYGZgt{}}Zhijun\PYG{+w}{ }Gao\PYG{n+nt}{\PYGZlt{}/name\PYGZgt{}}
\PYG{+w}{   }\PYG{n+nt}{\PYGZlt{}tel}\PYG{n+nt}{\PYGZgt{}}010\PYGZhy{}11112222\PYG{n+nt}{\PYGZlt{}/tel\PYGZgt{}}
\PYG{+w}{   }\PYG{n+nt}{\PYGZlt{}phone}\PYG{n+nt}{\PYGZgt{}}11122223333\PYG{n+nt}{\PYGZlt{}/phone\PYGZgt{}}
\PYG{+w}{   }\PYG{n+nt}{\PYGZlt{}email}\PYG{n+nt}{\PYGZgt{}}gaozhijun@ss.pku.edu.cn\PYG{n+nt}{\PYGZlt{}/email\PYGZgt{}}
\PYG{+w}{   }\PYG{n+nt}{\PYGZlt{}profession}\PYG{n+nt}{\PYGZgt{}}Researcher,\PYG{+w}{ }Technical\PYG{+w}{ }Communicator\PYG{n+nt}{\PYGZlt{}/profession\PYGZgt{}}
\PYG{+w}{   }
\PYG{n+nt}{\PYGZlt{}/businesscard\PYGZgt{}}
\end{sphinxVerbatim}

\sphinxAtStartPar
再解释各行语言之前,我们先开始上手写一些XSL\sphinxhyphen{}FO语句。

\begin{sphinxVerbatim}[commandchars=\\\{\}]
\PYG{c+cp}{\PYGZlt{}?xml version=\PYGZdq{}1.0\PYGZdq{} encoding=\PYGZdq{}utf\PYGZhy{}8\PYGZdq{}?\PYGZgt{}}
\PYG{n+nt}{\PYGZlt{}xsl:stylesheet}\PYG{+w}{ }\PYG{n+na}{version=}\PYG{l+s}{\PYGZdq{}1.0\PYGZdq{}}
\PYG{+w}{    }\PYG{n+na}{xmlns:xsl=}\PYG{l+s}{\PYGZdq{}http://www.w3.org/1999/XSL/Transform\PYGZdq{}}
\PYG{+w}{    }\PYG{n+na}{xmlns:fo=}\PYG{l+s}{\PYGZdq{}http://www.w3.org/1999/XSL/Format\PYGZdq{}}\PYG{n+nt}{\PYGZgt{}}
\PYG{+w}{    }\PYG{n+nt}{\PYGZlt{}xsl:output}\PYG{+w}{ }\PYG{n+na}{method=}\PYG{l+s}{\PYGZdq{}xml\PYGZdq{}}\PYG{+w}{ }\PYG{n+na}{indent=}\PYG{l+s}{\PYGZdq{}yes\PYGZdq{}}\PYG{n+nt}{/\PYGZgt{}}
\PYG{+w}{    }\PYG{n+nt}{\PYGZlt{}xsl:template}\PYG{+w}{ }\PYG{n+na}{match=}\PYG{l+s}{\PYGZdq{}/\PYGZdq{}}\PYG{n+nt}{\PYGZgt{}}
\PYG{+w}{        }\PYG{n+nt}{\PYGZlt{}fo:root}\PYG{n+nt}{\PYGZgt{}}
\PYG{+w}{            }\PYG{n+nt}{\PYGZlt{}fo:layout\PYGZhy{}master\PYGZhy{}set}\PYG{n+nt}{\PYGZgt{}}
\PYG{+w}{                }\PYG{n+nt}{\PYGZlt{}fo:simple\PYGZhy{}page\PYGZhy{}master}\PYG{+w}{ }\PYG{n+na}{master\PYGZhy{}name=}\PYG{l+s}{\PYGZdq{}A4\PYGZhy{}portrait\PYGZdq{}}
\PYG{+w}{                    }\PYG{n+na}{page\PYGZhy{}height=}\PYG{l+s}{\PYGZdq{}29.7cm\PYGZdq{}}\PYG{+w}{ }\PYG{n+na}{page\PYGZhy{}width=}\PYG{l+s}{\PYGZdq{}21.0cm\PYGZdq{}}\PYG{+w}{ }\PYG{n+na}{margin=}\PYG{l+s}{\PYGZdq{}2cm\PYGZdq{}}\PYG{n+nt}{\PYGZgt{}}
\PYG{+w}{                    }\PYG{n+nt}{\PYGZlt{}fo:region\PYGZhy{}body}\PYG{+w}{   }\PYG{n+na}{margin=}\PYG{l+s}{\PYGZdq{}3cm\PYGZdq{}}\PYG{n+nt}{/\PYGZgt{}}
\PYG{+w}{                    }\PYG{n+nt}{\PYGZlt{}fo:region\PYGZhy{}before}\PYG{+w}{ }\PYG{n+na}{extent=}\PYG{l+s}{\PYGZdq{}2cm\PYGZdq{}}\PYG{n+nt}{/\PYGZgt{}}
\PYG{+w}{                    }\PYG{n+nt}{\PYGZlt{}fo:region\PYGZhy{}after}\PYG{+w}{  }\PYG{n+na}{extent=}\PYG{l+s}{\PYGZdq{}2cm\PYGZdq{}}\PYG{n+nt}{/\PYGZgt{}}
\PYG{+w}{                    }\PYG{n+nt}{\PYGZlt{}fo:region\PYGZhy{}start}\PYG{+w}{  }\PYG{n+na}{extent=}\PYG{l+s}{\PYGZdq{}2cm\PYGZdq{}}\PYG{n+nt}{/\PYGZgt{}}
\PYG{+w}{                    }\PYG{n+nt}{\PYGZlt{}fo:region\PYGZhy{}end}\PYG{+w}{    }\PYG{n+na}{extent=}\PYG{l+s}{\PYGZdq{}2cm\PYGZdq{}}\PYG{n+nt}{/\PYGZgt{}}
\PYG{+w}{                }\PYG{n+nt}{\PYGZlt{}/fo:simple\PYGZhy{}page\PYGZhy{}master\PYGZgt{}}
\PYG{+w}{            }\PYG{n+nt}{\PYGZlt{}/fo:layout\PYGZhy{}master\PYGZhy{}set\PYGZgt{}}
\PYG{+w}{            }\PYG{n+nt}{\PYGZlt{}fo:page\PYGZhy{}sequence}\PYG{+w}{ }\PYG{n+na}{master\PYGZhy{}reference=}\PYG{l+s}{\PYGZdq{}A4\PYGZhy{}portrait\PYGZdq{}}\PYG{n+nt}{\PYGZgt{}}
\PYG{+w}{                }
\PYG{+w}{                }
\PYG{+w}{                }\PYG{n+nt}{\PYGZlt{}fo:flow}\PYG{+w}{ }\PYG{n+na}{flow\PYGZhy{}name=}\PYG{l+s}{\PYGZdq{}xsl\PYGZhy{}region\PYGZhy{}body\PYGZdq{}}\PYG{n+nt}{\PYGZgt{}}
\PYG{+w}{                    }\PYG{n+nt}{\PYGZlt{}fo:block}
\PYG{+w}{                        }\PYG{n+na}{border\PYGZhy{}width=}\PYG{l+s}{\PYGZdq{}1mm\PYGZdq{}}\PYG{+w}{ }
\PYG{+w}{                        }\PYG{n+na}{font\PYGZhy{}size=}\PYG{l+s}{\PYGZdq{}20pt\PYGZdq{}}
\PYG{+w}{                        }\PYG{n+na}{color=}\PYG{l+s}{\PYGZdq{}red\PYGZdq{}}
\PYG{+w}{                        }\PYG{n+na}{font\PYGZhy{}family=}\PYG{l+s}{\PYGZdq{}Microsoft YaHei\PYGZdq{}}
\PYG{+w}{                        }\PYG{n+na}{text\PYGZhy{}decoration=}\PYG{l+s}{\PYGZdq{}underline\PYGZdq{}}\PYG{n+nt}{\PYGZgt{}}
\PYG{+w}{                        }Brief\PYG{+w}{ }Introduction
\PYG{+w}{                    }\PYG{n+nt}{\PYGZlt{}/fo:block\PYGZgt{}}
\PYG{+w}{                    }
\PYG{+w}{                    }\PYG{n+nt}{\PYGZlt{}fo:block}\PYG{n+nt}{\PYGZgt{}}\PYG{+w}{       }
\PYG{+w}{                        }
\PYG{+w}{                        }\PYG{n+nt}{\PYGZlt{}fo:external\PYGZhy{}graphic}\PYG{+w}{ }\PYG{n+na}{src=}\PYG{l+s}{\PYGZdq{}url(\PYGZsq{}test.jpeg\PYGZsq{})\PYGZdq{}}
\PYG{+w}{                        }\PYG{n+na}{content\PYGZhy{}height=}\PYG{l+s}{\PYGZdq{}5em\PYGZdq{}}\PYG{+w}{ }\PYG{n+na}{content\PYGZhy{}width=}\PYG{l+s}{\PYGZdq{}5em\PYGZdq{}}\PYG{n+nt}{/\PYGZgt{}}
\PYG{+w}{                    }
\PYG{+w}{                    }\PYG{n+nt}{\PYGZlt{}/fo:block\PYGZgt{}}
\PYG{+w}{                    }
\PYG{+w}{                    }\PYG{n+nt}{\PYGZlt{}fo:block}\PYG{+w}{ }\PYG{n+na}{space\PYGZhy{}before=}\PYG{l+s}{\PYGZdq{}2mm\PYGZdq{}}\PYG{+w}{  }\PYG{n+na}{font\PYGZhy{}family=}\PYG{l+s}{\PYGZdq{}Microsoft YaHei\PYGZdq{}}\PYG{n+nt}{\PYGZgt{}}
\PYG{+w}{                        }Name:\PYG{+w}{ }\PYG{n+nt}{\PYGZlt{}xsl:value\PYGZhy{}of}\PYG{+w}{ }\PYG{n+na}{select=}\PYG{l+s}{\PYGZdq{}businesscard/name\PYGZdq{}}\PYG{n+nt}{/\PYGZgt{}}
\PYG{+w}{                 }
\PYG{+w}{                    }\PYG{n+nt}{\PYGZlt{}/fo:block\PYGZgt{}}
\PYG{+w}{                    }\PYG{n+nt}{\PYGZlt{}fo:block}\PYG{n+nt}{\PYGZgt{}}
\PYG{+w}{                        }Phone:\PYG{+w}{ }\PYG{n+nt}{\PYGZlt{}xsl:value\PYGZhy{}of}\PYG{+w}{ }\PYG{n+na}{select=}\PYG{l+s}{\PYGZdq{}businesscard/phone\PYGZdq{}}\PYG{n+nt}{/\PYGZgt{}}
\PYG{+w}{                    }\PYG{n+nt}{\PYGZlt{}/fo:block\PYGZgt{}}
\PYG{+w}{                    }\PYG{n+nt}{\PYGZlt{}fo:block}\PYG{n+nt}{\PYGZgt{}}
\PYG{+w}{                        }tel:\PYG{+w}{ }\PYG{n+nt}{\PYGZlt{}xsl:value\PYGZhy{}of}\PYG{+w}{ }\PYG{n+na}{select=}\PYG{l+s}{\PYGZdq{}businesscard/tel\PYGZdq{}}\PYG{n+nt}{/\PYGZgt{}}
\PYG{+w}{                    }\PYG{n+nt}{\PYGZlt{}/fo:block\PYGZgt{}}
\PYG{+w}{                    }\PYG{n+nt}{\PYGZlt{}fo:block}\PYG{n+nt}{\PYGZgt{}}
\PYG{+w}{                        }email:\PYG{+w}{ }\PYG{n+nt}{\PYGZlt{}xsl:value\PYGZhy{}of}\PYG{+w}{ }\PYG{n+na}{select=}\PYG{l+s}{\PYGZdq{}businesscard/email\PYGZdq{}}\PYG{n+nt}{/\PYGZgt{}}
\PYG{+w}{                    }\PYG{n+nt}{\PYGZlt{}/fo:block\PYGZgt{}}
\PYG{+w}{                    }\PYG{n+nt}{\PYGZlt{}fo:block}\PYG{n+nt}{\PYGZgt{}}
\PYG{+w}{                        }profession:\PYG{+w}{ }\PYG{n+nt}{\PYGZlt{}xsl:value\PYGZhy{}of}\PYG{+w}{ }\PYG{n+na}{select=}\PYG{l+s}{\PYGZdq{}businesscard/profession\PYGZdq{}}\PYG{n+nt}{/\PYGZgt{}}
\PYG{+w}{                }
\PYG{+w}{                    }\PYG{n+nt}{\PYGZlt{}/fo:block\PYGZgt{}}
\PYG{+w}{  }
\PYG{+w}{                }\PYG{n+nt}{\PYGZlt{}/fo:flow\PYGZgt{}}
\PYG{+w}{                }
\PYG{+w}{            }\PYG{n+nt}{\PYGZlt{}/fo:page\PYGZhy{}sequence\PYGZgt{}}
\PYG{+w}{        }\PYG{n+nt}{\PYGZlt{}/fo:root\PYGZgt{}}
\PYG{+w}{    }\PYG{n+nt}{\PYGZlt{}/xsl:template\PYGZgt{}}
\PYG{n+nt}{\PYGZlt{}/xsl:stylesheet\PYGZgt{}}
\end{sphinxVerbatim}


\section{XSL\sphinxhyphen{}FO语法说明}
\label{\detokenize{formatting/XSL-FO:id3}}\begin{enumerate}
\sphinxsetlistlabels{\arabic}{enumi}{enumii}{}{.}%
\item {} 
\sphinxAtStartPar
声明自己是XML文件,确实xsl\sphinxhyphen{}fo本身也是xml格式。

\begin{sphinxVerbatim}[commandchars=\\\{\}]
\PYG{c+cp}{\PYGZlt{}?xml version=\PYGZdq{}1.0\PYGZdq{} encoding=\PYGZdq{}utf\PYGZhy{}8\PYGZdq{}?\PYGZgt{}}
\end{sphinxVerbatim}

\item {} 
\sphinxAtStartPar
对所有元素生效,XPath语句 “/”

\begin{sphinxVerbatim}[commandchars=\\\{\}]
\PYG{+w}{ }\PYG{n+nt}{\PYGZlt{}xsl:template}\PYG{+w}{ }\PYG{n+na}{match=}\PYG{l+s}{\PYGZdq{}/\PYGZdq{}}\PYG{n+nt}{\PYGZgt{}}
\end{sphinxVerbatim}

\item {} 
\sphinxAtStartPar
定义页面模板,可以是一页或多页

\begin{sphinxVerbatim}[commandchars=\\\{\}]
\PYG{n+nt}{\PYGZlt{}fo:layout\PYGZhy{}master\PYGZhy{}set}\PYG{n+nt}{\PYGZgt{}}
\PYG{+w}{  }\PYG{c+cm}{\PYGZlt{}!\PYGZhy{}\PYGZhy{} All page templates go here \PYGZhy{}\PYGZhy{}\PYGZgt{}}
\PYG{n+nt}{\PYGZlt{}/fo:layout\PYGZhy{}master\PYGZhy{}set\PYGZgt{}}
\end{sphinxVerbatim}

\item {} 
\sphinxAtStartPar
定义单页模板,每个模板的名称必须唯一(\sphinxcode{\sphinxupquote{master\sphinxhyphen{}name}}),

\begin{sphinxVerbatim}[commandchars=\\\{\}]

\PYGZlt{}fo:simple\PYGZhy{}page\PYGZhy{}master master\PYGZhy{}name=\PYGZdq{}A4\PYGZhy{}portrait\PYGZdq{}\PYGZgt{}
  \PYGZlt{}!\PYGZhy{}\PYGZhy{} One page template goes here \PYGZhy{}\PYGZhy{}\PYGZgt{}
\PYGZlt{}/fo:simple\PYGZhy{}page\PYGZhy{}master\PYGZgt{}
\end{sphinxVerbatim}

\sphinxAtStartPar
这里定义的就是竖版A4的样式。需要注意的是这里的A4只是一个名字,软件并不能通过A4这个名称就知道具体的定义,所以我们依然需要描述什么是竖版A4样式。A4纸的ISO 216的定义规格是29.7CM ×21.0CM,翻译为 xml 语句为:

\begin{sphinxVerbatim}[commandchars=\\\{\}]
page\PYGZhy{}height=\PYGZdq{}29.7cm\PYGZdq{}\PYG{+w}{ }page\PYGZhy{}width=\PYGZdq{}21.0cm\PYGZdq{}
\end{sphinxVerbatim}

\item {} 
\sphinxAtStartPar
定义各区块的属性。

\sphinxAtStartPar
在解释具体属性之前,我们先看一下FO页面的区块布局。
\begin{itemize}
\item {} 
\sphinxAtStartPar
页面主体 \sphinxcode{\sphinxupquote{region\sphinxhyphen{}body}}

\item {} 
\sphinxAtStartPar
页眉\sphinxcode{\sphinxupquote{region\sphinxhyphen{}before}}

\item {} 
\sphinxAtStartPar
页脚\sphinxcode{\sphinxupquote{region\sphinxhyphen{}after}}

\item {} 
\sphinxAtStartPar
左侧边 \sphinxcode{\sphinxupquote{region\sphinxhyphen{}start}}

\item {} 
\sphinxAtStartPar
右侧边\sphinxcode{\sphinxupquote{region\sphinxhyphen{}end}}

\end{itemize}

\sphinxAtStartPar
\sphinxincludegraphics{{fo-regions}.jpeg}

\item {} 
\sphinxAtStartPar
定义fl ow和block

\sphinxAtStartPar
页面和区域定义好之后,就需要定义Flow 和 Block的信息了。flow是由block组成的。

\sphinxAtStartPar
在正文区定一个 flow,飞入到 “xsl\sphinxhyphen{}region\sphinxhyphen{}body”区域,这个任务翻译为xml语句,则为:

\begin{sphinxVerbatim}[commandchars=\\\{\}]
    \PYGZlt{}fo:flow flow\PYGZhy{}name=\PYGZdq{}xsl\PYGZhy{}region\PYGZhy{}body\PYGZdq{}\PYGZgt{}
      \PYGZlt{}fo:block\PYGZgt{}
        \PYGZlt{}!\PYGZhy{}\PYGZhy{} Output goes here \PYGZhy{}\PYGZhy{}\PYGZgt{}
      \PYGZlt{}/fo:block\PYGZgt{}
    \PYGZlt{}/fo:flow\PYGZgt{}
\end{sphinxVerbatim}

\item {} 
\sphinxAtStartPar
定义标题Block的属性。
Flow中的每个内容都是一个block,都需要分别来定义。

\sphinxAtStartPar
\sphinxincludegraphics{{block-models}.png}

\begin{sphinxVerbatim}[commandchars=\\\{\}]
\PYG{o}{\PYGZlt{}}\PYG{n}{fo}\PYG{p}{:}\PYG{n}{block}
\PYG{n}{border}\PYG{o}{\PYGZhy{}}\PYG{n}{width}\PYG{o}{=}\PYG{l+s+s2}{\PYGZdq{}}\PYG{l+s+s2}{1mm}\PYG{l+s+s2}{\PYGZdq{}} 
\PYG{n}{font}\PYG{o}{\PYGZhy{}}\PYG{n}{size}\PYG{o}{=}\PYG{l+s+s2}{\PYGZdq{}}\PYG{l+s+s2}{20pt}\PYG{l+s+s2}{\PYGZdq{}}
\PYG{n}{color}\PYG{o}{=}\PYG{l+s+s2}{\PYGZdq{}}\PYG{l+s+s2}{red}\PYG{l+s+s2}{\PYGZdq{}}
\PYG{n}{font}\PYG{o}{\PYGZhy{}}\PYG{n}{family}\PYG{o}{=}\PYG{l+s+s2}{\PYGZdq{}}\PYG{l+s+s2}{Microsoft YaHei}\PYG{l+s+s2}{\PYGZdq{}}
\PYG{n}{text}\PYG{o}{\PYGZhy{}}\PYG{n}{decoration}\PYG{o}{=}\PYG{l+s+s2}{\PYGZdq{}}\PYG{l+s+s2}{underline}\PYG{l+s+s2}{\PYGZdq{}}\PYG{o}{\PYGZgt{}}
\PYG{n}{Brief} \PYG{n}{Introduction}
\PYG{o}{\PYGZlt{}}\PYG{o}{/}\PYG{n}{fo}\PYG{p}{:}\PYG{n}{block}\PYG{o}{\PYGZgt{}}
  
\end{sphinxVerbatim}

\item {} 
\sphinxAtStartPar
显示XML的值。

\sphinxAtStartPar
例如名片的名字信息,我们定义了其block属性后,还需要将其xml中对应值读取过来,这里需要使用XPath预计。

\begin{sphinxVerbatim}[commandchars=\\\{\}]
\PYG{+w}{  }\PYG{n+nt}{\PYGZlt{}xsl:value\PYGZhy{}of}\PYG{+w}{ }\PYG{n+na}{select=}\PYG{l+s}{\PYGZdq{}businesscard/name\PYGZdq{}}\PYG{n+nt}{/\PYGZgt{}}
\end{sphinxVerbatim}

\item {} 
\sphinxAtStartPar
使用FOP将XML输出为所需要的PDF。

\end{enumerate}


\section{参考资料}
\label{\detokenize{formatting/XSL-FO:id4}}

\subsection{xsl:fo 教程}
\label{\detokenize{formatting/XSL-FO:id5}}\begin{itemize}
\item {} 
\sphinxAtStartPar
\sphinxhref{https://w3schools.sinsixx.com/xslfo/default.asp.htm}{XSL\sphinxhyphen{}FOT Tutorial by W3C School}

\item {} 
\sphinxAtStartPar
\sphinxhref{http://www.renderx.com/tutorial.html\#Hello\_World}{XSL Formatting Objects Tutorial}

\end{itemize}


\subsection{xsl:fo 参考资料}
\label{\detokenize{formatting/XSL-FO:id6}}\begin{itemize}
\item {} 
\sphinxAtStartPar
\sphinxhref{http://www.ibiblio.org/xml/books/bible3/chapters/ch16.html}{XML Bible Chapter 16}

\end{itemize}


\subsection{发布引擎}
\label{\detokenize{formatting/XSL-FO:id7}}\begin{itemize}
\item {} 
\sphinxAtStartPar
\sphinxhref{https://xmlgraphics.apache.org/fop/}{Apache™ FOP}

\item {} 
\sphinxAtStartPar
\sphinxhref{https://www.antennahouse.com/}{ANTENNA House}

\end{itemize}

\sphinxstepscope


\chapter{DITA Plugins}
\label{\detokenize{formatting/dita-plugins:dita-plugins}}\label{\detokenize{formatting/dita-plugins::doc}}
\sphinxAtStartPar
在开发DITA样式的时候,一个比较好的实践就是把不同样式制作为插件,例如用户手册插件,参考手册插件,WebHelp插件等。这样在使用的时候比较方便,例如可以让你的同事使用你的样式发布,不需要额外配置,plugin中有全部所需的内容。


\section{实例}
\label{\detokenize{formatting/dita-plugins:id1}}
\sphinxAtStartPar
oXygem XML 将他们公司的手册的样式作为插件分享出来了,我们可以尝试安装他们的样式插件,并用于发布自己的内容。

\sphinxAtStartPar
步骤:
\begin{enumerate}
\sphinxsetlistlabels{\arabic}{enumi}{enumii}{}{.}%
\item {} 
\sphinxAtStartPar
下载 \sphinxhref{https://github.com/oxygenxml/com.oxygenxml.pdf2.ug}{oXygen XML 插件}

\item {} 
\sphinxAtStartPar
安装插件

\sphinxAtStartPar
\sphinxcode{\sphinxupquote{dita install com.oxygenxml.pdf2.ug\sphinxhyphen{}master.zip}}

\sphinxAtStartPar
提示:Added com.oxygenxml.pdf2.ug

\item {} 
\sphinxAtStartPar
使用该插件发布内容。格式为(oxy\sphinxhyphen{}ug\sphinxhyphen{}pdf)

\sphinxAtStartPar
\sphinxcode{\sphinxupquote{dita \sphinxhyphen{}\sphinxhyphen{}input=projector\_user\_manual.ditamap \sphinxhyphen{}\sphinxhyphen{}format=oxy\sphinxhyphen{}ug\sphinxhyphen{}pdf}}

\sphinxAtStartPar
提示:需要定义中文字体,否则会出现乱码

\end{enumerate}


\section{其他插件}
\label{\detokenize{formatting/dita-plugins:id2}}\begin{enumerate}
\sphinxsetlistlabels{\arabic}{enumi}{enumii}{}{.}%
\item {} 
\sphinxAtStartPar
\sphinxhref{https://github.com/jelovirt/com.elovirta.ooxml}{DITA to Word plug\sphinxhyphen{}in}

\item {} 
\sphinxAtStartPar
\sphinxhref{https://github.com/infotexture/dita-bootstrap}{net.infotexture.dita\sphinxhyphen{}bootstrap},\sphinxhref{https://infotexture.github.io/dita-bootstrap/alerts.html}{demo}, \sphinxhref{https://themestr.app/theme}{Themestr.app}

\item {} 
\end{enumerate}

\sphinxstepscope


\chapter{DITA PDF 样式plugin制作}
\label{\detokenize{formatting/dita_plugin-dev:dita-pdf-plugin}}\label{\detokenize{formatting/dita_plugin-dev::doc}}
\sphinxAtStartPar
本文介绍开发\sphinxcode{\sphinxupquote{DITA\sphinxhyphen{}OT}}样式plugin的方法。oXygen 发布内容的引擎,也是基于\sphinxcode{\sphinxupquote{DITA\sphinxhyphen{}OT}}。本文以MacOS系统,并使用HomeBrew安装了DITA\sphinxhyphen{}OT为例来说明样式表的开发。其他系统或其他\sphinxcode{\sphinxupquote{dita\sphinxhyphen{}ot}}的安装形式,插件开发大同小异,只是安装方式或调用方式有所不同。


\section{前提条件}
\label{\detokenize{formatting/dita_plugin-dev:id1}}
\sphinxAtStartPar
已经安装好 \sphinxcode{\sphinxupquote{dita\sphinxhyphen{}ot}}	,如何安装\sphinxcode{\sphinxupquote{DITA\sphinxhyphen{}OT}},请参见\sphinxhref{https://www.dita-ot.org/dev/topics/installing-client.html}{官方文档}。


\section{org.dita.pdf2插件}
\label{\detokenize{formatting/dita_plugin-dev:org-dita-pdf2}}

\subsection{找到插件所在位置}
\label{\detokenize{formatting/dita_plugin-dev:id2}}
\sphinxAtStartPar
HomeBrew安装 \sphinxcode{\sphinxupquote{dita\sphinxhyphen{}ot}}文件夹所在位置,一般在如下几个位置。

\begin{sphinxVerbatim}[commandchars=\\\{\}]
\PYG{o}{/}\PYG{n}{usr}\PYG{o}{/}\PYG{n}{local}\PYG{o}{/}\PYG{n}{Cellar}\PYG{o}{/}
\PYG{o}{/}\PYG{n}{usr}\PYG{o}{/}\PYG{n}{local}\PYG{o}{/}\PYG{n}{opt}\PYG{o}{/}
\PYG{o}{/}\PYG{n}{usr}\PYG{o}{/}\PYG{n}{local}\PYG{o}{/}\PYG{n+nb}{bin}\PYG{o}{/}
\end{sphinxVerbatim}

\sphinxAtStartPar
以笔者系统为例:
\begin{enumerate}
\sphinxsetlistlabels{\arabic}{enumi}{enumii}{}{.}%
\item {} 
\sphinxAtStartPar
在Finder\sphinxhyphen{}>Go\sphinxhyphen{}>Go to Folder,然后输入

\end{enumerate}

\begin{sphinxVerbatim}[commandchars=\\\{\}]
\PYG{o}{/}\PYG{n}{usr}\PYG{o}{/}\PYG{n}{local}\PYG{o}{/}\PYG{n}{Cellar}\PYG{o}{/}
\end{sphinxVerbatim}
\begin{enumerate}
\sphinxsetlistlabels{\arabic}{enumi}{enumii}{}{.}%
\setcounter{enumi}{1}
\item {} 
\sphinxAtStartPar
在Cellar文件夹中按照下方路径,继续浏览

\end{enumerate}

\begin{sphinxVerbatim}[commandchars=\\\{\}]
\PYG{o}{/}\PYG{n}{dita}\PYG{o}{\PYGZhy{}}\PYG{n}{ot}\PYG{o}{/}\PYG{l+m+mf}{3.6}\PYG{l+m+mf}{.1}\PYG{o}{/}\PYG{n}{libexec}\PYG{o}{/}\PYG{n}{plugins}\PYG{o}{/}\PYG{n}{org}\PYG{o}{.}\PYG{n}{dita}\PYG{o}{.}\PYG{n}{pdf2}
\end{sphinxVerbatim}


\subsection{文件夹结构}
\label{\detokenize{formatting/dita_plugin-dev:id3}}
\sphinxAtStartPar
\sphinxincludegraphics{{plugin-folder-structure}.png}


\begin{savenotes}\sphinxattablestart
\sphinxthistablewithglobalstyle
\centering
\begin{tabulary}{\linewidth}[t]{TT}
\sphinxtoprule
\sphinxstyletheadfamily 
\sphinxAtStartPar
项目
&\sphinxstyletheadfamily 
\sphinxAtStartPar
含义
\\
\sphinxmidrule
\sphinxtableatstartofbodyhook
\sphinxAtStartPar
cfg
&
\sphinxAtStartPar
存放样式的主要定义信息,其中还包括 \sphinxcode{\sphinxupquote{common}} 和 \sphinxcode{\sphinxupquote{fo}} 两个文件夹
\\
\sphinxhline
\sphinxAtStartPar
Customization
&
\sphinxAtStartPar
存放用户自定义的样式信息
\\
\sphinxhline
\sphinxAtStartPar
lib
&
\sphinxAtStartPar
存放 Java 可执行程序 \sphinxcode{\sphinxupquote{fo.jar}}
\\
\sphinxhline
\sphinxAtStartPar
resource
&
\sphinxAtStartPar

\\
\sphinxhline
\sphinxAtStartPar
xsl
&
\sphinxAtStartPar
存放 XSLT stylesheet
\\
\sphinxbottomrule
\end{tabulary}
\sphinxtableafterendhook\par
\sphinxattableend\end{savenotes}


\section{创建PDF plugin}
\label{\detokenize{formatting/dita_plugin-dev:pdf-plugin}}\begin{enumerate}
\sphinxsetlistlabels{\arabic}{enumi}{enumii}{}{.}%
\item {} 
\sphinxAtStartPar
在plugins文件中,创建 \sphinxcode{\sphinxupquote{com.company.pdf}}文件夹。
\begin{quote}

\sphinxAtStartPar
Java的命名传统,例如你的公司叫deepdok,可以用 com.deepdok.pdf来命名。还可以进一步增加说明,例如用户手册 (ug)的样式,可以写成。com.deepdok.pdf.ug
\end{quote}

\item {} 
\sphinxAtStartPar
模仿 \sphinxcode{\sphinxupquote{org.dita.pdf}}的文件夹结构分别创建 \sphinxcode{\sphinxupquote{cfg}} ,\sphinxcode{\sphinxupquote{fo}}等文件夹和文件夹。

\item {} 
\sphinxAtStartPar
在\sphinxcode{\sphinxupquote{cfg}}文件夹中,新建catalog.xml,并在其中输入如下代码

\begin{sphinxVerbatim}[commandchars=\\\{\}]
\PYG{c+cp}{\PYGZlt{}?xml version=\PYGZdq{}1.0\PYGZdq{} encoding=\PYGZdq{}UTF\PYGZhy{}8\PYGZdq{}?\PYGZgt{}}
\PYG{n+nt}{\PYGZlt{}catalog}\PYG{+w}{ }\PYG{n+na}{prefer=}\PYG{l+s}{\PYGZdq{}system\PYGZdq{}}\PYG{+w}{ }\PYG{n+na}{xmlns=}\PYG{l+s}{\PYGZdq{}urn:oasis:names:tc:entity:xmlns:xml:catalog\PYGZdq{}}\PYG{n+nt}{\PYGZgt{}}
\PYG{+w}{  }\PYG{n+nt}{\PYGZlt{}uri}\PYG{+w}{ }\PYG{n+na}{name=}\PYG{l+s}{\PYGZdq{}cfg:fo/attrs/custom.xsl\PYGZdq{}}\PYG{+w}{ }\PYG{n+na}{uri=}\PYG{l+s}{\PYGZdq{}fo/attrs/custom.xsl\PYGZdq{}}\PYG{n+nt}{/\PYGZgt{}}
\PYG{+w}{  }\PYG{n+nt}{\PYGZlt{}uri}\PYG{+w}{ }\PYG{n+na}{name=}\PYG{l+s}{\PYGZdq{}cfg:fo/xsl/custom.xsl\PYGZdq{}}\PYG{+w}{ }\PYG{n+na}{uri=}\PYG{l+s}{\PYGZdq{}fo/xsl/custom.xsl\PYGZdq{}}\PYG{n+nt}{/\PYGZgt{}}
\PYG{n+nt}{\PYGZlt{}/catalog\PYGZgt{}}
\end{sphinxVerbatim}

\item {} 
\sphinxAtStartPar
在\sphinxcode{\sphinxupquote{fo/attrs}}创建 custom.xsl文件,并输入如下内容

\begin{sphinxVerbatim}[commandchars=\\\{\}]
\PYGZlt{}?xml version=\PYGZdq{}1.0\PYGZdq{}?\PYGZgt{}
\PYGZlt{}xsl:stylesheet xmlns:xsl=\PYGZdq{}http://www.w3.org/1999/XSL/Transform\PYGZdq{}
    xmlns:fo=\PYGZdq{}http://www.w3.org/1999/XSL/Format\PYGZdq{}
    version=\PYGZdq{}2.0\PYGZdq{}\PYGZgt{}
\PYGZlt{}/xsl:stylesheet\PYGZgt{}
\end{sphinxVerbatim}

\item {} 
\sphinxAtStartPar
将 custom.xsl 复制到 fo/xsl中。虽然名字一样,但是用途不一样,以后通过 custom.xsl (attrs) 和 custom.xsl(xsl)来区分。

\item {} 
\sphinxAtStartPar
在\sphinxcode{\sphinxupquote{com.company.pdf}}中新建plugin.xml,并输入如下内容

\begin{sphinxVerbatim}[commandchars=\\\{\}]
\PYG{c+cp}{\PYGZlt{}?xml version=\PYGZsq{}1.0\PYGZsq{} encoding=\PYGZsq{}utf\PYGZhy{}8\PYGZsq{}?\PYGZgt{}}
\PYG{n+nt}{\PYGZlt{}plugin}\PYG{+w}{ }\PYG{n+na}{id=}\PYG{l+s}{\PYGZdq{}com.company.pdf\PYGZdq{}}\PYG{n+nt}{\PYGZgt{}}
\PYG{+w}{    }\PYG{n+nt}{\PYGZlt{}require}\PYG{+w}{ }\PYG{n+na}{plugin=}\PYG{l+s}{\PYGZdq{}org.dita.pdf2\PYGZdq{}}\PYG{+w}{ }\PYG{n+nt}{/\PYGZgt{}}
\PYG{+w}{    }\PYG{n+nt}{\PYGZlt{}feature}\PYG{+w}{ }\PYG{n+na}{extension=}\PYG{l+s}{\PYGZdq{}dita.conductor.transtype.check\PYGZdq{}}\PYG{+w}{ }\PYG{n+na}{value=}\PYG{l+s}{\PYGZdq{}custpdf\PYGZdq{}}\PYG{+w}{ }\PYG{n+nt}{/\PYGZgt{}}
\PYG{+w}{    }\PYG{n+nt}{\PYGZlt{}feature}\PYG{+w}{ }\PYG{n+na}{extension=}\PYG{l+s}{\PYGZdq{}dita.transtype.print\PYGZdq{}}\PYG{+w}{ }\PYG{n+na}{value=}\PYG{l+s}{\PYGZdq{}custpdf\PYGZdq{}}\PYG{+w}{ }\PYG{n+nt}{/\PYGZgt{}}
\PYG{+w}{    }\PYG{n+nt}{\PYGZlt{}feature}\PYG{+w}{ }\PYG{n+na}{extension=}\PYG{l+s}{\PYGZdq{}dita.conductor.target.relative\PYGZdq{}}\PYG{+w}{ }\PYG{n+na}{file=}\PYG{l+s}{\PYGZdq{}integrator.xml\PYGZdq{}}\PYG{+w}{ }\PYG{n+nt}{/\PYGZgt{}}
\PYG{n+nt}{\PYGZlt{}/plugin\PYGZgt{}}
\end{sphinxVerbatim}
\begin{quote}

\sphinxAtStartPar
\sphinxcode{\sphinxupquote{dita.conductor.transtype.check}}和\sphinxcode{\sphinxupquote{dita.transtype.print}}的值为 \sphinxcode{\sphinxupquote{custpdf}},意思是使用 \sphinxcode{\sphinxupquote{custpdf}}作为转换类型来调用。
\end{quote}

\item {} 
\sphinxAtStartPar
在\sphinxcode{\sphinxupquote{com.company.pdf}}中新建integrator.xml,并输入如下内容

\begin{sphinxVerbatim}[commandchars=\\\{\}]
\PYG{c+cp}{\PYGZlt{}?xml version=\PYGZsq{}1.0\PYGZsq{} encoding=\PYGZsq{}utf\PYGZhy{}8\PYGZsq{}?\PYGZgt{}}
\PYG{n+nt}{\PYGZlt{}project}\PYG{+w}{ }\PYG{n+na}{name=}\PYG{l+s}{\PYGZdq{}com.company.pdf\PYGZdq{}}\PYG{n+nt}{\PYGZgt{}}
\PYG{+w}{    }\PYG{n+nt}{\PYGZlt{}target}\PYG{+w}{ }\PYG{n+na}{name=}\PYG{l+s}{\PYGZdq{}dita2custpdf.init\PYGZdq{}}\PYG{n+nt}{\PYGZgt{}}
\PYG{+w}{        }\PYG{n+nt}{\PYGZlt{}property}\PYG{+w}{ }\PYG{n+na}{name=}\PYG{l+s}{\PYGZdq{}customization.dir\PYGZdq{}}\PYG{+w}{ }\PYG{n+na}{location=}\PYG{l+s}{\PYGZdq{}\PYGZdl{}\PYGZob{}dita.plugin.com.company.pdf.dir\PYGZcb{}/cfg\PYGZdq{}}\PYG{n+nt}{/\PYGZgt{}}
\PYG{+w}{        }\PYG{n+nt}{\PYGZlt{}property}\PYG{+w}{ }\PYG{n+na}{name=}\PYG{l+s}{\PYGZdq{}pdf2.i18n.skip\PYGZdq{}}\PYG{+w}{ }\PYG{n+na}{value=}\PYG{l+s}{\PYGZdq{}true\PYGZdq{}}\PYG{n+nt}{/\PYGZgt{}}
\PYG{+w}{    }\PYG{n+nt}{\PYGZlt{}/target\PYGZgt{}}
\PYG{+w}{    }\PYG{n+nt}{\PYGZlt{}target}\PYG{+w}{ }\PYG{n+na}{name=}\PYG{l+s}{\PYGZdq{}dita2custpdf\PYGZdq{}}\PYG{+w}{ }\PYG{n+na}{depends=}\PYG{l+s}{\PYGZdq{}dita2custpdf.init, dita2pdf2\PYGZdq{}}\PYG{n+nt}{/\PYGZgt{}}
\PYG{n+nt}{\PYGZlt{}/project\PYGZgt{}}
\end{sphinxVerbatim}
\begin{quote}

\sphinxAtStartPar
Project name 对应插件文件夹
\end{quote}

\item {} 
\sphinxAtStartPar
将 \sphinxcode{\sphinxupquote{org.dita.pdf2/cfg/fo}}中的\sphinxcode{\sphinxupquote{layout\sphinxhyphen{}master.xsl}} 复制一份到 \sphinxcode{\sphinxupquote{com.company.pdf/cfg/fo/}}

\end{enumerate}

\sphinxAtStartPar
最后的文件夹结构图

\sphinxAtStartPar
\sphinxincludegraphics{{plugin-structure}.png}
\begin{enumerate}
\sphinxsetlistlabels{\arabic}{enumi}{enumii}{}{.}%
\setcounter{enumi}{8}
\item {} 
\sphinxAtStartPar
注册插件,在plugins文件中,运行 \sphinxcode{\sphinxupquote{dita install}},即可注册。

\end{enumerate}


\section{参考教程}
\label{\detokenize{formatting/dita_plugin-dev:id4}}
\sphinxAtStartPar
\sphinxhref{https://www.dita-ot.org/dev/topics/pdf-customization-example.html}{DITA OT Creating a simple PDF plug\sphinxhyphen{}in}


\section{相关资源}
\label{\detokenize{formatting/dita_plugin-dev:id5}}\begin{itemize}
\item {} 
\sphinxAtStartPar
\sphinxhref{https://dita-generator.elovirta.com}{PDF Plugin Generator}

\item {} 
\sphinxAtStartPar
\sphinxhref{https://github.com/oxygenxml/dita-ot-css-pdf}{DITA OT CSS PDF}

\item {} 
\sphinxAtStartPar
\sphinxhref{https://github.com/oxygenxml/com.oxygenxml.pdf2.ug}{DITA\sphinxhyphen{}OT PDF Customization Plugin for oXygen User Manual}

\end{itemize}

\sphinxstepscope


\chapter{WebHelp}
\label{\detokenize{formatting/webhelp:webhelp}}\label{\detokenize{formatting/webhelp::doc}}

\section{皮肤构构建器}
\label{\detokenize{formatting/webhelp:id1}}
\sphinxAtStartPar
\sphinxhref{https://www.oxygenxml.com/webhelp-skin-builder}{oXygen Webhelp Skin Builder}

\sphinxstepscope


\chapter{文档版面设计}
\label{\detokenize{formatting/layout-design:id1}}\label{\detokenize{formatting/layout-design::doc}}

\section{设计原则 (CRAP)}
\label{\detokenize{formatting/layout-design:crap}}

\subsection{Contrast}
\label{\detokenize{formatting/layout-design:contrast}}\begin{enumerate}
\sphinxsetlistlabels{\arabic}{enumi}{enumii}{}{.}%
\item {} 
\sphinxAtStartPar
增加对比度,可以让页面更有吸引力,同时也能增加页面的可读性;

\item {} 
\sphinxAtStartPar
大小、前后、粗细等都可以形成对比;

\end{enumerate}


\subsection{Repetition}
\label{\detokenize{formatting/layout-design:repetition}}\begin{enumerate}
\sphinxsetlistlabels{\arabic}{enumi}{enumii}{}{.}%
\item {} 
\sphinxAtStartPar
重复可以形成一致性;

\item {} 
\sphinxAtStartPar
字体、色彩、线条等都可以是重复的元素;

\end{enumerate}


\subsection{Alignment}
\label{\detokenize{formatting/layout-design:alignment}}\begin{enumerate}
\sphinxsetlistlabels{\arabic}{enumi}{enumii}{}{.}%
\item {} 
\sphinxAtStartPar
对齐可以形成秩序和美感;

\item {} 
\sphinxAtStartPar
左对齐、右对齐、两端对齐、或居中对齐等,坚持一种对齐方式;

\end{enumerate}


\subsection{Proximity}
\label{\detokenize{formatting/layout-design:proximity}}\begin{enumerate}
\sphinxsetlistlabels{\arabic}{enumi}{enumii}{}{.}%
\item {} 
\sphinxAtStartPar
相似或有关系的内容考得更近一些;

\item {} 
\sphinxAtStartPar
大脑天然会认为靠的近的事物是有关系的

\end{enumerate}


\section{配色}
\label{\detokenize{formatting/layout-design:id2}}
\sphinxAtStartPar
使用 \sphinxhref{https://color.adobe.com/create/color-wheel}{Adobe Color} 色轮


\section{字体}
\label{\detokenize{formatting/layout-design:id3}}\begin{enumerate}
\sphinxsetlistlabels{\arabic}{enumi}{enumii}{}{.}%
\item {} 
\sphinxAtStartPar
\sphinxhref{https://style-guide-statr.webflow.io/\#Typography}{Startr Typography}

\end{enumerate}


\section{Layout}
\label{\detokenize{formatting/layout-design:layout}}\begin{enumerate}
\sphinxsetlistlabels{\arabic}{enumi}{enumii}{}{.}%
\item {} 
\sphinxAtStartPar
\sphinxhref{https://design-system.service.gov.uk/styles/layout/}{GOV UK Layout}

\item {} 
\sphinxAtStartPar
\sphinxhref{https://design-system.service.gov.uk/styles/page-template/}{GOV UK Page Layout}

\item {} 
\sphinxAtStartPar
\sphinxhref{https://designsystem.gov.scot/styles/layout/}{Digital Scottland Layout}

\end{enumerate}


\section{Design System}
\label{\detokenize{formatting/layout-design:design-system}}\begin{enumerate}
\sphinxsetlistlabels{\arabic}{enumi}{enumii}{}{.}%
\item {} 
\sphinxAtStartPar
\sphinxhref{https://www.carbondesignsystem.com/guidelines/color/overview/}{IBM Carbon Design System}

\end{enumerate}

\sphinxstepscope


\chapter{通过CSS定制PDF输出样式}
\label{\detokenize{formatting/css-pdf:csspdf}}\label{\detokenize{formatting/css-pdf::doc}}

\section{排版基础}
\label{\detokenize{formatting/css-pdf:id1}}

\subsection{字体}
\label{\detokenize{formatting/css-pdf:id2}}
\sphinxAtStartPar
当我们选择加粗的时候,实际上应用的是粗体的字集,所以一般都要提前准备好对应的字库。

\sphinxAtStartPar
本次练习,我们选择MI Sans 作为汉字字体,下载:\sphinxhref{https://hyperos.mi.com/font/en/download/}{MI Sans}


\section{定制模板}
\label{\detokenize{formatting/css-pdf:id3}}\begin{enumerate}
\sphinxsetlistlabels{\arabic}{enumi}{enumii}{}{.}%
\item {} 
\sphinxAtStartPar
导出模板。

\end{enumerate}


\section{PDF外观定制}
\label{\detokenize{formatting/css-pdf:pdf}}

\subsection{导入字体}
\label{\detokenize{formatting/css-pdf:id4}}
\sphinxAtStartPar
\sphinxstylestrong{\sphinxcode{\sphinxupquote{@font\sphinxhyphen{}face}}} 是一个 CSS 规则,用于自定义字体,让开发者可以将特定的字体文件加载到PDF中,而不依赖用户系统中预装的字体。

\sphinxAtStartPar
这里我们引入MISans的不同字重(Weight)的字体,分别是:
\begin{itemize}
\item {} 
\sphinxAtStartPar
Normal

\item {} 
\sphinxAtStartPar
Bold

\item {} 
\sphinxAtStartPar
Thin

\end{itemize}

\begin{sphinxVerbatim}[commandchars=\\\{\}]
\PYG{p}{@}\PYG{k}{font\PYGZhy{}face}\PYG{+w}{ }\PYG{p}{\PYGZob{}}
\PYG{+w}{	}\PYG{n+nt}{font\PYGZhy{}family}\PYG{o}{:}\PYG{+w}{ }\PYG{n+nt}{xiaomi\PYGZhy{}normal}\PYG{o}{;}
\PYG{+w}{	}\PYG{n+nt}{src}\PYG{o}{:}\PYG{+w}{ }\PYG{n+nt}{url}\PYG{o}{(}\PYG{n+nt}{resources}\PYG{o}{/}\PYG{n+nt}{fonts}\PYG{o}{/}\PYG{n+nt}{MiSans\PYGZhy{}Normal}\PYG{p}{.}\PYG{n+nc}{ttf}\PYG{o}{)}\PYG{o}{;}
\PYG{p}{\PYGZcb{}}

\PYG{p}{@}\PYG{k}{font\PYGZhy{}face}\PYG{+w}{ }\PYG{p}{\PYGZob{}}
\PYG{+w}{	}\PYG{n+nt}{font\PYGZhy{}family}\PYG{o}{:}\PYG{+w}{ }\PYG{n+nt}{xiaomi\PYGZhy{}bold}\PYG{o}{;}
\PYG{+w}{	}\PYG{n+nt}{src}\PYG{o}{:}\PYG{+w}{ }\PYG{n+nt}{url}\PYG{o}{(}\PYG{n+nt}{resources}\PYG{o}{/}\PYG{n+nt}{fonts}\PYG{o}{/}\PYG{n+nt}{MiSans\PYGZhy{}Bold}\PYG{p}{.}\PYG{n+nc}{ttf}\PYG{o}{)}\PYG{o}{;}
\PYG{p}{\PYGZcb{}}


\PYG{p}{@}\PYG{k}{font\PYGZhy{}face}\PYG{+w}{ }\PYG{p}{\PYGZob{}}
\PYG{+w}{	}\PYG{n+nt}{font\PYGZhy{}family}\PYG{o}{:}\PYG{+w}{ }\PYG{n+nt}{xiaomi\PYGZhy{}thin}\PYG{o}{;}
\PYG{+w}{	}\PYG{n+nt}{src}\PYG{o}{:}\PYG{+w}{ }\PYG{n+nt}{url}\PYG{o}{(}\PYG{n+nt}{resources}\PYG{o}{/}\PYG{n+nt}{fonts}\PYG{o}{/}\PYG{n+nt}{MiSans\PYGZhy{}Thin}\PYG{p}{.}\PYG{n+nc}{ttf}\PYG{o}{)}\PYG{o}{;}
\PYG{p}{\PYGZcb{}}
\end{sphinxVerbatim}


\subsection{指定正文字体}
\label{\detokenize{formatting/css-pdf:id5}}
\sphinxAtStartPar
正文字体决定了整个文档的观感,因为正文是全手册文字最多的。

\begin{sphinxVerbatim}[commandchars=\\\{\}]
\PYG{c}{/* 用正常字体 */}

\PYG{n+nt}{body}\PYG{+w}{ }\PYG{p}{\PYGZob{}}
\PYG{+w}{	}\PYG{k}{font\PYGZhy{}size}\PYG{p}{:}\PYG{+w}{ }\PYG{l+m+mi}{10}\PYG{k+kt}{pt}\PYG{p}{;}
\PYG{+w}{	}\PYG{k}{font\PYGZhy{}family}\PYG{p}{:}\PYG{+w}{ }\PYG{n}{xiaomi\PYGZhy{}normal}\PYG{p}{;}
\PYG{p}{\PYGZcb{}}

\PYG{c}{/* 用瘦体 */}

\PYG{n+nt}{body}\PYG{+w}{ }\PYG{p}{\PYGZob{}}
\PYG{+w}{	}\PYG{k}{font\PYGZhy{}size}\PYG{p}{:}\PYG{+w}{ }\PYG{l+m+mi}{10}\PYG{k+kt}{pt}\PYG{p}{;}
\PYG{+w}{	}\PYG{k}{font\PYGZhy{}family}\PYG{p}{:}\PYG{+w}{ }\PYG{n}{xiaomi\PYGZhy{}thin}\PYG{p}{;}
\PYG{p}{\PYGZcb{}}
\end{sphinxVerbatim}

\sphinxAtStartPar
大家感觉一下,哪个更高级,是不是瘦的字体更显高级?
\begin{quote}

\sphinxAtStartPar
S.H.E法则(纤细的,瘦弱的)
\end{quote}


\subsection{定制封面}
\label{\detokenize{formatting/css-pdf:id6}}
\sphinxAtStartPar
需要在封面添加下方三个要素:
\begin{enumerate}
\sphinxsetlistlabels{\arabic}{enumi}{enumii}{}{.}%
\item {} 
\sphinxAtStartPar
右上角的小米logo

\item {} 
\sphinxAtStartPar
正下方添加出品机构  Xiaomi Group

\item {} 
\sphinxAtStartPar
说明书标题调整(dita map的title)

\end{enumerate}

\sphinxAtStartPar
Page Media各区域的划分如下:

\sphinxAtStartPar
\sphinxincludegraphics{{css-page-area}.png}


\subsubsection{添加logo与出品机构}
\label{\detokenize{formatting/css-pdf:logo}}
\begin{sphinxVerbatim}[commandchars=\\\{\}]
\PYG{c}{/* 添加 logo */}
\PYG{c}{/* 添加出品机构 */}


\PYG{p}{@}\PYG{k}{page}\PYG{+w}{ }\PYG{n+nt}{front\PYGZhy{}page}\PYG{+w}{ }\PYG{p}{\PYGZob{}}
\PYG{+w}{	}\PYG{p}{@}\PYG{k}{top\PYGZhy{}right\PYGZhy{}corner}\PYG{+w}{ }\PYG{p}{\PYGZob{}}
\PYG{+w}{		}\PYG{n+nt}{content}\PYG{o}{:}\PYG{+w}{ }\PYG{n+nt}{url}\PYG{o}{(}\PYG{l+s+s2}{\PYGZdq{}resources/images/logo.png\PYGZdq{}}\PYG{o}{)}\PYG{o}{;}
\PYG{+w}{	}\PYG{p}{\PYGZcb{}}

\PYG{+w}{	}\PYG{p}{@}\PYG{k}{bottom\PYGZhy{}center}\PYG{p}{\PYGZob{}}
\PYG{+w}{		}\PYG{n+nt}{content}\PYG{o}{:}\PYG{+w}{ }\PYG{l+s+s2}{\PYGZdq{}Xiaomi Group\PYGZdq{}}\PYG{o}{;}
\PYG{+w}{	}\PYG{p}{\PYGZcb{}}

\PYG{p}{\PYGZcb{}}

\PYG{+w}{  }
\end{sphinxVerbatim}


\subsubsection{微调logo与出品机构}
\label{\detokenize{formatting/css-pdf:id7}}
\begin{sphinxVerbatim}[commandchars=\\\{\}]
\PYG{p}{@}\PYG{k}{page}\PYG{+w}{ }\PYG{n+nt}{front\PYGZhy{}page}\PYG{+w}{ }\PYG{p}{\PYGZob{}}
\PYG{+w}{	}\PYG{p}{@}\PYG{k}{top\PYGZhy{}right\PYGZhy{}corner}\PYG{+w}{ }\PYG{p}{\PYGZob{}}
\PYG{+w}{		}\PYG{n+nt}{content}\PYG{o}{:}\PYG{+w}{ }\PYG{n+nt}{url}\PYG{o}{(}\PYG{l+s+s2}{\PYGZdq{}resources/images/logo.png\PYGZdq{}}\PYG{o}{)}\PYG{o}{;}
\PYG{c}{/* 往下调整一下logo的位置 */}\PYG{+w}{		}
\PYG{+w}{		}\PYG{n+nt}{margin\PYGZhy{}top}\PYG{o}{:}\PYG{+w}{ }\PYG{n+nt}{30px}\PYG{o}{;}
\PYG{+w}{	}\PYG{p}{\PYGZcb{}}

\PYG{+w}{	}\PYG{p}{@}\PYG{k}{bottom\PYGZhy{}center}\PYG{p}{\PYGZob{}}
\PYG{+w}{		}\PYG{n+nt}{content}\PYG{o}{:}\PYG{+w}{ }\PYG{l+s+s2}{\PYGZdq{}Xiaomi Group \PYGZbs{}A 2024.11.11 \PYGZdq{}}\PYG{o}{;}
\PYG{c}{/* 微调字体 */}
\PYG{+w}{    }\PYG{n+nt}{white\PYGZhy{}space}\PYG{o}{:}\PYG{+w}{ }\PYG{n+nt}{pre}\PYG{o}{;}
\PYG{+w}{		}\PYG{n+nt}{font\PYGZhy{}size}\PYG{o}{:}\PYG{+w}{ }\PYG{n+nt}{10pt}\PYG{o}{;}
\PYG{+w}{		}\PYG{n+nt}{color}\PYG{o}{:}\PYG{+w}{ }\PYG{n+nt}{gray}\PYG{o}{;}
\PYG{+w}{		}\PYG{n+nt}{font\PYGZhy{}style}\PYG{o}{:}\PYG{+w}{ }\PYG{n+nt}{italic}\PYG{o}{;}
\PYG{+w}{	}\PYG{p}{\PYGZcb{}}

\PYG{p}{\PYGZcb{}}

\PYG{+w}{  }
\end{sphinxVerbatim}


\subsection{\sphinxstylestrong{\sphinxstyleliteralintitle{\sphinxupquote{\textbackslash{}A}} 的作用}}
\label{\detokenize{formatting/css-pdf:a}}\begin{itemize}
\item {} 
\sphinxAtStartPar
\sphinxstylestrong{\sphinxcode{\sphinxupquote{\textbackslash{}A}}} 是 CSS 的一种特殊字符表示法,用于在 \sphinxcode{\sphinxupquote{content}} 属性中插入“换行符”。

\item {} 
\sphinxAtStartPar
它的效果需要结合 \sphinxstylestrong{\sphinxcode{\sphinxupquote{white\sphinxhyphen{}space: pre}}} 或 \sphinxstylestrong{\sphinxcode{\sphinxupquote{white\sphinxhyphen{}space: pre\sphinxhyphen{}wrap}}} 属性才能实现换行。

\item {} 
\sphinxAtStartPar
工作原理

\sphinxAtStartPar
:
\begin{itemize}
\item {} 
\sphinxAtStartPar
当 CSS 渲染内容时,\sphinxcode{\sphinxupquote{content: "Line 1\textbackslash{}A Line 2";}} 会在“Line 1”和“Line 2”之间插入一个换行符。

\item {} 
\sphinxAtStartPar
如果没有启用 \sphinxcode{\sphinxupquote{white\sphinxhyphen{}space: pre}} 或类似值,换行符会被忽略。

\end{itemize}

\end{itemize}


\subsubsection{微调封面标题}
\label{\detokenize{formatting/css-pdf:id8}}
\sphinxAtStartPar
基本语法

\sphinxAtStartPar
\sphinxstylestrong{1. \sphinxcode{\sphinxupquote{*}}(通配选择器)}
\begin{itemize}
\item {} 
\sphinxAtStartPar
\sphinxstylestrong{作用}:选择所有元素。

\item {} 
\sphinxAtStartPar
\sphinxstylestrong{范围}:无论元素是 \sphinxcode{\sphinxupquote{div}}、\sphinxcode{\sphinxupquote{p}}、\sphinxcode{\sphinxupquote{h1}},只要它满足条件,都会应用此样式。

\end{itemize}

\sphinxAtStartPar
\sphinxstylestrong{2. \sphinxcode{\sphinxupquote{{[}class \textasciitilde{}= "front\sphinxhyphen{}page/front\sphinxhyphen{}page\sphinxhyphen{}title"{]}}}}
\begin{itemize}
\item {} 
\sphinxAtStartPar
\sphinxstylestrong{属性选择器}:匹配带有 \sphinxcode{\sphinxupquote{class}} 属性,且其值中包含 \sphinxstylestrong{以空格分隔的单词 \sphinxcode{\sphinxupquote{front\sphinxhyphen{}page/front\sphinxhyphen{}page\sphinxhyphen{}title}}} 的元素。

\item {} 
\sphinxAtStartPar
\sphinxstylestrong{条件}:\sphinxcode{\sphinxupquote{\textasciitilde{}=}} 操作符要求 \sphinxcode{\sphinxupquote{class}} 的值中存在一个单独的单词与 \sphinxcode{\sphinxupquote{front\sphinxhyphen{}page/front\sphinxhyphen{}page\sphinxhyphen{}title}} 完全匹配。

\end{itemize}

\sphinxAtStartPar
可以找到 \sphinxcode{\sphinxupquote{projector.merged.html}},在html中查找各元素的属性。

\begin{sphinxVerbatim}[commandchars=\\\{\}]
\PYG{o}{*}\PYG{o}{[}\PYG{n+nt}{class}\PYG{+w}{ }\PYG{o}{\PYGZti{}}\PYG{o}{=}\PYG{+w}{ }\PYG{l+s+s2}{\PYGZdq{}front\PYGZhy{}page/front\PYGZhy{}page\PYGZhy{}title\PYGZdq{}}\PYG{o}{]}\PYG{+w}{ }\PYG{p}{\PYGZob{}}
\PYG{+w}{    }\PYG{k}{font\PYGZhy{}family}\PYG{p}{:}\PYG{+w}{ }\PYG{n}{xiaomi\PYGZhy{}bold}\PYG{p}{;}
\PYG{+w}{	}\PYG{k}{font\PYGZhy{}size}\PYG{p}{:}\PYG{+w}{ }\PYG{l+m+mi}{26}\PYG{k+kt}{pt}\PYG{p}{;}

\PYG{p}{\PYGZcb{}}
\end{sphinxVerbatim}


\subsection{定制章节标题}
\label{\detokenize{formatting/css-pdf:id9}}
\sphinxAtStartPar
将标题设置为小米黄\sphinxcode{\sphinxupquote{\#FF6900}},同时调大字体。

\begin{sphinxVerbatim}[commandchars=\\\{\}]
\PYG{o}{*}\PYG{o}{[}\PYG{n+nt}{class}\PYG{+w}{ }\PYG{o}{\PYGZti{}}\PYG{o}{=}\PYG{+w}{ }\PYG{l+s+s2}{\PYGZdq{}topictitle1\PYGZdq{}}\PYG{o}{]}\PYG{+w}{ }\PYG{p}{\PYGZob{}}
\PYG{+w}{	}\PYG{k}{font\PYGZhy{}family}\PYG{p}{:}\PYG{+w}{ }\PYG{n}{xiaomi\PYGZhy{}normal}\PYG{p}{;}
\PYG{+w}{	}\PYG{k}{font\PYGZhy{}size}\PYG{p}{:}\PYG{+w}{ }\PYG{l+m+mi}{16}\PYG{k+kt}{pt}\PYG{p}{;}
\PYG{+w}{	}\PYG{k}{color}\PYG{p}{:}\PYG{+w}{ }\PYG{l+m+mh}{\PYGZsh{}FF6900}\PYG{p}{;}

\PYG{p}{\PYGZcb{}}

\PYG{o}{*}\PYG{o}{[}\PYG{n+nt}{class}\PYG{+w}{ }\PYG{o}{\PYGZti{}}\PYG{o}{=}\PYG{+w}{ }\PYG{l+s+s2}{\PYGZdq{}topictitle2\PYGZdq{}}\PYG{o}{]}\PYG{+w}{ }\PYG{p}{\PYGZob{}}
\PYG{+w}{	}\PYG{k}{font\PYGZhy{}family}\PYG{p}{:}\PYG{+w}{ }\PYG{n}{xiaomi\PYGZhy{}normal}\PYG{p}{;}

\PYG{+w}{	}\PYG{k}{font\PYGZhy{}size}\PYG{p}{:}\PYG{+w}{ }\PYG{l+m+mi}{14}\PYG{k+kt}{pt}\PYG{p}{;}
\PYG{+w}{	}\PYG{k}{color}\PYG{p}{:}\PYG{+w}{ }\PYG{l+m+mh}{\PYGZsh{}FF6900}\PYG{p}{;}

\PYG{p}{\PYGZcb{}}
\end{sphinxVerbatim}


\section{简化页眉页脚}
\label{\detokenize{formatting/css-pdf:id10}}
\sphinxAtStartPar
只保留章节标题和页码。

\sphinxAtStartPar
\sphinxstylestrong{\sphinxcode{\sphinxupquote{counter(page)}}}:
\begin{itemize}
\item {} 
\sphinxAtStartPar
动态计数器,表示当前页面的页码。

\item {} 
\sphinxAtStartPar
\sphinxstylestrong{\sphinxcode{\sphinxupquote{page}} 是 CSS 内置计数器},自动递增,无需显式定义。

\end{itemize}

\begin{sphinxVerbatim}[commandchars=\\\{\}]
\PYG{c}{/*从全局变量 chaptertitle 中获取动态内容。*/}


\PYG{p}{@}\PYG{k}{page}\PYG{+w}{ }\PYG{p}{:}\PYG{n+nd}{left}\PYG{+w}{ }\PYG{p}{\PYGZob{}}
\PYG{+w}{    }\PYG{p}{@}\PYG{k}{top\PYGZhy{}left}\PYG{+w}{ }\PYG{p}{\PYGZob{}}
\PYG{+w}{        }\PYG{n+nt}{content}\PYG{o}{:}\PYG{+w}{ }\PYG{n+nt}{string}\PYG{o}{(}\PYG{n+nt}{chaptertitle}\PYG{o}{)}\PYG{+w}{ }\PYG{l+s+s2}{\PYGZdq{} | \PYGZdq{}}\PYG{+w}{ }\PYG{n+nt}{counter}\PYG{o}{(}\PYG{n+nt}{page}\PYG{o}{)}\PYG{o}{;}\PYG{+w}{ }
\PYG{+w}{    }\PYG{p}{\PYGZcb{}}
\PYG{p}{\PYGZcb{}}

\PYG{p}{@}\PYG{k}{page}\PYG{+w}{ }\PYG{p}{:}\PYG{n+nd}{right}\PYG{p}{\PYGZob{}}
\PYG{+w}{    }\PYG{p}{@}\PYG{k}{top\PYGZhy{}right}\PYG{+w}{ }\PYG{p}{\PYGZob{}}
\PYG{+w}{        }\PYG{n+nt}{content}\PYG{o}{:}\PYG{+w}{ }\PYG{n+nt}{string}\PYG{o}{(}\PYG{n+nt}{chaptertitle}\PYG{o}{)}\PYG{+w}{ }\PYG{l+s+s2}{\PYGZdq{} | \PYGZdq{}}\PYG{+w}{ }\PYG{n+nt}{counter}\PYG{o}{(}\PYG{n+nt}{page}\PYG{o}{)}\PYG{o}{;}
\PYG{+w}{    }\PYG{p}{\PYGZcb{}}
\PYG{p}{\PYGZcb{}}
\end{sphinxVerbatim}


\section{傻瓜式定制模板}
\label{\detokenize{formatting/css-pdf:id11}}
\sphinxAtStartPar
\sphinxhref{https://styles.oxygenxml.com/?\_gl=1*1jy168z*\_ga*MTI4NzYyMjkxOC4xNzI5NDE3MzI5*\_ga\_CKSFNYE9EY*MTczMzY3MDc1Ni4zMC4xLjE3MzM2NzEwNzguNTkuMC4w*\_ga\_HEWSDXWJSN*MTczMzY3MDc1Ni4yOC4xLjE3MzM2NzEwNzguMC4wLjA.}{Styles Basket} 可以让用户通过图形界面简单操作即可生成基础模板,并可使用css进一步定制。


\section{参考}
\label{\detokenize{formatting/css-pdf:id12}}\begin{enumerate}
\sphinxsetlistlabels{\arabic}{enumi}{enumii}{}{.}%
\item {} 
\sphinxAtStartPar
\sphinxhref{https://www.oxygenxml.com/doc/versions/23.0/ug-ope/topics/dcpp\_the\_customization\_css.html}{Customizing PDF Output Using CSS}

\item {} 
\sphinxAtStartPar
\sphinxhref{https://hyperos.mi.com/font/en/download/}{MI Sans}

\item {} 
\sphinxAtStartPar
\sphinxhref{https://styles.oxygenxml.com/?\_gl=1*1mtx7jj*\_ga*MTI4NzYyMjkxOC4xNzI5NDE3MzI5*\_ga\_CKSFNYE9EY*MTczMzY3MDc1Ni4zMC4xLjE3MzM2NzEwNzguNTkuMC4w*\_ga\_HEWSDXWJSN*MTczMzY3MDc1Ni4yOC4xLjE3MzM2NzEwNzguMC4wLjA.}{Oxygen Styles Basket}

\item {} 
\sphinxAtStartPar
\sphinxhref{https://www.oxygenxml.com/doc/versions/27.0/ug-editor/topics/dcpp\_overview.html}{CSS Customization Webinar}

\end{enumerate}

\sphinxstepscope


\chapter{技术文档质量}
\label{\detokenize{doc-quality/doc-quality-intro:id1}}\label{\detokenize{doc-quality/doc-quality-intro::doc}}
\sphinxAtStartPar
技术文档的质量是一个系统工程,整体是由“人+流程+工具”来保证的。


\section{人}
\label{\detokenize{doc-quality/doc-quality-intro:id2}}
\sphinxAtStartPar
从事技术写作的人员应接受过相应的教育,这里可以是某个技术写作的本科专业,或者通过了系统化的培训来获得,例如 \sphinxhref{https://www.stc.org/certification/}{STC Certification}、\sphinxhref{https://www.technical-communication.org/technical-writing/tekom-certification}{Tekom Certification} 和中国标准化协会技术传播服务委员会均有相应的课程。


\section{流程}
\label{\detokenize{doc-quality/doc-quality-intro:id3}}
\sphinxAtStartPar
流程是做一件事情的基本步骤,例如我们可以参加 \sphinxhref{https://www.iso.org/standard/71620.html}{IEC 82079} 的流程规定。


\section{工具}
\label{\detokenize{doc-quality/doc-quality-intro:id4}}
\sphinxAtStartPar
使用各种工具降低人的负担,提高流程的可靠性。


\section{技术文档质量的层次}
\label{\detokenize{doc-quality/doc-quality-intro:id5}}\begin{itemize}
\item {} 
\sphinxAtStartPar
准确性

\item {} 
\sphinxAtStartPar
可用性

\item {} 
\sphinxAtStartPar
用户体验/信息体验

\end{itemize}

\sphinxAtStartPar
好的文档应能确保内容的准确性,在此基础之上要考虑可用性,最后考虑整体的信息体验。


\subsection{准确性}
\label{\detokenize{doc-quality/doc-quality-intro:id6}}
\sphinxAtStartPar
在确保内容准确性上,目前通常从如下几个方面做约束。


\subsubsection{内容结构标准}
\label{\detokenize{doc-quality/doc-quality-intro:id7}}
\sphinxAtStartPar
使用DITA将内容的结构和模块进行标准化。使用oXygen XML 基于 DTD 或 Schema 来 Validate DITA 内容,确保DITA的正确性。


\subsubsection{受限语言}
\label{\detokenize{doc-quality/doc-quality-intro:id8}}
\sphinxAtStartPar
使用受限语言降低内容的难度,确保内容的一致性。例如 \sphinxhref{https://asd-ste100.org/}{ASD\sphinxhyphen{}STE}语言。


\subsubsection{语言检测工具}
\label{\detokenize{doc-quality/doc-quality-intro:id9}}
\sphinxAtStartPar
文本的准确性通常可以从以下三个方面来检测。
\begin{itemize}
\item {} 
\sphinxAtStartPar
拼写

\item {} 
\sphinxAtStartPar
句法

\item {} 
\sphinxAtStartPar
风格

\end{itemize}


\begin{savenotes}\sphinxattablestart
\sphinxthistablewithglobalstyle
\centering
\begin{tabulary}{\linewidth}[t]{TTT}
\sphinxtoprule
\sphinxstyletheadfamily 
\sphinxAtStartPar
工具
&\sphinxstyletheadfamily 
\sphinxAtStartPar
性质
&\sphinxstyletheadfamily 
\sphinxAtStartPar
功能
\\
\sphinxmidrule
\sphinxtableatstartofbodyhook
\sphinxAtStartPar
\sphinxhref{https://languagetool.org/}{Langauge Tool}
&
\sphinxAtStartPar
开源+商业
&
\sphinxAtStartPar
可自定义规则检测文档风格
\\
\sphinxhline
\sphinxAtStartPar
\sphinxhref{http://proselint.com/}{ProseLint}
&
\sphinxAtStartPar
开源
&
\sphinxAtStartPar
可通过命令行检测文本质量
\\
\sphinxhline
\sphinxAtStartPar
\sphinxhref{https://www.acrolinx.com/}{Acrolinx}
&
\sphinxAtStartPar
商业
&
\sphinxAtStartPar
完整的文档质量检查解决方案
\\
\sphinxbottomrule
\end{tabulary}
\sphinxtableafterendhook\par
\sphinxattableend\end{savenotes}


\begin{savenotes}\sphinxattablestart
\sphinxthistablewithglobalstyle
\centering
\begin{tabulary}{\linewidth}[t]{TTT}
\sphinxtoprule
\sphinxstyletheadfamily 
\sphinxAtStartPar
指标
&\sphinxstyletheadfamily 
\sphinxAtStartPar
含义
&\sphinxstyletheadfamily 
\sphinxAtStartPar
计算方式
\\
\sphinxmidrule
\sphinxtableatstartofbodyhook
\sphinxAtStartPar
Readability
&
\sphinxAtStartPar
评价文档阅读的难易程度
&
\sphinxAtStartPar
\sphinxhref{https://readable.com/readability/flesch-reading-ease-flesch-kincaid-grade-level/}{文档}
\\
\sphinxhline
\sphinxAtStartPar
Coh\sphinxhyphen{}Metrix
&
\sphinxAtStartPar
评价文档的衔接程度
&
\sphinxAtStartPar
\sphinxhref{http://cohmetrix.memphis.edu/cohmetrixhome/documentation\_indices.html}{文档}
\\
\sphinxbottomrule
\end{tabulary}
\sphinxtableafterendhook\par
\sphinxattableend\end{savenotes}


\section{可用性}
\label{\detokenize{doc-quality/doc-quality-intro:id10}}
\sphinxAtStartPar
内容准确的资料用户未必觉得好用,重要的资料还需要做可用性测试,确保内容的可用性。

\sphinxAtStartPar
关注的指标:
\begin{itemize}
\item {} 
\sphinxAtStartPar
完成率。使用资料完成任务的概率。

\item {} 
\sphinxAtStartPar
效率。使用资料完成任务的时间。

\item {} 
\sphinxAtStartPar
满意度。大众使用资料后的满意程度。

\end{itemize}


\section{用户体验/信息体验设计}
\label{\detokenize{doc-quality/doc-quality-intro:id11}}
\sphinxAtStartPar
在确保了内容准确性和可用性的基础之上,则可以探索内容的信息体验,让用户阅读时有赏心悦目,喜闻乐见的同时,增强用户粘性 。

\sphinxAtStartPar
{[}{]}(Comment text goes here)

\sphinxstepscope


\chapter{风格指南与检查器}
\label{\detokenize{doc-quality/style-guide:id1}}\label{\detokenize{doc-quality/style-guide::doc}}
\sphinxAtStartPar
风格指南一般凝练了整个技术写作团队的多年的精华,成熟的风格指南可以让技术作家快速熟悉团队遣词造句以及行文风格的要求,确保内容的一致性。在做技术文档写作的时候,成熟的团队都有有自己的风格指南(style guide)。


\section{风格}
\label{\detokenize{doc-quality/style-guide:id2}}
\sphinxAtStartPar
在讲解风格指南之前,我们先说一说风格,因为有了风格才需要有指南。

\sphinxAtStartPar
假设有两个人:甲和乙进入咖啡厅点咖啡,他们分别是这样点咖啡的:

\sphinxAtStartPar
甲
\begin{quote}

\sphinxAtStartPar
Small Coffee

\sphinxAtStartPar
Extra light, no sugar.

\sphinxAtStartPar
I’m in a rush.
\end{quote}

\sphinxAtStartPar
乙
\begin{quote}

\sphinxAtStartPar
Hi, Cynthia. Can I get

\sphinxAtStartPar
A small coffee,

\sphinxAtStartPar
Extra light, no sugar?

\sphinxAtStartPar
Thanks.
\end{quote}

\sphinxAtStartPar
甲显得比较莽撞,听起来就比较忙,也很不耐烦,而乙则会让人觉得比较体贴,不着急,而且比较友好。语言是思想的外衣,不同的表达方式,会让听众形成不同的观感。

\sphinxAtStartPar
再以GE和MINI汽车为例,因为历史和品牌定位的不同,两个品牌有了自己的风格。


\begin{savenotes}\sphinxattablestart
\sphinxthistablewithglobalstyle
\centering
\begin{tabulary}{\linewidth}[t]{TTT}
\sphinxtoprule

\sphinxAtStartPar

&\sphinxstyletheadfamily 
\sphinxAtStartPar
GE
&\sphinxstyletheadfamily 
\sphinxAtStartPar
MINI
\\
\sphinxmidrule
\sphinxtableatstartofbodyhook
\sphinxAtStartPar
风格
&
\sphinxAtStartPar
\sphinxhyphen{} imaginative \sphinxhyphen{} responsible \sphinxhyphen{} inspirational
&
\sphinxAtStartPar
\sphinxhyphen{} Confident \sphinxhyphen{} Joyful \sphinxhyphen{} Rebellious
\\
\sphinxhline
\sphinxAtStartPar
文案
&
\sphinxAtStartPar
Ge works hard to make house a home. We’re determined to solve the world’s biggest problems.
&
\sphinxAtStartPar
You know that the dream you have. But in the vent of a late\sphinxhyphen{}night flat tire or wild armadillo attack
\\
\sphinxbottomrule
\end{tabulary}
\sphinxtableafterendhook\par
\sphinxattableend\end{savenotes}

\sphinxAtStartPar
GE的信条是“imagination at work”,他们总是想让世界知道,GE每天的工作会让整个世界收益。正是以为你这个目的,我们可以感觉到他们的文案也是想让受众觉得他们的业务非常重要。

\sphinxAtStartPar
MNI的人群定位是有生活的幽默感,反对世俗,想开心生活,所以文案上也能看出来这一点。

\sphinxAtStartPar
因为个性所以有了自己的风格,而风格又会影响一个人的讲话的方式。


\section{微软风格}
\label{\detokenize{doc-quality/style-guide:id3}}
\sphinxAtStartPar
不仅消费者公司有调性,其实高科技公司也有自己的调性。例如微软公司希望用户能感受到微软是:inspirational, responsible, empathetic, polite, supportive and encouraging,为了达到这个效果,微软也制定了具体的细则来指导工程师的写作。


\subsection{inspirational}
\label{\detokenize{doc-quality/style-guide:inspirational}}
\sphinxAtStartPar
要求:强调用户可以做哪些事情,而不是不能做哪些。鼓励用户是帮助用户实现潜能的重要组成部分,帮助用户解决问题,但是如果能有机会让用户尝试之前没有想到的功能也是非常好的。
\begin{quote}

\sphinxAtStartPar
You can add a personal touch to your computer by changing the computer’s theme, color, sounds, desktop background, screen saver, font size, and user account picture.
\end{quote}


\subsection{Responsible}
\label{\detokenize{doc-quality/style-guide:responsible}}
\sphinxAtStartPar
我们的责任不仅仅是一个伟大产品的公司,我们还应该思考用户会如何看待我们的评价。
\begin{quote}

\sphinxAtStartPar
Microsoft style

\sphinxAtStartPar
Free technical support is available when you register with Microsoft.
\end{quote}
\begin{quote}

\sphinxAtStartPar
Non Microsoft style

\sphinxAtStartPar
You must register with Microsoft to received free technical support.
\end{quote}


\subsection{Polite}
\label{\detokenize{doc-quality/style-guide:polite}}
\sphinxAtStartPar
不应该让用户觉得被冒犯,或者甲方高高在上。
\begin{quote}

\sphinxAtStartPar
Microsoft style

\sphinxAtStartPar
The file is protected and can’t be deleted without specific permission.
\end{quote}
\begin{quote}

\sphinxAtStartPar
Not Microsoft style

\sphinxAtStartPar
Can’t delete New Text Document: Access is denied.
\end{quote}

\sphinxAtStartPar
更多内容,大家可以参见 \sphinxhref{https://www.amazon.com/Microsoft-Manual-Style-4th-Corporation/dp/0735648719/ref=sr\_1\_1?keywords=Microsoft+Manual+of+Style\&amp;qid=1639042143\&amp;sr=8-1}{Microsoft Manual of Style}


\section{风格指南}
\label{\detokenize{doc-quality/style-guide:id4}}
\sphinxAtStartPar
常见的风格指南:
\begin{itemize}
\item {} 
\sphinxAtStartPar
\sphinxhref{https://docs.microsoft.com/en-us/style-guide/welcome/}{Microsoft Style Guide}

\item {} 
\sphinxAtStartPar
\sphinxhref{https://developers.google.com/style/}{Google Documentation Style Guide}

\item {} 
\sphinxAtStartPar
\sphinxhref{https://content-guide.18f.gov/our-style/}{18F Content Guide}

\item {} 
\sphinxAtStartPar
\sphinxhref{https://books.apple.com/us/book/apple-style-guide/id1161855204}{Apple Style Guide}

\end{itemize}


\section{Linter}
\label{\detokenize{doc-quality/style-guide:linter}}
\sphinxAtStartPar
风格指南一般需要搭配检查器共同工作,否则完全靠写作者的记忆,压力较大而且容易出错。

\sphinxAtStartPar
开发者在写代码的时候,常常需要依赖 \sphinxstylestrong{Linter}查找错误、冲突或代码风格等问题。在Docs as Code的开发模式中,也常常将文本当成代码来进行检查。

\sphinxAtStartPar
常见Linter
\begin{itemize}
\item {} 
\sphinxAtStartPar
\sphinxhref{https://alexjs.com/}{alex}

\item {} 
\sphinxAtStartPar
\sphinxhref{https://github.com/errata-ai}{Vale} | \sphinxhref{https://github.com/errata-ai/vale-vscode}{vale\sphinxhyphen{}vscode}

\item {} 
\sphinxAtStartPar
proselint

\item {} 
\sphinxAtStartPar
\sphinxhref{https://languagetool.org/}{languagetool}

\end{itemize}

\sphinxAtStartPar
可检查的规则
\begin{itemize}
\item {} 
\sphinxAtStartPar
\sphinxhref{https://github.com/errata-ai/Microsoft}{Vale 的 \sphinxstyleemphasis{Microsoft Writing Style Guide} 的检查规则}

\end{itemize}

\sphinxstepscope


\chapter{Vale}
\label{\detokenize{doc-quality/Vale:vale}}\label{\detokenize{doc-quality/Vale::doc}}
\sphinxAtStartPar
Vale是一个功能非常强大的 Prose Linter,除了检查markdown之外,还可以检查DITA等标记文本。


\section{如何使用}
\label{\detokenize{doc-quality/Vale:id1}}\begin{enumerate}
\sphinxsetlistlabels{\arabic}{enumi}{enumii}{}{.}%
\item {} 
\sphinxAtStartPar
安装 Vale \sphinxcode{\sphinxupquote{brew install vale}}

\item {} 
\sphinxAtStartPar
在需要检查的文件的根目录下创建 \sphinxcode{\sphinxupquote{.vale.ini}}

\item {} 
\sphinxAtStartPar
在  \sphinxcode{\sphinxupquote{.vale.ini}} 中输入规则所在路径,以及需要启用的规则

\begin{sphinxVerbatim}[commandchars=\\\{\}]
\PYG{n}{StylesPath} \PYG{o}{=} \PYG{n}{vale}\PYG{o}{\PYGZhy{}}\PYG{n}{styles}\PYG{o}{/}
\PYG{n}{MinAlertLevel} \PYG{o}{=} \PYG{n}{error}

\PYG{p}{[}\PYG{o}{*}\PYG{p}{]}
\PYG{n}{BasedOnStyles} \PYG{o}{=} \PYG{n}{write}\PYG{o}{\PYGZhy{}}\PYG{n}{good}
\PYG{n}{vale}\PYG{o}{.}\PYG{n}{Editorializing} \PYG{o}{=} \PYG{n}{YES}
\PYG{n}{vale}\PYG{o}{.}\PYG{n}{Hedging} \PYG{o}{=} \PYG{n}{error}
\end{sphinxVerbatim}

\end{enumerate}


\section{创建规则}
\label{\detokenize{doc-quality/Vale:id2}}

\subsection{规则解读}
\label{\detokenize{doc-quality/Vale:id3}}
\sphinxAtStartPar
一般技术文档中,不推荐使用被动语态,如果希望linter检查此规则,则可以使用如下规则。
\begin{enumerate}
\sphinxsetlistlabels{\arabic}{enumi}{enumii}{}{.}%
\item {} 
\sphinxAtStartPar
识别出be动词

\item {} 
\sphinxAtStartPar
识别be动词后所跟词汇(tokens)

\end{enumerate}

\sphinxAtStartPar
一旦匹配,即会给用户提示错误,并告诉修改的方式。

\begin{sphinxVerbatim}[commandchars=\\\{\}]
extends: existence
message: \PYGZdq{}\PYGZsq{}\PYGZpc{}s\PYGZsq{} may be passive voice. Use active voice if you can.\PYGZdq{}
ignorecase: true
level: warning
raw:
  \PYGZhy{} \PYGZbs{}b(am|are|were|being|is|been|was|be)\PYGZbs{}b\PYGZbs{}s*
tokens:
  \PYGZhy{} \PYGZsq{}[\PYGZbs{}w]+ed\PYGZsq{}
  \PYGZhy{} awoken
  \PYGZhy{} beat
  \PYGZhy{} become
  \PYGZhy{} been
  \PYGZhy{} begun
  \PYGZhy{} bent
  \PYGZhy{} beset
  \PYGZhy{} bet
  \PYGZhy{} bid
  \PYGZhy{} bidden
  \PYGZhy{} bitten
  \PYGZhy{} bled
  \PYGZhy{} blown
  \PYGZhy{} born
  \PYGZhy{} bought
  \PYGZhy{} bound
  \PYGZhy{} bred
  \PYGZhy{} broadcast
  \PYGZhy{} broken
  \PYGZhy{} brought
  \PYGZhy{} built
  \PYGZhy{} burnt
  \PYGZhy{} burst
  \PYGZhy{} cast
  \PYGZhy{} caught
  \PYGZhy{} chosen
  \PYGZhy{} clung
  \PYGZhy{} come
  \PYGZhy{} cost
  \PYGZhy{} crept
  \PYGZhy{} cut
  \PYGZhy{} dealt
  \PYGZhy{} dived
  \PYGZhy{} done
  \PYGZhy{} drawn
  \PYGZhy{} dreamt
  \PYGZhy{} driven
  \PYGZhy{} drunk
  \PYGZhy{} dug
  \PYGZhy{} eaten
  \PYGZhy{} fallen
  \PYGZhy{} fed
  \PYGZhy{} felt
  \PYGZhy{} fit
  \PYGZhy{} fled
  \PYGZhy{} flown
  \PYGZhy{} flung
  \PYGZhy{} forbidden
  \PYGZhy{} foregone
  \PYGZhy{} forgiven
  \PYGZhy{} forgotten
  \PYGZhy{} forsaken
  \PYGZhy{} fought
  \PYGZhy{} found
  \PYGZhy{} frozen
  \PYGZhy{} given
  \PYGZhy{} gone
  \PYGZhy{} gotten
  \PYGZhy{} ground
  \PYGZhy{} grown
  \PYGZhy{} heard
  \PYGZhy{} held
  \PYGZhy{} hidden
  \PYGZhy{} hit
  \PYGZhy{} hung
  \PYGZhy{} hurt
  \PYGZhy{} kept
  \PYGZhy{} knelt
  \PYGZhy{} knit
  \PYGZhy{} known
  \PYGZhy{} laid
  \PYGZhy{} lain
  \PYGZhy{} leapt
  \PYGZhy{} learnt
  \PYGZhy{} led
  \PYGZhy{} left
  \PYGZhy{} lent
  \PYGZhy{} let
  \PYGZhy{} lighted
  \PYGZhy{} lost
  \PYGZhy{} made
  \PYGZhy{} meant
  \PYGZhy{} met
  \PYGZhy{} misspelt
  \PYGZhy{} mistaken
  \PYGZhy{} mown
  \PYGZhy{} overcome
  \PYGZhy{} overdone
  \PYGZhy{} overtaken
  \PYGZhy{} overthrown
  \PYGZhy{} paid
  \PYGZhy{} pled
  \PYGZhy{} proven
  \PYGZhy{} put
(为节约空间,省去更多词表)
\end{sphinxVerbatim}


\subsection{创建规则}
\label{\detokenize{doc-quality/Vale:id4}}

\subsubsection{Vale Studio}
\label{\detokenize{doc-quality/Vale:vale-studio}}
\sphinxAtStartPar
Vale 提供了一个可视化的规则编辑器 \sphinxhref{https://vale-studio.errata.ai/}{Vale Studio}。


\subsubsection{基本语法}
\label{\detokenize{doc-quality/Vale:id5}}
\begin{sphinxVerbatim}[commandchars=\\\{\}]
\PYG{c+c1}{\PYGZsh{} All rules should define the following header keys:}
\PYG{c+c1}{\PYGZsh{}}
\PYG{c+c1}{\PYGZsh{} `extends` indicates the extension point being used (see below for information}
\PYG{c+c1}{\PYGZsh{} on the possible values).}
\PYG{n+nt}{extends}\PYG{p}{:}\PYG{+w}{ }\PYG{l+lScalar+lScalarPlain}{existence}
\PYG{c+c1}{\PYGZsh{} `message` is shown to the user when the rule is broken.}
\PYG{c+c1}{\PYGZsh{}}
\PYG{c+c1}{\PYGZsh{} Many extension points accept format specifiers (\PYGZpc{}s), which are replaced by}
\PYG{c+c1}{\PYGZsh{} extracted values. See the exention\PYGZhy{}specific sections below for more details.}
\PYG{n+nt}{message}\PYG{p}{:}\PYG{+w}{ }\PYG{l+s}{\PYGZdq{}}\PYG{l+s}{Consider}\PYG{n+nv}{ }\PYG{l+s}{removing}\PYG{n+nv}{ }\PYG{l+s}{\PYGZsq{}\PYGZpc{}s\PYGZsq{}}\PYG{l+s}{\PYGZdq{}}
\PYG{c+c1}{\PYGZsh{} `level` assigns the rule\PYGZsq{}s severity.}
\PYG{c+c1}{\PYGZsh{}}
\PYG{c+c1}{\PYGZsh{} The accepted values are suggestion, warning, and error.}
\PYG{n+nt}{level}\PYG{p}{:}\PYG{+w}{ }\PYG{l+lScalar+lScalarPlain}{warning}
\PYG{c+c1}{\PYGZsh{} `scope` specifies where this rule should apply \PYGZhy{}\PYGZhy{} e.g., headings, sentences, etc.}
\PYG{c+c1}{\PYGZsh{}}
\PYG{c+c1}{\PYGZsh{} See the Markup section for more information on scoping.}
\PYG{n+nt}{scope}\PYG{p}{:}\PYG{+w}{ }\PYG{l+lScalar+lScalarPlain}{heading}
\PYG{c+c1}{\PYGZsh{} `code` determines whether or not the content of code spans \PYGZhy{}\PYGZhy{} e.g., `foo` for}
\PYG{c+c1}{\PYGZsh{} Markdown \PYGZhy{}\PYGZhy{} is ignored.}
\PYG{n+nt}{code}\PYG{p}{:}\PYG{+w}{ }\PYG{l+lScalar+lScalarPlain}{false}
\PYG{c+c1}{\PYGZsh{} `link` gives the source for this rule.}
\PYG{n+nt}{link}\PYG{p}{:}\PYG{+w}{ }\PYG{l+s}{\PYGZsq{}}\PYG{l+s}{https://errata.ai/}\PYG{l+s}{\PYGZsq{}}
\PYG{c+c1}{\PYGZsh{} The number of times this rule should raise an alert.}
\PYG{c+c1}{\PYGZsh{}}
\PYG{c+c1}{\PYGZsh{} By default, there is no limit.}
\PYG{n+nt}{limit}\PYG{p}{:}\PYG{+w}{ }\PYG{l+lScalar+lScalarPlain}{1}
\end{sphinxVerbatim}


\subsubsection{规则示例}
\label{\detokenize{doc-quality/Vale:id6}}
\sphinxAtStartPar
PingCAP公司是一个知名数据库公司,不少外部人士常常将公司名称写为pingcap或Pingcap等,需要一个规则文本进行检查,确保所有的文本或代码中,均使用了 PingCAP 的正确写法。

\sphinxAtStartPar
在 Styles 中新建PingCAP文件夹,并在其中新建 \sphinxcode{\sphinxupquote{branding.yml}} 文件,并在其中输入下方规则

\begin{sphinxVerbatim}[commandchars=\\\{\}]
\PYG{n+nn}{\PYGZhy{}\PYGZhy{}\PYGZhy{}}
\PYG{n+nt}{extends}\PYG{p}{:}\PYG{+w}{ }\PYG{l+lScalar+lScalarPlain}{substitution}
\PYG{n+nt}{message}\PYG{p}{:}\PYG{+w}{ }\PYG{l+s}{\PYGZdq{}}\PYG{l+s}{使用}\PYG{n+nv}{ }\PYG{l+s}{\PYGZsq{}\PYGZpc{}s\PYGZsq{}}\PYG{n+nv}{ }\PYG{l+s}{,而非}\PYG{n+nv}{ }\PYG{l+s}{\PYGZsq{}\PYGZpc{}s\PYGZsq{}}\PYG{l+s}{\PYGZdq{}}
\PYG{n+nt}{level}\PYG{p}{:}\PYG{+w}{ }\PYG{l+lScalar+lScalarPlain}{error}
\PYG{n+nt}{ignorecase}\PYG{p}{:}\PYG{+w}{ }\PYG{l+lScalar+lScalarPlain}{false}
\PYG{c+c1}{\PYGZsh{} swap maps tokens in form of bad: good}
\PYG{n+nt}{swap}\PYG{p}{:}
\PYG{+w}{  }\PYG{n+nt}{pingcap}\PYG{p}{:}\PYG{+w}{ }\PYG{l+lScalar+lScalarPlain}{PingCAP}
\PYG{+w}{  }\PYG{n+nt}{Pingcap}\PYG{p}{:}\PYG{+w}{ }\PYG{l+lScalar+lScalarPlain}{PingCAP}
\PYG{+w}{  }\PYG{n+nt}{PingCap}\PYG{p}{:}\PYG{+w}{ }\PYG{l+lScalar+lScalarPlain}{PingCAP}
\end{sphinxVerbatim}

\sphinxAtStartPar
在VS Code中的检测效果

\sphinxAtStartPar
\sphinxincludegraphics{{PingCAP-demo}.png}


\section{阅读材料}
\label{\detokenize{doc-quality/Vale:id7}}\begin{enumerate}
\sphinxsetlistlabels{\arabic}{enumi}{enumii}{}{.}%
\item {} 
\sphinxAtStartPar
\sphinxhref{https://errata-ai.github.io/vale-server/docs/style}{Vale Style 教程}

\item {} 
\sphinxAtStartPar
正则表达式教程 \sphinxhref{https://regex101.com/r/NxwiwQ/1}{Regular expression 101}

\end{enumerate}


\section{文件下载}
\label{\detokenize{doc-quality/Vale:id8}}\begin{enumerate}
\sphinxsetlistlabels{\arabic}{enumi}{enumii}{}{.}%
\item {} 
\sphinxAtStartPar
Sample.md 下载

\end{enumerate}

\sphinxstepscope


\chapter{可用性测试}
\label{\detokenize{content_test/usability_testing:id1}}\label{\detokenize{content_test/usability_testing::doc}}

\section{可用性指标}
\label{\detokenize{content_test/usability_testing:id2}}

\subsection{Effectiveness}
\label{\detokenize{content_test/usability_testing:effectiveness}}
\sphinxAtStartPar
用户能不能完成任务?

\sphinxAtStartPar
计算方式:


\section{Efficiency}
\label{\detokenize{content_test/usability_testing:efficiency}}
\sphinxAtStartPar
用户完成任务的效率


\section{满意度}
\label{\detokenize{content_test/usability_testing:id3}}
\sphinxAtStartPar
用户完成任务后的满意程度


\section{可用性实验}
\label{\detokenize{content_test/usability_testing:id4}}

\subsection{实验流程}
\label{\detokenize{content_test/usability_testing:id5}}\begin{enumerate}
\sphinxsetlistlabels{\arabic}{enumi}{enumii}{}{.}%
\item {} 
\sphinxAtStartPar
Develop the Test Plan

\item {} 
\sphinxAtStartPar
Setup a testing environment

\item {} 
\sphinxAtStartPar
Find and select participants

\item {} 
\sphinxAtStartPar
Prepare test materials

\item {} 
\sphinxAtStartPar
Conduct the test sessions

\item {} 
\sphinxAtStartPar
Debrief the participant and observers

\item {} 
\sphinxAtStartPar
Analyze data and observations

\item {} 
\sphinxAtStartPar
Report findings and recommendations

\end{enumerate}


\subsubsection{测试计划}
\label{\detokenize{content_test/usability_testing:id6}}\begin{itemize}
\item {} 
\sphinxAtStartPar
Purpose, goals and objectives of the test

\item {} 
\sphinxAtStartPar
Research questions

\item {} 
\sphinxAtStartPar
Participant characteristics

\item {} 
\sphinxAtStartPar
Method

\item {} 
\sphinxAtStartPar
Task list

\item {} 
\sphinxAtStartPar
Test environment, equipment, and logistics

\item {} 
\sphinxAtStartPar
Test moderator role

\item {} 
\sphinxAtStartPar
Data to be collected and evaluation measures

\item {} 
\sphinxAtStartPar
Report contents and presentation

\end{itemize}


\section{Sphinx 文档可用性测试}
\label{\detokenize{content_test/usability_testing:sphinx}}

\subsection{环境准备}
\label{\detokenize{content_test/usability_testing:id7}}\begin{enumerate}
\sphinxsetlistlabels{\arabic}{enumi}{enumii}{}{.}%
\item {} 
\sphinxAtStartPar
使用腾讯会议录制桌面与人像

\item {} 
\sphinxAtStartPar
使用 \sphinxhref{https://www.userfocus.co.uk/resources/datalogger.html}{Usability Test Data Logger tool v5.1.1} 记录与分析数据

\end{enumerate}


\subsection{测试任务}
\label{\detokenize{content_test/usability_testing:id8}}\begin{enumerate}
\sphinxsetlistlabels{\arabic}{enumi}{enumii}{}{.}%
\item {} 
\sphinxAtStartPar
安装sphinx。成功标志:

\begin{sphinxVerbatim}[commandchars=\\\{\}]
 \PYGZdl{} sphinx\PYGZhy{}quickstart \PYGZhy{}\PYGZhy{}version
 sphinx\PYGZhy{}quickstart 4.0.1
\end{sphinxVerbatim}

\item {} 
\sphinxAtStartPar
运行quickstart,写一个rST文件 (Title, Section Title, Bold)

\item {} 
\sphinxAtStartPar
将整个项目发布为 html 站点

\end{enumerate}


\subsection{分析结果}
\label{\detokenize{content_test/usability_testing:id9}}\begin{enumerate}
\sphinxsetlistlabels{\arabic}{enumi}{enumii}{}{.}%
\item {} 
\sphinxAtStartPar
使用 UTDL 登录各测试数据

\item {} 
\sphinxAtStartPar
观察视频,观察其中用户完成任务的过程中遇到的问题

\item {} 
\sphinxAtStartPar
数据结果的分析并用于文档的改进

\end{enumerate}


\section{参考资料}
\label{\detokenize{content_test/usability_testing:id10}}\begin{enumerate}
\sphinxsetlistlabels{\arabic}{enumi}{enumii}{}{.}%
\item {} 
\sphinxAtStartPar
\sphinxhref{https://www.userfocus.co.uk/articles/datalogging.html}{Log usability tests like a pro}

\end{enumerate}

\sphinxstepscope


\chapter{EndNote 文档可用性测试小组作业}
\label{\detokenize{capstone-project/endnote-doc-usability:endnote}}\label{\detokenize{capstone-project/endnote-doc-usability::doc}}
\sphinxAtStartPar
写在前面

\sphinxAtStartPar
技术文档是帮助用户用产品的,虽然目前用户使用文档学习软件产品的心智还没有形成,但是可用性测试这个方法是通用的。依据最小化文档设计的理论,新手不看说明,或者喜欢通过视频教程入门,这样的话在视频说明书录制后,依然需要对视频教程做可用性测试,确保最终的设计能满足设计目标,并在测试过程中发现文档问题,从而能及时改进设计。


\section{作业要求}
\label{\detokenize{capstone-project/endnote-doc-usability:id1}}\begin{enumerate}
\sphinxsetlistlabels{\arabic}{enumi}{enumii}{}{.}%
\item {} 
\sphinxAtStartPar
小组作业,大家分工完成可用性测试的报告;

\item {} 
\sphinxAtStartPar
下方为完成作业的建议,可以自由发挥。

\item {} 
\sphinxAtStartPar
截止时间:
\begin{itemize}
\item {} 
\sphinxAtStartPar
软院:2021年12月21日 上午8点,课堂派提交

\item {} 
\sphinxAtStartPar
外院:2021年12月23日 上午8点,课堂派提交

\end{itemize}

\end{enumerate}

\sphinxAtStartPar
因为时间限制,建议大家既做被试,又作为可用性测试研究人员,建议的完成步骤如下:


\section{自己作为被试}
\label{\detokenize{capstone-project/endnote-doc-usability:id2}}

\subsection{测试前的准备:}
\label{\detokenize{capstone-project/endnote-doc-usability:id3}}\begin{enumerate}
\sphinxsetlistlabels{\arabic}{enumi}{enumii}{}{.}%
\item {} 
\sphinxAtStartPar
\sphinxhref{http://software.pku.edu.cn}{通过北大正版软件共享平台} ,下载EndNote 9,并完成激活;

\item {} 
\sphinxAtStartPar
安装腾讯会议,并通过腾讯会议录制操作和被试表情;

\item {} 
\sphinxAtStartPar
找到相对安静的地方,并熟悉一下 Think aloud方法

\end{enumerate}


\subsection{实验前的说明}
\label{\detokenize{capstone-project/endnote-doc-usability:id4}}\begin{enumerate}
\sphinxsetlistlabels{\arabic}{enumi}{enumii}{}{.}%
\item {} 
\sphinxAtStartPar
测试的是文档,而非同学的学习能力,大家的表现会用于改进文档;

\item {} 
\sphinxAtStartPar
被试视频仅供课堂实验之用,不会外传;

\end{enumerate}


\subsection{开始实验}
\label{\detokenize{capstone-project/endnote-doc-usability:id5}}\begin{enumerate}
\sphinxsetlistlabels{\arabic}{enumi}{enumii}{}{.}%
\item {} 
\sphinxAtStartPar
启动腾讯会议,录制屏幕和人像,全程录制,完成任务时think aloud 直至完成全部任务;

\item {} 
\sphinxAtStartPar
阅读 Help > Getting Started with Endnote 和 Help > Online User Guide 两个手册

\item {} 
\sphinxAtStartPar
依次完成如下任务(做任务过程中可以随时参阅上方手册,但不可看其他材料):

\sphinxAtStartPar
\sphinxstylestrong{任务1}: 	下载文献 \sphinxhref{https://dl.acm.org/doi/10.1145/3328020.3353936}{Designing metrics to evaluate the help center of Baidu cloud} 的PDF,并通过 \sphinxcode{\sphinxupquote{”}} (Export Citation) 功能导出EndNote的适用的文献数据(.enw)。

\sphinxAtStartPar
\sphinxincludegraphics{{download-article}.png}

\sphinxAtStartPar
\sphinxstylestrong{任务2}:  将任务1中下载的“文献引用”和 PDF 论文导入到 Endnote 的 Library中;

\sphinxAtStartPar
\sphinxstylestrong{任务3}:  将文献引用插入到 Word 中。与下方效果一致即为成功:

\sphinxAtStartPar
\sphinxincludegraphics{{insert-sucess}.png}

\end{enumerate}


\section{作为可用性研究团队}
\label{\detokenize{capstone-project/endnote-doc-usability:id6}}\begin{enumerate}
\sphinxsetlistlabels{\arabic}{enumi}{enumii}{}{.}%
\item {} 
\sphinxAtStartPar
以小组成员或其他组的录屏作为研究样本(N>=5);

\item {} 
\sphinxAtStartPar
使用 Usability Datalogger,模拟可用性实验的实施过程,即:观察同学们的录屏,录入上方各被试在完成各任务时的:完成情况、完成时间、完成任务时所犯错误、完成任务的满意度(可使用SEQ量表)等

\item {} 
\sphinxAtStartPar
形成 Endnote文档的可用性分值 (\sphinxhref{https://measuringu.com/sum-2/}{SUM值}),分析EndNote文档的可用性问题,并形成测试报告,以及你的改进建议。

\end{enumerate}


\subsection{作业工具包:}
\label{\detokenize{capstone-project/endnote-doc-usability:id7}}\begin{enumerate}
\sphinxsetlistlabels{\arabic}{enumi}{enumii}{}{.}%
\item {} 
\sphinxAtStartPar
\sphinxhref{https://www.userfocus.co.uk/resources/datalogger.html}{Usability Test Data Logger tool v5.1.1}

\item {} 
\sphinxAtStartPar
\sphinxhref{https://www.userfocus.co.uk/articles/testplan.html}{Usability Test Plan Toolkit}

\item {} 
\sphinxAtStartPar
\sphinxhref{https://www.userfocus.co.uk/articles/datalogging.html}{UX Observation Coding}

\item {} 
\sphinxAtStartPar
\sphinxhref{https://www.usability.gov/how-to-and-tools/resources/templates.html}{各种UX测试模板}

\end{enumerate}

\sphinxstepscope


\chapter{《技术写作实用教程》写作说明}
\label{\detokenize{capstone-project/book-authoring:id1}}\label{\detokenize{capstone-project/book-authoring::doc}}

\section{作业要求}
\label{\detokenize{capstone-project/book-authoring:id2}}\begin{enumerate}
\sphinxsetlistlabels{\arabic}{enumi}{enumii}{}{.}%
\item {} 
\sphinxAtStartPar
每人认领 \sphinxhref{https://tw.gaozhijun.me/about/tw-comp-model.html}{“技术写作从业能力要求”} 的一个章节;

\item {} 
\sphinxAtStartPar
写作前阅读课程课件、自己的学习笔记、其他相关教程和论文;

\item {} 
\sphinxAtStartPar
拟定写作内容的提纲,可以是思维导图或Excel表的形式,然后约时间与老师讨论,初步明确写作内容后即可上手写作,后续依然可以动态调整;

\item {} 
\sphinxAtStartPar
使用DITA语言写作,写作工具不限(oXygen, XMLMind,Codex)都可以,写作后提交全部的dita工程文件,用于后续PDF和webhelp的发布;

\item {} 
\sphinxAtStartPar
写作之前,参照教 \sphinxhref{https://tw.gaozhijun.me/about/style.html}{“教材写作风格指南”} 的规定(如截图尺寸,文件名命名方式等);

\item {} 
\sphinxAtStartPar
提交方式:课堂派,截止日期 2022年1月25日 17:00;

\end{enumerate}

\begin{sphinxadmonition}{note}{备注:}
\sphinxAtStartPar
期待大家的作品!

\sphinxAtStartPar
有问题请随时和我联系。微信或邮箱均可。
\end{sphinxadmonition}


\section{为什么写这本教程}
\label{\detokenize{capstone-project/book-authoring:id5}}

\subsection{教学相长}
\label{\detokenize{capstone-project/book-authoring:id6}}
\sphinxAtStartPar
写这本书最核心的目的,还是让大家将本学期所学内容系统性地整理与回顾一遍。平时授课,因为时间有限,不少内容只能蜻蜓点水过一遍,大家可以利用寒假时间,就自己感兴趣的话题深入探究一下。一种增加自己对某个知识了解的方法,就是去尝试教别人,在教授的过程中,你的各种准备会促使你深入学习某个知识,在讲解的过程中你会发现很多自己原来没有意识到的视角,这又会进一步拓宽自己的知识面。

\sphinxAtStartPar
本书要求大家使用DITA创作,这个过程又能增加大家对DITA标记的了解,同时我们使用 \sphinxtitleref{Docs as Code} 的模式,过程中会用到很多新技术来确保教材的质量,相信这一创新型的写作模式,会让大家有很多新的收获。


\subsection{填补空白}
\label{\detokenize{capstone-project/book-authoring:id7}}
\sphinxAtStartPar
目前我国市面上还没有中文的技术写作教程,已经发行的几本技术写作教程都是英文的,而且更多的偏向实用写作或一般性介绍,我们这本书将可以弥补这一不足,我们以中文写作,面向应用,读者将能学习到入门技术写作行业所需的必要知识。这本书将能大大推动技术写作在我国的发展,为国家培养更多的技术写作人才,同时高质量的技术信息又可以反哺我们的研发体系,促进国家的工业化进程。


\section{教程的特色}
\label{\detokenize{capstone-project/book-authoring:id8}}
\sphinxAtStartPar
图书完成后,将会有如下创新:


\subsection{开源}
\label{\detokenize{capstone-project/book-authoring:id9}}
\sphinxAtStartPar
这本书将会是开源的(遵循  \sphinxhref{https://creativecommons.org/licenses/by-nc/4.0/deed.zh}{The Creative Commons Attribution\sphinxhyphen{}NonCommercial 4.0 Unported License}  协议),未来的读者可以自由的使用本教材进行学习,学生或拟转行做技术写作的同学,均能以此为基石。


\subsection{活教材}
\label{\detokenize{capstone-project/book-authoring:id10}}
\sphinxAtStartPar
对教材的刻板印象,就是出版社制作的书籍,对于大部分人来说,从写作到发行,如果没有出版社的协助,是比较困难的。当下,大部分作者用Word写作,写作后由出版社请美工排版,最后印刷后供读者购买使用。整个过程太长,极端情况下内容刚出版就已经过时了。我们目前尝试使用技术写作的工具链,大家可以使用所学知识,完成数字出版,而且教材可以边写边用,并能即时将读者的反馈更新到教程中,让教程一直保持旺盛的生命力。

\sphinxAtStartPar
大家可以DIY一切,自己动手丰衣足食。其实教材本来就是作者和读者的事情,完全没有必要通过出版社,这个模式说不定可以给未来的其他老师提供一些新思路。


\subsection{同侪写作}
\label{\detokenize{capstone-project/book-authoring:id11}}
\sphinxAtStartPar
这一本由学生写给学生 (by students for students),由学生进行同侪校阅 (Peer Review)的图书。本班同学,既自己创作又校阅他人书稿,一边输出知识,一边学习其他同学的知识。

\sphinxstepscope


\chapter{教材写作风格指南}
\label{\detokenize{about/style:id1}}\label{\detokenize{about/style::doc}}
\sphinxAtStartPar
(不断更新中)


\section{格式}
\label{\detokenize{about/style:id2}}

\subsection{文件名}
\label{\detokenize{about/style:id3}}\begin{enumerate}
\sphinxsetlistlabels{\arabic}{enumi}{enumii}{}{.}%
\item {} 
\sphinxAtStartPar
文件名命名时,采用标题的英译,单词之间使用”\sphinxhyphen{}“连接,文件名中不得包含空格。
\begin{quote}
\end{quote}

\end{enumerate}


\subsection{图片}
\label{\detokenize{about/style:id4}}\begin{enumerate}
\sphinxsetlistlabels{\arabic}{enumi}{enumii}{}{.}%
\item {} 
\sphinxAtStartPar
截图。在截图时,考虑全屏和局部;

\item {} 
\sphinxAtStartPar
尺寸。待定

\item {} 
\sphinxAtStartPar
存放路径。图片统一放入 \sphinxtitleref{images} 文件夹

\item {} 
\sphinxAtStartPar
图片命名时,采取图片 \sphinxtitleref{Caption} 英译,规则同上方文件名命名规则。

\end{enumerate}



\renewcommand{\indexname}{索引}
\printindex
\end{document}